
% Default to the notebook output style

    


% Inherit from the specified cell style.




    
\documentclass[11pt]{article}

    
    
    \usepackage[T1]{fontenc}
    % Nicer default font (+ math font) than Computer Modern for most use cases
    \usepackage{mathpazo}

    % Basic figure setup, for now with no caption control since it's done
    % automatically by Pandoc (which extracts ![](path) syntax from Markdown).
    \usepackage{graphicx}
    % We will generate all images so they have a width \maxwidth. This means
    % that they will get their normal width if they fit onto the page, but
    % are scaled down if they would overflow the margins.
    \makeatletter
    \def\maxwidth{\ifdim\Gin@nat@width>\linewidth\linewidth
    \else\Gin@nat@width\fi}
    \makeatother
    \let\Oldincludegraphics\includegraphics
    % Set max figure width to be 80% of text width, for now hardcoded.
    \renewcommand{\includegraphics}[1]{\Oldincludegraphics[width=.8\maxwidth]{#1}}
    % Ensure that by default, figures have no caption (until we provide a
    % proper Figure object with a Caption API and a way to capture that
    % in the conversion process - todo).
    \usepackage{caption}
    \DeclareCaptionLabelFormat{nolabel}{}
    \captionsetup{labelformat=nolabel}

    \usepackage{adjustbox} % Used to constrain images to a maximum size 
    \usepackage{xcolor} % Allow colors to be defined
    \usepackage{enumerate} % Needed for markdown enumerations to work
    \usepackage{geometry} % Used to adjust the document margins
    \usepackage{amsmath} % Equations
    \usepackage{amssymb} % Equations
    \usepackage{textcomp} % defines textquotesingle
    % Hack from http://tex.stackexchange.com/a/47451/13684:
    \AtBeginDocument{%
        \def\PYZsq{\textquotesingle}% Upright quotes in Pygmentized code
    }
    \usepackage{upquote} % Upright quotes for verbatim code
    \usepackage{eurosym} % defines \euro
    \usepackage[mathletters]{ucs} % Extended unicode (utf-8) support
    \usepackage[utf8x]{inputenc} % Allow utf-8 characters in the tex document
    \usepackage{fancyvrb} % verbatim replacement that allows latex
    \usepackage{grffile} % extends the file name processing of package graphics 
                         % to support a larger range 
    % The hyperref package gives us a pdf with properly built
    % internal navigation ('pdf bookmarks' for the table of contents,
    % internal cross-reference links, web links for URLs, etc.)
    \usepackage{hyperref}
    \usepackage{longtable} % longtable support required by pandoc >1.10
    \usepackage{booktabs}  % table support for pandoc > 1.12.2
    \usepackage[inline]{enumitem} % IRkernel/repr support (it uses the enumerate* environment)
    \usepackage[normalem]{ulem} % ulem is needed to support strikethroughs (\sout)
                                % normalem makes italics be italics, not underlines
    

    
    
    % Colors for the hyperref package
    \definecolor{urlcolor}{rgb}{0,.145,.698}
    \definecolor{linkcolor}{rgb}{.71,0.21,0.01}
    \definecolor{citecolor}{rgb}{.12,.54,.11}

    % ANSI colors
    \definecolor{ansi-black}{HTML}{3E424D}
    \definecolor{ansi-black-intense}{HTML}{282C36}
    \definecolor{ansi-red}{HTML}{E75C58}
    \definecolor{ansi-red-intense}{HTML}{B22B31}
    \definecolor{ansi-green}{HTML}{00A250}
    \definecolor{ansi-green-intense}{HTML}{007427}
    \definecolor{ansi-yellow}{HTML}{DDB62B}
    \definecolor{ansi-yellow-intense}{HTML}{B27D12}
    \definecolor{ansi-blue}{HTML}{208FFB}
    \definecolor{ansi-blue-intense}{HTML}{0065CA}
    \definecolor{ansi-magenta}{HTML}{D160C4}
    \definecolor{ansi-magenta-intense}{HTML}{A03196}
    \definecolor{ansi-cyan}{HTML}{60C6C8}
    \definecolor{ansi-cyan-intense}{HTML}{258F8F}
    \definecolor{ansi-white}{HTML}{C5C1B4}
    \definecolor{ansi-white-intense}{HTML}{A1A6B2}

    % commands and environments needed by pandoc snippets
    % extracted from the output of `pandoc -s`
    \providecommand{\tightlist}{%
      \setlength{\itemsep}{0pt}\setlength{\parskip}{0pt}}
    \DefineVerbatimEnvironment{Highlighting}{Verbatim}{commandchars=\\\{\}}
    % Add ',fontsize=\small' for more characters per line
    \newenvironment{Shaded}{}{}
    \newcommand{\KeywordTok}[1]{\textcolor[rgb]{0.00,0.44,0.13}{\textbf{{#1}}}}
    \newcommand{\DataTypeTok}[1]{\textcolor[rgb]{0.56,0.13,0.00}{{#1}}}
    \newcommand{\DecValTok}[1]{\textcolor[rgb]{0.25,0.63,0.44}{{#1}}}
    \newcommand{\BaseNTok}[1]{\textcolor[rgb]{0.25,0.63,0.44}{{#1}}}
    \newcommand{\FloatTok}[1]{\textcolor[rgb]{0.25,0.63,0.44}{{#1}}}
    \newcommand{\CharTok}[1]{\textcolor[rgb]{0.25,0.44,0.63}{{#1}}}
    \newcommand{\StringTok}[1]{\textcolor[rgb]{0.25,0.44,0.63}{{#1}}}
    \newcommand{\CommentTok}[1]{\textcolor[rgb]{0.38,0.63,0.69}{\textit{{#1}}}}
    \newcommand{\OtherTok}[1]{\textcolor[rgb]{0.00,0.44,0.13}{{#1}}}
    \newcommand{\AlertTok}[1]{\textcolor[rgb]{1.00,0.00,0.00}{\textbf{{#1}}}}
    \newcommand{\FunctionTok}[1]{\textcolor[rgb]{0.02,0.16,0.49}{{#1}}}
    \newcommand{\RegionMarkerTok}[1]{{#1}}
    \newcommand{\ErrorTok}[1]{\textcolor[rgb]{1.00,0.00,0.00}{\textbf{{#1}}}}
    \newcommand{\NormalTok}[1]{{#1}}
    
    % Additional commands for more recent versions of Pandoc
    \newcommand{\ConstantTok}[1]{\textcolor[rgb]{0.53,0.00,0.00}{{#1}}}
    \newcommand{\SpecialCharTok}[1]{\textcolor[rgb]{0.25,0.44,0.63}{{#1}}}
    \newcommand{\VerbatimStringTok}[1]{\textcolor[rgb]{0.25,0.44,0.63}{{#1}}}
    \newcommand{\SpecialStringTok}[1]{\textcolor[rgb]{0.73,0.40,0.53}{{#1}}}
    \newcommand{\ImportTok}[1]{{#1}}
    \newcommand{\DocumentationTok}[1]{\textcolor[rgb]{0.73,0.13,0.13}{\textit{{#1}}}}
    \newcommand{\AnnotationTok}[1]{\textcolor[rgb]{0.38,0.63,0.69}{\textbf{\textit{{#1}}}}}
    \newcommand{\CommentVarTok}[1]{\textcolor[rgb]{0.38,0.63,0.69}{\textbf{\textit{{#1}}}}}
    \newcommand{\VariableTok}[1]{\textcolor[rgb]{0.10,0.09,0.49}{{#1}}}
    \newcommand{\ControlFlowTok}[1]{\textcolor[rgb]{0.00,0.44,0.13}{\textbf{{#1}}}}
    \newcommand{\OperatorTok}[1]{\textcolor[rgb]{0.40,0.40,0.40}{{#1}}}
    \newcommand{\BuiltInTok}[1]{{#1}}
    \newcommand{\ExtensionTok}[1]{{#1}}
    \newcommand{\PreprocessorTok}[1]{\textcolor[rgb]{0.74,0.48,0.00}{{#1}}}
    \newcommand{\AttributeTok}[1]{\textcolor[rgb]{0.49,0.56,0.16}{{#1}}}
    \newcommand{\InformationTok}[1]{\textcolor[rgb]{0.38,0.63,0.69}{\textbf{\textit{{#1}}}}}
    \newcommand{\WarningTok}[1]{\textcolor[rgb]{0.38,0.63,0.69}{\textbf{\textit{{#1}}}}}
    
    
    % Define a nice break command that doesn't care if a line doesn't already
    % exist.
    \def\br{\hspace*{\fill} \\* }
    % Math Jax compatability definitions
    \def\gt{>}
    \def\lt{<}
    % Document parameters
    \title{Chapter1}
    
    
    

    % Pygments definitions
    
\makeatletter
\def\PY@reset{\let\PY@it=\relax \let\PY@bf=\relax%
    \let\PY@ul=\relax \let\PY@tc=\relax%
    \let\PY@bc=\relax \let\PY@ff=\relax}
\def\PY@tok#1{\csname PY@tok@#1\endcsname}
\def\PY@toks#1+{\ifx\relax#1\empty\else%
    \PY@tok{#1}\expandafter\PY@toks\fi}
\def\PY@do#1{\PY@bc{\PY@tc{\PY@ul{%
    \PY@it{\PY@bf{\PY@ff{#1}}}}}}}
\def\PY#1#2{\PY@reset\PY@toks#1+\relax+\PY@do{#2}}

\expandafter\def\csname PY@tok@w\endcsname{\def\PY@tc##1{\textcolor[rgb]{0.73,0.73,0.73}{##1}}}
\expandafter\def\csname PY@tok@c\endcsname{\let\PY@it=\textit\def\PY@tc##1{\textcolor[rgb]{0.25,0.50,0.50}{##1}}}
\expandafter\def\csname PY@tok@cp\endcsname{\def\PY@tc##1{\textcolor[rgb]{0.74,0.48,0.00}{##1}}}
\expandafter\def\csname PY@tok@k\endcsname{\let\PY@bf=\textbf\def\PY@tc##1{\textcolor[rgb]{0.00,0.50,0.00}{##1}}}
\expandafter\def\csname PY@tok@kp\endcsname{\def\PY@tc##1{\textcolor[rgb]{0.00,0.50,0.00}{##1}}}
\expandafter\def\csname PY@tok@kt\endcsname{\def\PY@tc##1{\textcolor[rgb]{0.69,0.00,0.25}{##1}}}
\expandafter\def\csname PY@tok@o\endcsname{\def\PY@tc##1{\textcolor[rgb]{0.40,0.40,0.40}{##1}}}
\expandafter\def\csname PY@tok@ow\endcsname{\let\PY@bf=\textbf\def\PY@tc##1{\textcolor[rgb]{0.67,0.13,1.00}{##1}}}
\expandafter\def\csname PY@tok@nb\endcsname{\def\PY@tc##1{\textcolor[rgb]{0.00,0.50,0.00}{##1}}}
\expandafter\def\csname PY@tok@nf\endcsname{\def\PY@tc##1{\textcolor[rgb]{0.00,0.00,1.00}{##1}}}
\expandafter\def\csname PY@tok@nc\endcsname{\let\PY@bf=\textbf\def\PY@tc##1{\textcolor[rgb]{0.00,0.00,1.00}{##1}}}
\expandafter\def\csname PY@tok@nn\endcsname{\let\PY@bf=\textbf\def\PY@tc##1{\textcolor[rgb]{0.00,0.00,1.00}{##1}}}
\expandafter\def\csname PY@tok@ne\endcsname{\let\PY@bf=\textbf\def\PY@tc##1{\textcolor[rgb]{0.82,0.25,0.23}{##1}}}
\expandafter\def\csname PY@tok@nv\endcsname{\def\PY@tc##1{\textcolor[rgb]{0.10,0.09,0.49}{##1}}}
\expandafter\def\csname PY@tok@no\endcsname{\def\PY@tc##1{\textcolor[rgb]{0.53,0.00,0.00}{##1}}}
\expandafter\def\csname PY@tok@nl\endcsname{\def\PY@tc##1{\textcolor[rgb]{0.63,0.63,0.00}{##1}}}
\expandafter\def\csname PY@tok@ni\endcsname{\let\PY@bf=\textbf\def\PY@tc##1{\textcolor[rgb]{0.60,0.60,0.60}{##1}}}
\expandafter\def\csname PY@tok@na\endcsname{\def\PY@tc##1{\textcolor[rgb]{0.49,0.56,0.16}{##1}}}
\expandafter\def\csname PY@tok@nt\endcsname{\let\PY@bf=\textbf\def\PY@tc##1{\textcolor[rgb]{0.00,0.50,0.00}{##1}}}
\expandafter\def\csname PY@tok@nd\endcsname{\def\PY@tc##1{\textcolor[rgb]{0.67,0.13,1.00}{##1}}}
\expandafter\def\csname PY@tok@s\endcsname{\def\PY@tc##1{\textcolor[rgb]{0.73,0.13,0.13}{##1}}}
\expandafter\def\csname PY@tok@sd\endcsname{\let\PY@it=\textit\def\PY@tc##1{\textcolor[rgb]{0.73,0.13,0.13}{##1}}}
\expandafter\def\csname PY@tok@si\endcsname{\let\PY@bf=\textbf\def\PY@tc##1{\textcolor[rgb]{0.73,0.40,0.53}{##1}}}
\expandafter\def\csname PY@tok@se\endcsname{\let\PY@bf=\textbf\def\PY@tc##1{\textcolor[rgb]{0.73,0.40,0.13}{##1}}}
\expandafter\def\csname PY@tok@sr\endcsname{\def\PY@tc##1{\textcolor[rgb]{0.73,0.40,0.53}{##1}}}
\expandafter\def\csname PY@tok@ss\endcsname{\def\PY@tc##1{\textcolor[rgb]{0.10,0.09,0.49}{##1}}}
\expandafter\def\csname PY@tok@sx\endcsname{\def\PY@tc##1{\textcolor[rgb]{0.00,0.50,0.00}{##1}}}
\expandafter\def\csname PY@tok@m\endcsname{\def\PY@tc##1{\textcolor[rgb]{0.40,0.40,0.40}{##1}}}
\expandafter\def\csname PY@tok@gh\endcsname{\let\PY@bf=\textbf\def\PY@tc##1{\textcolor[rgb]{0.00,0.00,0.50}{##1}}}
\expandafter\def\csname PY@tok@gu\endcsname{\let\PY@bf=\textbf\def\PY@tc##1{\textcolor[rgb]{0.50,0.00,0.50}{##1}}}
\expandafter\def\csname PY@tok@gd\endcsname{\def\PY@tc##1{\textcolor[rgb]{0.63,0.00,0.00}{##1}}}
\expandafter\def\csname PY@tok@gi\endcsname{\def\PY@tc##1{\textcolor[rgb]{0.00,0.63,0.00}{##1}}}
\expandafter\def\csname PY@tok@gr\endcsname{\def\PY@tc##1{\textcolor[rgb]{1.00,0.00,0.00}{##1}}}
\expandafter\def\csname PY@tok@ge\endcsname{\let\PY@it=\textit}
\expandafter\def\csname PY@tok@gs\endcsname{\let\PY@bf=\textbf}
\expandafter\def\csname PY@tok@gp\endcsname{\let\PY@bf=\textbf\def\PY@tc##1{\textcolor[rgb]{0.00,0.00,0.50}{##1}}}
\expandafter\def\csname PY@tok@go\endcsname{\def\PY@tc##1{\textcolor[rgb]{0.53,0.53,0.53}{##1}}}
\expandafter\def\csname PY@tok@gt\endcsname{\def\PY@tc##1{\textcolor[rgb]{0.00,0.27,0.87}{##1}}}
\expandafter\def\csname PY@tok@err\endcsname{\def\PY@bc##1{\setlength{\fboxsep}{0pt}\fcolorbox[rgb]{1.00,0.00,0.00}{1,1,1}{\strut ##1}}}
\expandafter\def\csname PY@tok@kc\endcsname{\let\PY@bf=\textbf\def\PY@tc##1{\textcolor[rgb]{0.00,0.50,0.00}{##1}}}
\expandafter\def\csname PY@tok@kd\endcsname{\let\PY@bf=\textbf\def\PY@tc##1{\textcolor[rgb]{0.00,0.50,0.00}{##1}}}
\expandafter\def\csname PY@tok@kn\endcsname{\let\PY@bf=\textbf\def\PY@tc##1{\textcolor[rgb]{0.00,0.50,0.00}{##1}}}
\expandafter\def\csname PY@tok@kr\endcsname{\let\PY@bf=\textbf\def\PY@tc##1{\textcolor[rgb]{0.00,0.50,0.00}{##1}}}
\expandafter\def\csname PY@tok@bp\endcsname{\def\PY@tc##1{\textcolor[rgb]{0.00,0.50,0.00}{##1}}}
\expandafter\def\csname PY@tok@fm\endcsname{\def\PY@tc##1{\textcolor[rgb]{0.00,0.00,1.00}{##1}}}
\expandafter\def\csname PY@tok@vc\endcsname{\def\PY@tc##1{\textcolor[rgb]{0.10,0.09,0.49}{##1}}}
\expandafter\def\csname PY@tok@vg\endcsname{\def\PY@tc##1{\textcolor[rgb]{0.10,0.09,0.49}{##1}}}
\expandafter\def\csname PY@tok@vi\endcsname{\def\PY@tc##1{\textcolor[rgb]{0.10,0.09,0.49}{##1}}}
\expandafter\def\csname PY@tok@vm\endcsname{\def\PY@tc##1{\textcolor[rgb]{0.10,0.09,0.49}{##1}}}
\expandafter\def\csname PY@tok@sa\endcsname{\def\PY@tc##1{\textcolor[rgb]{0.73,0.13,0.13}{##1}}}
\expandafter\def\csname PY@tok@sb\endcsname{\def\PY@tc##1{\textcolor[rgb]{0.73,0.13,0.13}{##1}}}
\expandafter\def\csname PY@tok@sc\endcsname{\def\PY@tc##1{\textcolor[rgb]{0.73,0.13,0.13}{##1}}}
\expandafter\def\csname PY@tok@dl\endcsname{\def\PY@tc##1{\textcolor[rgb]{0.73,0.13,0.13}{##1}}}
\expandafter\def\csname PY@tok@s2\endcsname{\def\PY@tc##1{\textcolor[rgb]{0.73,0.13,0.13}{##1}}}
\expandafter\def\csname PY@tok@sh\endcsname{\def\PY@tc##1{\textcolor[rgb]{0.73,0.13,0.13}{##1}}}
\expandafter\def\csname PY@tok@s1\endcsname{\def\PY@tc##1{\textcolor[rgb]{0.73,0.13,0.13}{##1}}}
\expandafter\def\csname PY@tok@mb\endcsname{\def\PY@tc##1{\textcolor[rgb]{0.40,0.40,0.40}{##1}}}
\expandafter\def\csname PY@tok@mf\endcsname{\def\PY@tc##1{\textcolor[rgb]{0.40,0.40,0.40}{##1}}}
\expandafter\def\csname PY@tok@mh\endcsname{\def\PY@tc##1{\textcolor[rgb]{0.40,0.40,0.40}{##1}}}
\expandafter\def\csname PY@tok@mi\endcsname{\def\PY@tc##1{\textcolor[rgb]{0.40,0.40,0.40}{##1}}}
\expandafter\def\csname PY@tok@il\endcsname{\def\PY@tc##1{\textcolor[rgb]{0.40,0.40,0.40}{##1}}}
\expandafter\def\csname PY@tok@mo\endcsname{\def\PY@tc##1{\textcolor[rgb]{0.40,0.40,0.40}{##1}}}
\expandafter\def\csname PY@tok@ch\endcsname{\let\PY@it=\textit\def\PY@tc##1{\textcolor[rgb]{0.25,0.50,0.50}{##1}}}
\expandafter\def\csname PY@tok@cm\endcsname{\let\PY@it=\textit\def\PY@tc##1{\textcolor[rgb]{0.25,0.50,0.50}{##1}}}
\expandafter\def\csname PY@tok@cpf\endcsname{\let\PY@it=\textit\def\PY@tc##1{\textcolor[rgb]{0.25,0.50,0.50}{##1}}}
\expandafter\def\csname PY@tok@c1\endcsname{\let\PY@it=\textit\def\PY@tc##1{\textcolor[rgb]{0.25,0.50,0.50}{##1}}}
\expandafter\def\csname PY@tok@cs\endcsname{\let\PY@it=\textit\def\PY@tc##1{\textcolor[rgb]{0.25,0.50,0.50}{##1}}}

\def\PYZbs{\char`\\}
\def\PYZus{\char`\_}
\def\PYZob{\char`\{}
\def\PYZcb{\char`\}}
\def\PYZca{\char`\^}
\def\PYZam{\char`\&}
\def\PYZlt{\char`\<}
\def\PYZgt{\char`\>}
\def\PYZsh{\char`\#}
\def\PYZpc{\char`\%}
\def\PYZdl{\char`\$}
\def\PYZhy{\char`\-}
\def\PYZsq{\char`\'}
\def\PYZdq{\char`\"}
\def\PYZti{\char`\~}
% for compatibility with earlier versions
\def\PYZat{@}
\def\PYZlb{[}
\def\PYZrb{]}
\makeatother


    % Exact colors from NB
    \definecolor{incolor}{rgb}{0.0, 0.0, 0.5}
    \definecolor{outcolor}{rgb}{0.545, 0.0, 0.0}



    
    % Prevent overflowing lines due to hard-to-break entities
    \sloppy 
    % Setup hyperref package
    \hypersetup{
      breaklinks=true,  % so long urls are correctly broken across lines
      colorlinks=true,
      urlcolor=urlcolor,
      linkcolor=linkcolor,
      citecolor=citecolor,
      }
    % Slightly bigger margins than the latex defaults
    
    \geometry{verbose,tmargin=1in,bmargin=1in,lmargin=1in,rmargin=1in}
    
    

    \begin{document}
    
    
    \maketitle
    
    

    
    \hypertarget{chapter-1-finding-words-phrases-names-and-concepts}{%
\section{Chapter 1: Finding words, phrases, names and
concepts}\label{chapter-1-finding-words-phrases-names-and-concepts}}

This chapter will introduce you to the basics of text processing with
spaCy. You'll learn about the data structures, how to work with
statistical models, and how to use them to predict linguistic features
in your text.

\hypertarget{getting-started}{%
\subsubsection{1.1 Getting Started}\label{getting-started}}

Let's get started and try out spaCy! You'll be able to try out some of
the 55+ available languages.

\begin{itemize}
\tightlist
\item
  Import the English class from spacy.lang.en and create the
  \texttt{nlp} object.
\item
  Let's create a \texttt{doc} and print its \texttt{text}.
\end{itemize}

    \begin{Verbatim}[commandchars=\\\{\}]
{\color{incolor}In [{\color{incolor}1}]:} \PY{c+c1}{\PYZsh{} Import the English language class}
        \PY{k+kn}{from} \PY{n+nn}{spacy}\PY{n+nn}{.}\PY{n+nn}{lang}\PY{n+nn}{.}\PY{n+nn}{en} \PY{k}{import} \PY{n}{English}
        
        \PY{c+c1}{\PYZsh{} Create the nlp object}
        \PY{n}{nlp} \PY{o}{=} \PY{n}{English}\PY{p}{(}\PY{p}{)}
        
        \PY{c+c1}{\PYZsh{} Process a text}
        \PY{n}{doc} \PY{o}{=} \PY{n}{nlp}\PY{p}{(}\PY{l+s+s2}{\PYZdq{}}\PY{l+s+s2}{This is a sentence.}\PY{l+s+s2}{\PYZdq{}}\PY{p}{)}
        
        \PY{c+c1}{\PYZsh{} Print the document text}
        
        \PY{c+c1}{\PYZsh{} .text() näyttää alkuperäisen docin, }
        \PY{c+c1}{\PYZsh{} eli tässä tapauksessa lauseen}
        \PY{n+nb}{print}\PY{p}{(}\PY{n}{doc}\PY{o}{.}\PY{n}{text}\PY{p}{)}
\end{Verbatim}


    \begin{Verbatim}[commandchars=\\\{\}]
This is a sentence.

    \end{Verbatim}

    \hypertarget{exercise-1.1-repeat-chapter-1.1-with-finnish.}{%
\paragraph{Exercise 1.1: Repeat chapter 1.1 with
Finnish.}\label{exercise-1.1-repeat-chapter-1.1-with-finnish.}}

    \begin{Verbatim}[commandchars=\\\{\}]
{\color{incolor}In [{\color{incolor}3}]:} \PY{c+c1}{\PYZsh{} Exercise 1.1:}
        \PY{c+c1}{\PYZsh{} \PYZhy{}\PYZhy{}\PYZhy{}\PYZhy{}\PYZhy{}\PYZhy{}\PYZhy{}\PYZhy{}\PYZhy{}\PYZhy{}\PYZhy{}\PYZhy{}\PYZhy{}\PYZhy{}\PYZhy{}\PYZhy{}\PYZhy{}}
        \PY{c+c1}{\PYZsh{} Repeat chapter 1.1 with Finnish.}
        \PY{k+kn}{from} \PY{n+nn}{spacy}\PY{n+nn}{.}\PY{n+nn}{lang}\PY{n+nn}{.}\PY{n+nn}{fi} \PY{k}{import} \PY{n}{Finnish}
        
        \PY{n}{nlp} \PY{o}{=} \PY{n}{Finnish}\PY{p}{(}\PY{p}{)}
        
        \PY{n}{doc\PYZus{}f} \PY{o}{=} \PY{n}{nlp}\PY{p}{(}\PY{l+s+s2}{\PYZdq{}}\PY{l+s+s2}{Tämä on lause.}\PY{l+s+s2}{\PYZdq{}}\PY{p}{)}
        \PY{n+nb}{print}\PY{p}{(}\PY{n}{doc\PYZus{}f}\PY{o}{.}\PY{n}{text}\PY{p}{)}
\end{Verbatim}


    \begin{Verbatim}[commandchars=\\\{\}]
Tämä on lause.

    \end{Verbatim}

    \hypertarget{documents-spans-and-tokens}{%
\subsubsection{1.2 Documents, spans and
tokens}\label{documents-spans-and-tokens}}

When you call \texttt{nlp} on a string, spaCy first tokenizes the text
and creates a document object. In this exercise, you'll learn more about
the \texttt{Doc}, as well as its views \texttt{Token} and \texttt{Span}.

\textbf{Step 1}

\begin{itemize}
\tightlist
\item
  Let's import the English language class and create the \texttt{nlp}
  object.
\item
  Process the text and instantiate a Doc object in the variable doc.
\item
  Select the first token of the Doc and print its text.
\end{itemize}

    \begin{Verbatim}[commandchars=\\\{\}]
{\color{incolor}In [{\color{incolor}4}]:} \PY{c+c1}{\PYZsh{} Import the English language class and create the nlp object}
        \PY{k+kn}{from} \PY{n+nn}{spacy}\PY{n+nn}{.}\PY{n+nn}{lang}\PY{n+nn}{.}\PY{n+nn}{en} \PY{k}{import} \PY{n}{English}
        
        \PY{n}{nlp} \PY{o}{=} \PY{n}{English}\PY{p}{(}\PY{p}{)}
        
        \PY{c+c1}{\PYZsh{} Process the text}
        \PY{n}{doc} \PY{o}{=} \PY{n}{nlp}\PY{p}{(}\PY{l+s+s2}{\PYZdq{}}\PY{l+s+s2}{I like tree kangaroos and narwhals.}\PY{l+s+s2}{\PYZdq{}}\PY{p}{)}
        
        \PY{c+c1}{\PYZsh{} Select the first token}
        \PY{n}{first\PYZus{}token} \PY{o}{=} \PY{n}{doc}\PY{p}{[}\PY{l+m+mi}{0}\PY{p}{]}
        
        \PY{c+c1}{\PYZsh{} Print the first token\PYZsq{}s text}
        \PY{n+nb}{print}\PY{p}{(}\PY{n}{first\PYZus{}token}\PY{o}{.}\PY{n}{text}\PY{p}{)}
\end{Verbatim}


    \begin{Verbatim}[commandchars=\\\{\}]
I

    \end{Verbatim}

    \textbf{Step 2}

\begin{itemize}
\tightlist
\item
  Import the English language class and create the nlp object.
\item
  Process the text and instantiate a Doc object in the variable doc.
\item
  Create a slice of the Doc for the tokens ``tree kangaroos'' and ``tree
  kangaroos and narwhals''.
\end{itemize}

    \begin{Verbatim}[commandchars=\\\{\}]
{\color{incolor}In [{\color{incolor}5}]:} \PY{c+c1}{\PYZsh{} Import the English language class and create the nlp object}
        \PY{k+kn}{from} \PY{n+nn}{spacy}\PY{n+nn}{.}\PY{n+nn}{lang}\PY{n+nn}{.}\PY{n+nn}{en} \PY{k}{import} \PY{n}{English}
        
        \PY{n}{nlp} \PY{o}{=} \PY{n}{English}\PY{p}{(}\PY{p}{)}
        
        \PY{c+c1}{\PYZsh{} Process the text}
        \PY{n}{doc} \PY{o}{=} \PY{n}{nlp}\PY{p}{(}\PY{l+s+s2}{\PYZdq{}}\PY{l+s+s2}{I like tree kangaroos and narwhals.}\PY{l+s+s2}{\PYZdq{}}\PY{p}{)}
        
        \PY{c+c1}{\PYZsh{} A slice of the Doc for \PYZdq{}tree kangaroos\PYZdq{}}
        \PY{n}{tree\PYZus{}kangaroos} \PY{o}{=} \PY{n}{doc}\PY{p}{[}\PY{l+m+mi}{2}\PY{p}{:}\PY{l+m+mi}{4}\PY{p}{]}
        \PY{n+nb}{print}\PY{p}{(}\PY{n}{tree\PYZus{}kangaroos}\PY{o}{.}\PY{n}{text}\PY{p}{)}
        
        \PY{c+c1}{\PYZsh{} A slice of the Doc for \PYZdq{}tree kangaroos and narwhals\PYZdq{} (without the \PYZdq{}.\PYZdq{})}
        \PY{n}{tree\PYZus{}kangaroos\PYZus{}and\PYZus{}narwhals} \PY{o}{=} \PY{n}{doc}\PY{p}{[}\PY{l+m+mi}{2}\PY{p}{:}\PY{l+m+mi}{6}\PY{p}{]}
        \PY{n+nb}{print}\PY{p}{(}\PY{n}{tree\PYZus{}kangaroos\PYZus{}and\PYZus{}narwhals}\PY{o}{.}\PY{n}{text}\PY{p}{)}
\end{Verbatim}


    \begin{Verbatim}[commandchars=\\\{\}]
tree kangaroos
tree kangaroos and narwhals

    \end{Verbatim}

    \hypertarget{lexical-attributes}{%
\subsubsection{1.3 Lexical attributes}\label{lexical-attributes}}

In this example, you'll use spaCy's Doc and Token objects, and lexical
attributes to find percentages in a text. You'll be looking for two
subsequent tokens: a number and a percent sign.

\begin{itemize}
\tightlist
\item
  Use the like\_num token attribute to check whether a token in the doc
  resembles a number.
\item
  Get the token following the current token in the document. The index
  of the next token in the doc is token.i + 1.
\item
  Check whether the next token's text attribute is a percent sign
  ''\%``.
\end{itemize}

    \begin{Verbatim}[commandchars=\\\{\}]
{\color{incolor}In [{\color{incolor}6}]:} \PY{k+kn}{from} \PY{n+nn}{spacy}\PY{n+nn}{.}\PY{n+nn}{lang}\PY{n+nn}{.}\PY{n+nn}{en} \PY{k}{import} \PY{n}{English}
        
        \PY{n}{nlp} \PY{o}{=} \PY{n}{English}\PY{p}{(}\PY{p}{)}
        
        \PY{c+c1}{\PYZsh{} Process the text}
        \PY{n}{doc} \PY{o}{=} \PY{n}{nlp}\PY{p}{(}
            \PY{l+s+s2}{\PYZdq{}}\PY{l+s+s2}{In 1990, more than 60}\PY{l+s+si}{\PYZpc{} o}\PY{l+s+s2}{f people in East Asia were in extreme poverty. }\PY{l+s+s2}{\PYZdq{}}
            \PY{l+s+s2}{\PYZdq{}}\PY{l+s+s2}{Now less than 4}\PY{l+s+si}{\PYZpc{} a}\PY{l+s+s2}{re.}\PY{l+s+s2}{\PYZdq{}}
        \PY{p}{)}
        
        \PY{c+c1}{\PYZsh{} Iterate over the tokens in the doc}
        \PY{k}{for} \PY{n}{token} \PY{o+ow}{in} \PY{n}{doc}\PY{p}{:}
            \PY{c+c1}{\PYZsh{} Check if the token resembles a number}
            \PY{k}{if} \PY{n}{token}\PY{o}{.}\PY{n}{like\PYZus{}num}\PY{p}{:}
                \PY{c+c1}{\PYZsh{} Get the next token in the document}
                \PY{c+c1}{\PYZsh{} token.i + 1 saadaan seuraava tokeni tarkastelun kohteeksi}
                \PY{n}{next\PYZus{}token} \PY{o}{=} \PY{n}{doc}\PY{p}{[}\PY{n}{token}\PY{o}{.}\PY{n}{i} \PY{o}{+} \PY{l+m+mi}{1}\PY{p}{]}
                \PY{c+c1}{\PYZsh{} Check if the next token\PYZsq{}s text equals \PYZdq{}\PYZpc{}\PYZdq{}}
                \PY{k}{if} \PY{n}{next\PYZus{}token}\PY{o}{.}\PY{n}{text} \PY{o}{==} \PY{l+s+s2}{\PYZdq{}}\PY{l+s+s2}{\PYZpc{}}\PY{l+s+s2}{\PYZdq{}}\PY{p}{:}
                    \PY{n+nb}{print}\PY{p}{(}\PY{l+s+s2}{\PYZdq{}}\PY{l+s+s2}{Percentage found:}\PY{l+s+s2}{\PYZdq{}}\PY{p}{,} \PY{n}{token}\PY{o}{.}\PY{n}{text}\PY{p}{)}
                
\end{Verbatim}


    \begin{Verbatim}[commandchars=\\\{\}]
Percentage found: 60
Percentage found: 4

    \end{Verbatim}

    \hypertarget{exercise-1.2-repeat-chapter-1.3-with-finnish.}{%
\paragraph{Exercise 1.2: Repeat chapter 1.3 with
Finnish.}\label{exercise-1.2-repeat-chapter-1.3-with-finnish.}}

    \begin{Verbatim}[commandchars=\\\{\}]
{\color{incolor}In [{\color{incolor}11}]:} \PY{c+c1}{\PYZsh{} Exercise 1.2:}
         \PY{c+c1}{\PYZsh{} \PYZhy{}\PYZhy{}\PYZhy{}\PYZhy{}\PYZhy{}\PYZhy{}\PYZhy{}\PYZhy{}\PYZhy{}\PYZhy{}\PYZhy{}\PYZhy{}\PYZhy{}\PYZhy{}\PYZhy{}\PYZhy{}\PYZhy{}}
         \PY{c+c1}{\PYZsh{} Repeat chapter 1.3 with Finnish.}
         
         \PY{k+kn}{from} \PY{n+nn}{spacy}\PY{n+nn}{.}\PY{n+nn}{lang}\PY{n+nn}{.}\PY{n+nn}{fi} \PY{k}{import} \PY{n}{Finnish}
         
         \PY{n}{nlp} \PY{o}{=} \PY{n}{Finnish}\PY{p}{(}\PY{p}{)}
         
         \PY{n}{doc\PYZus{}f} \PY{o}{=} \PY{n}{nlp}\PY{p}{(}
         \PY{l+s+s2}{\PYZdq{}}\PY{l+s+s2}{Vuonna 1990 yli 60}\PY{l+s+s2}{\PYZpc{}}\PY{l+s+s2}{ Itä\PYZhy{}Aasian ihmisistä oli äärimmäisessä köyhyydessä.}\PY{l+s+s2}{\PYZdq{}}
         \PY{l+s+s2}{\PYZdq{}}\PY{l+s+s2}{Nykyään alle 4}\PY{l+s+s2}{\PYZpc{}}\PY{l+s+s2}{. }\PY{l+s+s2}{\PYZdq{}}
         \PY{p}{)}
         
         
         \PY{n}{doc\PYZus{}f2}\PY{o}{=} \PY{n}{nlp}\PY{p}{(}
         \PY{l+s+s2}{\PYZdq{}}\PY{l+s+s2}{Yli puolet 60}\PY{l+s+si}{\PYZpc{} s}\PY{l+s+s2}{uomalaisista ajattelee, että lihan syömiseen suhtaudutaan tällä hetkellä liian tuomitsevasti.}\PY{l+s+s2}{\PYZdq{}}
         \PY{l+s+s2}{\PYZdq{}}\PY{l+s+s2}{Lähes yhtä moni 58}\PY{l+s+si}{\PYZpc{} s}\PY{l+s+s2}{uomalaisista kokee, että lihansyönnin vähentämiselle asetetaan yhteiskunnallisia ja sosiaalisia paineita.}\PY{l+s+s2}{\PYZdq{}}
         \PY{p}{)}
         
         \PY{c+c1}{\PYZsh{} Etsitään prosenttimerkki suomen kielisen tekstin seasta yksikertaisen loopin avulla}
         
         \PY{c+c1}{\PYZsh{} Käydään läpi docin (tekstin) tokenit}
         \PY{k}{for} \PY{n}{token} \PY{o+ow}{in} \PY{n}{doc\PYZus{}f2}\PY{p}{:}
             \PY{c+c1}{\PYZsh{} Jos tokeni muistuttaa numeroa, otetaan se tarkasteluun}
             \PY{k}{if} \PY{n}{token}\PY{o}{.}\PY{n}{like\PYZus{}num}\PY{p}{:}
                 \PY{c+c1}{\PYZsh{} Seuraava tokeni saadaan token.i + 1:llä}
                 \PY{n}{next\PYZus{}token} \PY{o}{=} \PY{n}{doc\PYZus{}f2}\PY{p}{[}\PY{n}{token}\PY{o}{.}\PY{n}{i}\PY{o}{+}\PY{l+m+mi}{1}\PY{p}{]}
                 \PY{c+c1}{\PYZsh{} Jos seuraava token on prosenttimerkki ...}
                 \PY{c+c1}{\PYZsh{} ... olemme löytäneet prosenttiluvun teksistä}
                 \PY{k}{if} \PY{n}{next\PYZus{}token}\PY{o}{.}\PY{n}{text} \PY{o}{==} \PY{l+s+s2}{\PYZdq{}}\PY{l+s+s2}{\PYZpc{}}\PY{l+s+s2}{\PYZdq{}}\PY{p}{:}
                     \PY{n+nb}{print}\PY{p}{(}\PY{l+s+s2}{\PYZdq{}}\PY{l+s+s2}{Prosenttiluku löytyi: }\PY{l+s+s2}{\PYZdq{}}\PY{p}{,} \PY{n}{token}\PY{o}{.}\PY{n}{text}\PY{p}{)}
\end{Verbatim}


    \begin{Verbatim}[commandchars=\\\{\}]
Prosenttiluku löytyi:  60
Prosenttiluku löytyi:  58

    \end{Verbatim}

    \hypertarget{loading-models}{%
\subsubsection{1.4 Loading Models}\label{loading-models}}

Use spacy.load to load the small English model ``en\_core\_web\_sm''.
Process the text and print the document text.

    \begin{Verbatim}[commandchars=\\\{\}]
{\color{incolor}In [{\color{incolor}13}]:} \PY{k+kn}{import} \PY{n+nn}{spacy}
         
         \PY{c+c1}{\PYZsh{} Load the small English model}
         \PY{n}{nlp} \PY{o}{=} \PY{n}{spacy}\PY{o}{.}\PY{n}{load}\PY{p}{(}\PY{l+s+s2}{\PYZdq{}}\PY{l+s+s2}{en\PYZus{}core\PYZus{}web\PYZus{}sm}\PY{l+s+s2}{\PYZdq{}}\PY{p}{)}
         
         \PY{n}{text} \PY{o}{=} \PY{l+s+s2}{\PYZdq{}}\PY{l+s+s2}{It’s official: Apple is the first U.S. public company to reach a \PYZdl{}1 trillion market value}\PY{l+s+s2}{\PYZdq{}}
         
         \PY{c+c1}{\PYZsh{} Process the text}
         \PY{n}{doc} \PY{o}{=} \PY{n}{nlp}\PY{p}{(}\PY{n}{text}\PY{p}{)}
         
         \PY{c+c1}{\PYZsh{} Print the document text}
         \PY{n+nb}{print}\PY{p}{(}\PY{n}{doc}\PY{o}{.}\PY{n}{text}\PY{p}{)}
\end{Verbatim}


    \begin{Verbatim}[commandchars=\\\{\}]
It’s official: Apple is the first U.S. public company to reach a \$1 trillion market value

    \end{Verbatim}

    \hypertarget{predicting-linquistic-annotations}{%
\subsubsection{1.5 Predicting linquistic
annotations}\label{predicting-linquistic-annotations}}

You'll now get to try one of spaCy's pre-trained model packages and see
its predictions in action. Feel free to try it out on your own text! To
find out what a tag or label means, you can call spacy.explain in the
loop. For example: spacy.explain(``PROPN'') or spacy.explain(``GPE'').

\textbf{Part 1}

\begin{itemize}
\tightlist
\item
  Process the text with the nlp object and create a doc.
\item
  For each token, print the token text, the token's .pos\_
  (part-of-speech tag) and the token's .dep\_ (dependency label).
\end{itemize}

    \begin{Verbatim}[commandchars=\\\{\}]
{\color{incolor}In [{\color{incolor}14}]:} \PY{k+kn}{import} \PY{n+nn}{spacy}
         
         \PY{n}{nlp} \PY{o}{=} \PY{n}{spacy}\PY{o}{.}\PY{n}{load}\PY{p}{(}\PY{l+s+s2}{\PYZdq{}}\PY{l+s+s2}{en\PYZus{}core\PYZus{}web\PYZus{}sm}\PY{l+s+s2}{\PYZdq{}}\PY{p}{)}
         
         \PY{n}{text} \PY{o}{=} \PY{l+s+s2}{\PYZdq{}}\PY{l+s+s2}{It’s official: Apple is the first U.S. public company to reach a \PYZdl{}1 trillion market value}\PY{l+s+s2}{\PYZdq{}}
         
         \PY{c+c1}{\PYZsh{} Process the text}
         \PY{n}{doc} \PY{o}{=} \PY{n}{nlp}\PY{p}{(}\PY{n}{text}\PY{p}{)}
         
         \PY{k}{for} \PY{n}{token} \PY{o+ow}{in} \PY{n}{doc}\PY{p}{:}
             \PY{c+c1}{\PYZsh{} Get the token text, part\PYZhy{}of\PYZhy{}speech tag and dependency label}
             \PY{n}{token\PYZus{}text} \PY{o}{=} \PY{n}{token}\PY{o}{.}\PY{n}{text}
             \PY{n}{token\PYZus{}pos} \PY{o}{=} \PY{n}{token}\PY{o}{.}\PY{n}{pos\PYZus{}}
             \PY{n}{token\PYZus{}dep} \PY{o}{=} \PY{n}{token}\PY{o}{.}\PY{n}{dep\PYZus{}}
             \PY{c+c1}{\PYZsh{} This is for formatting only}
             \PY{n+nb}{print}\PY{p}{(}\PY{n}{f}\PY{l+s+s2}{\PYZdq{}}\PY{l+s+si}{\PYZob{}token\PYZus{}text:\PYZlt{}12\PYZcb{}}\PY{l+s+si}{\PYZob{}token\PYZus{}pos:\PYZlt{}10\PYZcb{}}\PY{l+s+si}{\PYZob{}token\PYZus{}dep:\PYZlt{}10\PYZcb{}}\PY{l+s+s2}{\PYZdq{}}\PY{p}{)}
\end{Verbatim}


    \begin{Verbatim}[commandchars=\\\{\}]
It          PRON      nsubj     
’s          VERB      ccomp     
official    ADJ       acomp     
:           PUNCT     punct     
Apple       PROPN     nsubj     
is          AUX       ROOT      
the         DET       det       
first       ADJ       amod      
U.S.        PROPN     nmod      
public      ADJ       amod      
company     NOUN      attr      
to          PART      aux       
reach       VERB      relcl     
a           DET       det       
\$           SYM       quantmod  
1           NUM       compound  
trillion    NUM       nummod    
market      NOUN      compound  
value       NOUN      dobj      

    \end{Verbatim}

    \hypertarget{exercise-1.3-print-explanations-for-propn-nsubj-and-quantmod.}{%
\paragraph{Exercise 1.3: Print explanations for PROPN, nsubj and
quantmod.}\label{exercise-1.3-print-explanations-for-propn-nsubj-and-quantmod.}}

    \begin{Verbatim}[commandchars=\\\{\}]
{\color{incolor}In [{\color{incolor}16}]:} \PY{c+c1}{\PYZsh{} Exercise 1.3:}
         \PY{c+c1}{\PYZsh{} \PYZhy{}\PYZhy{}\PYZhy{}\PYZhy{}\PYZhy{}\PYZhy{}\PYZhy{}\PYZhy{}\PYZhy{}\PYZhy{}\PYZhy{}\PYZhy{}\PYZhy{}\PYZhy{}\PYZhy{}\PYZhy{}\PYZhy{}}
         \PY{c+c1}{\PYZsh{} Print explanations for PROPN, nsubj and quantmod. }
         \PY{n+nb}{print}\PY{p}{(}\PY{n}{spacy}\PY{o}{.}\PY{n}{explain}\PY{p}{(}\PY{l+s+s2}{\PYZdq{}}\PY{l+s+s2}{PROPN}\PY{l+s+s2}{\PYZdq{}}\PY{p}{)}\PY{p}{)}
         \PY{n+nb}{print}\PY{p}{(}\PY{n}{spacy}\PY{o}{.}\PY{n}{explain}\PY{p}{(}\PY{l+s+s2}{\PYZdq{}}\PY{l+s+s2}{nsubj}\PY{l+s+s2}{\PYZdq{}}\PY{p}{)}\PY{p}{)}
         \PY{n+nb}{print}\PY{p}{(}\PY{n}{spacy}\PY{o}{.}\PY{n}{explain}\PY{p}{(}\PY{l+s+s2}{\PYZdq{}}\PY{l+s+s2}{quantmod}\PY{l+s+s2}{\PYZdq{}}\PY{p}{)}\PY{p}{)}
\end{Verbatim}


    \begin{Verbatim}[commandchars=\\\{\}]
proper noun
nominal subject
modifier of quantifier

    \end{Verbatim}

    \textbf{Part 2}

Process the text and create a doc object. Iterate over the doc.ents and
print the entity text and label\_ attribute.

    \begin{Verbatim}[commandchars=\\\{\}]
{\color{incolor}In [{\color{incolor}17}]:} \PY{k+kn}{import} \PY{n+nn}{spacy}
         
         \PY{n}{nlp} \PY{o}{=} \PY{n}{spacy}\PY{o}{.}\PY{n}{load}\PY{p}{(}\PY{l+s+s2}{\PYZdq{}}\PY{l+s+s2}{en\PYZus{}core\PYZus{}web\PYZus{}sm}\PY{l+s+s2}{\PYZdq{}}\PY{p}{)}
         
         \PY{n}{text} \PY{o}{=} \PY{l+s+s2}{\PYZdq{}}\PY{l+s+s2}{It’s official: Apple is the first U.S. public company to reach a \PYZdl{}1 trillion market value}\PY{l+s+s2}{\PYZdq{}}
         
         \PY{c+c1}{\PYZsh{} Process the text}
         \PY{n}{doc} \PY{o}{=} \PY{n}{nlp}\PY{p}{(}\PY{n}{text}\PY{p}{)}
         
         \PY{c+c1}{\PYZsh{} Iterate over the predicted entities}
         \PY{k}{for} \PY{n}{ent} \PY{o+ow}{in} \PY{n}{doc}\PY{o}{.}\PY{n}{ents}\PY{p}{:}
             \PY{c+c1}{\PYZsh{} Print the entity text and its label}
             \PY{n+nb}{print}\PY{p}{(}\PY{n}{ent}\PY{o}{.}\PY{n}{text}\PY{p}{,} \PY{n}{ent}\PY{o}{.}\PY{n}{label\PYZus{}}\PY{p}{)}
\end{Verbatim}


    \begin{Verbatim}[commandchars=\\\{\}]
Apple ORG
first ORDINAL
U.S. GPE
\$1 trillion MONEY

    \end{Verbatim}

    \hypertarget{predicting-named-entities-in-context}{%
\subsubsection{1.6 Predicting named entities in
context}\label{predicting-named-entities-in-context}}

Models are statistical and not always right. Whether their predictions
are correct depends on the training data and the text you're processing.
Let's take a look at an example.

\begin{itemize}
\tightlist
\item
  Process the text with the nlp object.
\item
  Iterate over the entities and print the entity text and label.
\item
  Looks like the model didn't predict ``iPhone X''. Create a span for
  those tokens manually.
\end{itemize}

    \begin{Verbatim}[commandchars=\\\{\}]
{\color{incolor}In [{\color{incolor}18}]:} \PY{k+kn}{import} \PY{n+nn}{spacy}
         
         \PY{n}{nlp} \PY{o}{=} \PY{n}{spacy}\PY{o}{.}\PY{n}{load}\PY{p}{(}\PY{l+s+s2}{\PYZdq{}}\PY{l+s+s2}{en\PYZus{}core\PYZus{}web\PYZus{}sm}\PY{l+s+s2}{\PYZdq{}}\PY{p}{)}
         
         \PY{n}{text} \PY{o}{=} \PY{l+s+s2}{\PYZdq{}}\PY{l+s+s2}{Upcoming iPhone X release date leaked as Apple reveals pre\PYZhy{}orders}\PY{l+s+s2}{\PYZdq{}}
         
         \PY{c+c1}{\PYZsh{} Process the text}
         \PY{n}{doc} \PY{o}{=} \PY{n}{nlp}\PY{p}{(}\PY{n}{text}\PY{p}{)}
         
         \PY{c+c1}{\PYZsh{} Iterate over the entities}
         \PY{k}{for} \PY{n}{ent} \PY{o+ow}{in} \PY{n}{doc}\PY{o}{.}\PY{n}{ents}\PY{p}{:}
             \PY{c+c1}{\PYZsh{} Print the entity text and label}
             \PY{n+nb}{print}\PY{p}{(}\PY{n}{ent}\PY{o}{.}\PY{n}{text}\PY{p}{,} \PY{n}{ent}\PY{o}{.}\PY{n}{label\PYZus{}}\PY{p}{)}
         
         \PY{c+c1}{\PYZsh{} Get the span for \PYZdq{}iPhone X\PYZdq{}}
         \PY{n}{iphone\PYZus{}x} \PY{o}{=} \PY{n}{doc}\PY{p}{[}\PY{l+m+mi}{1}\PY{p}{:}\PY{l+m+mi}{3}\PY{p}{]}
         
         \PY{c+c1}{\PYZsh{} Print the span text}
         \PY{n+nb}{print}\PY{p}{(}\PY{l+s+s2}{\PYZdq{}}\PY{l+s+s2}{Missing entity:}\PY{l+s+s2}{\PYZdq{}}\PY{p}{,} \PY{n}{iphone\PYZus{}x}\PY{o}{.}\PY{n}{text}\PY{p}{)}
\end{Verbatim}


    \begin{Verbatim}[commandchars=\\\{\}]
Apple ORG
Missing entity: iPhone X

    \end{Verbatim}

    \hypertarget{using-the-matcher}{%
\subsubsection{1.7 Using the matcher}\label{using-the-matcher}}

Let's try spaCy's rule-based Matcher. You'll be using the example from
the previous exercise and write a pattern that can match the phrase
``iPhone X'' in the text.

\begin{itemize}
\tightlist
\item
  Import the Matcher from spacy.matcher.
\item
  Initialize it with the nlp object's shared vocab.
\item
  Create a pattern that matches the ``TEXT'' values of two tokens:
  ``iPhone'' and ``X''.
\item
  Use the matcher.add method to add the pattern to the matcher.
\item
  Call the matcher on the doc and store the result in the variable
  matches.
\item
  Iterate over the matches and get the matched span from the start to
  the end index.
\end{itemize}

\hypertarget{exercise-1.4-write-matcher-for-iphone-followed-by-x.}{%
\paragraph{Exercise 1.4: Write matcher for ``iPhone'' followed by
``X''.}\label{exercise-1.4-write-matcher-for-iphone-followed-by-x.}}

    \begin{Verbatim}[commandchars=\\\{\}]
{\color{incolor}In [{\color{incolor}33}]:} \PY{k+kn}{import} \PY{n+nn}{spacy}
         
         \PY{c+c1}{\PYZsh{} Import the Matcher}
         \PY{k+kn}{from} \PY{n+nn}{spacy}\PY{n+nn}{.}\PY{n+nn}{matcher} \PY{k}{import} \PY{n}{Matcher}
         
         \PY{n}{nlp} \PY{o}{=} \PY{n}{spacy}\PY{o}{.}\PY{n}{load}\PY{p}{(}\PY{l+s+s2}{\PYZdq{}}\PY{l+s+s2}{en\PYZus{}core\PYZus{}web\PYZus{}sm}\PY{l+s+s2}{\PYZdq{}}\PY{p}{)}
         \PY{n}{doc} \PY{o}{=} \PY{n}{nlp}\PY{p}{(}\PY{l+s+s2}{\PYZdq{}}\PY{l+s+s2}{Upcoming iPhone X release date leaked as Apple reveals pre\PYZhy{}orders}\PY{l+s+s2}{\PYZdq{}}\PY{p}{)}
         
         \PY{c+c1}{\PYZsh{} Initialize the Matcher with the shared vocabulary}
         \PY{n}{matcher} \PY{o}{=} \PY{n}{Matcher}\PY{p}{(}\PY{n}{nlp}\PY{o}{.}\PY{n}{vocab}\PY{p}{)}
         
         \PY{c+c1}{\PYZsh{} Exercise 1.4: Write matcher for \PYZdq{}iPhone\PYZdq{} followed by \PYZdq{}X\PYZdq{}}
         \PY{c+c1}{\PYZsh{} \PYZhy{}\PYZhy{}\PYZhy{}\PYZhy{}\PYZhy{}\PYZhy{}\PYZhy{}\PYZhy{}\PYZhy{}\PYZhy{}\PYZhy{}}
         \PY{c+c1}{\PYZsh{} Create a pattern matching two tokens: \PYZdq{}iPhone\PYZdq{} and \PYZdq{}X\PYZdq{}}
         \PY{n}{pattern} \PY{o}{=} \PY{p}{[}\PY{p}{[}\PY{p}{\PYZob{}}\PY{l+s+s2}{\PYZdq{}}\PY{l+s+s2}{TEXT}\PY{l+s+s2}{\PYZdq{}}\PY{p}{:} \PY{l+s+s2}{\PYZdq{}}\PY{l+s+s2}{iPhone}\PY{l+s+s2}{\PYZdq{}}\PY{p}{\PYZcb{}}\PY{p}{,} \PY{p}{\PYZob{}}\PY{l+s+s2}{\PYZdq{}}\PY{l+s+s2}{TEXT}\PY{l+s+s2}{\PYZdq{}}\PY{p}{:} \PY{l+s+s2}{\PYZdq{}}\PY{l+s+s2}{X}\PY{l+s+s2}{\PYZdq{}}\PY{p}{\PYZcb{}}\PY{p}{]}\PY{p}{]}
         
         \PY{c+c1}{\PYZsh{} Add the pattern to the matcher}
         \PY{n}{matcher}\PY{o}{.}\PY{n}{add}\PY{p}{(}\PY{l+s+s2}{\PYZdq{}}\PY{l+s+s2}{IPHONE\PYZus{}X\PYZus{}PATTERN}\PY{l+s+s2}{\PYZdq{}}\PY{p}{,} \PY{n}{pattern}\PY{p}{)}
         
         \PY{c+c1}{\PYZsh{} Use the matcher on the doc}
         \PY{n}{matches} \PY{o}{=} \PY{n}{matcher}\PY{p}{(}\PY{n}{doc}\PY{p}{)}
         \PY{n+nb}{print}\PY{p}{(}\PY{l+s+s2}{\PYZdq{}}\PY{l+s+s2}{Matches:}\PY{l+s+s2}{\PYZdq{}}\PY{p}{,} \PY{p}{[}\PY{n}{doc}\PY{p}{[}\PY{n}{start}\PY{p}{:}\PY{n}{end}\PY{p}{]}\PY{o}{.}\PY{n}{text} \PY{k}{for} \PY{n}{match\PYZus{}id}\PY{p}{,} \PY{n}{start}\PY{p}{,} \PY{n}{end} \PY{o+ow}{in} \PY{n}{matches}\PY{p}{]}\PY{p}{)}
\end{Verbatim}


    \begin{Verbatim}[commandchars=\\\{\}]
Matches: ['iPhone X']

    \end{Verbatim}

    \hypertarget{writing-match-patterns}{%
\subsubsection{1.8 Writing match
patterns}\label{writing-match-patterns}}

In this exercise, you'll practice writing more complex match patterns
using different token attributes and operators.

\textbf{Part 1}

Write one pattern that only matches mentions of the full iOS versions:
``iOS 7'', ``iOS 11'' and ``iOS 10''.

    \begin{Verbatim}[commandchars=\\\{\}]
{\color{incolor}In [{\color{incolor}34}]:} \PY{k+kn}{import} \PY{n+nn}{spacy}
         \PY{k+kn}{from} \PY{n+nn}{spacy}\PY{n+nn}{.}\PY{n+nn}{matcher} \PY{k}{import} \PY{n}{Matcher}
         
         \PY{n}{nlp} \PY{o}{=} \PY{n}{spacy}\PY{o}{.}\PY{n}{load}\PY{p}{(}\PY{l+s+s2}{\PYZdq{}}\PY{l+s+s2}{en\PYZus{}core\PYZus{}web\PYZus{}sm}\PY{l+s+s2}{\PYZdq{}}\PY{p}{)}
         \PY{n}{matcher} \PY{o}{=} \PY{n}{Matcher}\PY{p}{(}\PY{n}{nlp}\PY{o}{.}\PY{n}{vocab}\PY{p}{)}
         
         \PY{n}{doc} \PY{o}{=} \PY{n}{nlp}\PY{p}{(}
             \PY{l+s+s2}{\PYZdq{}}\PY{l+s+s2}{After making the iOS update you won}\PY{l+s+s2}{\PYZsq{}}\PY{l+s+s2}{t notice a radical system\PYZhy{}wide }\PY{l+s+s2}{\PYZdq{}}
             \PY{l+s+s2}{\PYZdq{}}\PY{l+s+s2}{redesign: nothing like the aesthetic upheaval we got with iOS 7. Most of }\PY{l+s+s2}{\PYZdq{}}
             \PY{l+s+s2}{\PYZdq{}}\PY{l+s+s2}{iOS 11}\PY{l+s+s2}{\PYZsq{}}\PY{l+s+s2}{s furniture remains the same as in iOS 10. But you will discover }\PY{l+s+s2}{\PYZdq{}}
             \PY{l+s+s2}{\PYZdq{}}\PY{l+s+s2}{some tweaks once you delve a little deeper.}\PY{l+s+s2}{\PYZdq{}}
         \PY{p}{)}
         
         \PY{c+c1}{\PYZsh{} Write a pattern for full iOS versions (\PYZdq{}iOS 7\PYZdq{}, \PYZdq{}iOS 11\PYZdq{}, \PYZdq{}iOS 10\PYZdq{})}
         \PY{n}{pattern} \PY{o}{=} \PY{p}{[}\PY{p}{[}\PY{p}{\PYZob{}}\PY{l+s+s2}{\PYZdq{}}\PY{l+s+s2}{TEXT}\PY{l+s+s2}{\PYZdq{}}\PY{p}{:} \PY{l+s+s2}{\PYZdq{}}\PY{l+s+s2}{iOS}\PY{l+s+s2}{\PYZdq{}}\PY{p}{\PYZcb{}}\PY{p}{,} \PY{p}{\PYZob{}}\PY{l+s+s2}{\PYZdq{}}\PY{l+s+s2}{IS\PYZus{}DIGIT}\PY{l+s+s2}{\PYZdq{}}\PY{p}{:} \PY{k+kc}{True}\PY{p}{\PYZcb{}}\PY{p}{]}\PY{p}{]}
         
         \PY{c+c1}{\PYZsh{} Add the pattern to the matcher and apply the matcher to the doc}
         \PY{n}{matcher}\PY{o}{.}\PY{n}{add}\PY{p}{(}\PY{l+s+s2}{\PYZdq{}}\PY{l+s+s2}{IOS\PYZus{}VERSION\PYZus{}PATTERN}\PY{l+s+s2}{\PYZdq{}}\PY{p}{,} \PY{n}{pattern}\PY{p}{)}
         \PY{n}{matches} \PY{o}{=} \PY{n}{matcher}\PY{p}{(}\PY{n}{doc}\PY{p}{)}
         \PY{n+nb}{print}\PY{p}{(}\PY{l+s+s2}{\PYZdq{}}\PY{l+s+s2}{Total matches found:}\PY{l+s+s2}{\PYZdq{}}\PY{p}{,} \PY{n+nb}{len}\PY{p}{(}\PY{n}{matches}\PY{p}{)}\PY{p}{)}
         
         \PY{c+c1}{\PYZsh{} Iterate over the matches and print the span text}
         \PY{k}{for} \PY{n}{match\PYZus{}id}\PY{p}{,} \PY{n}{start}\PY{p}{,} \PY{n}{end} \PY{o+ow}{in} \PY{n}{matches}\PY{p}{:}
             \PY{n+nb}{print}\PY{p}{(}\PY{l+s+s2}{\PYZdq{}}\PY{l+s+s2}{Match found:}\PY{l+s+s2}{\PYZdq{}}\PY{p}{,} \PY{n}{doc}\PY{p}{[}\PY{n}{start}\PY{p}{:}\PY{n}{end}\PY{p}{]}\PY{o}{.}\PY{n}{text}\PY{p}{)}
\end{Verbatim}


    \begin{Verbatim}[commandchars=\\\{\}]
Total matches found: 3
Match found: iOS 7
Match found: iOS 11
Match found: iOS 10

    \end{Verbatim}

    \textbf{Part 2}

Write one pattern that only matches forms of ``download'' (tokens with
the lemma ``download''), followed by a token with the part-of-speech tag
``PROPN'' (proper noun).

    \begin{Verbatim}[commandchars=\\\{\}]
{\color{incolor}In [{\color{incolor}35}]:} \PY{k+kn}{import} \PY{n+nn}{spacy}
         \PY{k+kn}{from} \PY{n+nn}{spacy}\PY{n+nn}{.}\PY{n+nn}{matcher} \PY{k}{import} \PY{n}{Matcher}
         
         \PY{n}{nlp} \PY{o}{=} \PY{n}{spacy}\PY{o}{.}\PY{n}{load}\PY{p}{(}\PY{l+s+s2}{\PYZdq{}}\PY{l+s+s2}{en\PYZus{}core\PYZus{}web\PYZus{}sm}\PY{l+s+s2}{\PYZdq{}}\PY{p}{)}
         \PY{n}{matcher} \PY{o}{=} \PY{n}{Matcher}\PY{p}{(}\PY{n}{nlp}\PY{o}{.}\PY{n}{vocab}\PY{p}{)}
         
         \PY{n}{doc} \PY{o}{=} \PY{n}{nlp}\PY{p}{(}
             \PY{l+s+s2}{\PYZdq{}}\PY{l+s+s2}{i downloaded Fortnite on my laptop and can}\PY{l+s+s2}{\PYZsq{}}\PY{l+s+s2}{t open the game at all. Help? }\PY{l+s+s2}{\PYZdq{}}
             \PY{l+s+s2}{\PYZdq{}}\PY{l+s+s2}{so when I was downloading Minecraft, I got the Windows version where it }\PY{l+s+s2}{\PYZdq{}}
             \PY{l+s+s2}{\PYZdq{}}\PY{l+s+s2}{is the }\PY{l+s+s2}{\PYZsq{}}\PY{l+s+s2}{.zip}\PY{l+s+s2}{\PYZsq{}}\PY{l+s+s2}{ folder and I used the default program to unpack it... do }\PY{l+s+s2}{\PYZdq{}}
             \PY{l+s+s2}{\PYZdq{}}\PY{l+s+s2}{I also need to download Winzip?}\PY{l+s+s2}{\PYZdq{}}
         \PY{p}{)}
         
         \PY{c+c1}{\PYZsh{} Write a pattern that matches a form of \PYZdq{}download\PYZdq{} plus proper noun}
         \PY{n}{pattern} \PY{o}{=} \PY{p}{[}\PY{p}{[}\PY{p}{\PYZob{}}\PY{l+s+s2}{\PYZdq{}}\PY{l+s+s2}{LEMMA}\PY{l+s+s2}{\PYZdq{}}\PY{p}{:} \PY{l+s+s2}{\PYZdq{}}\PY{l+s+s2}{download}\PY{l+s+s2}{\PYZdq{}}\PY{p}{\PYZcb{}}\PY{p}{,} \PY{p}{\PYZob{}}\PY{l+s+s2}{\PYZdq{}}\PY{l+s+s2}{POS}\PY{l+s+s2}{\PYZdq{}}\PY{p}{:} \PY{l+s+s2}{\PYZdq{}}\PY{l+s+s2}{PROPN}\PY{l+s+s2}{\PYZdq{}}\PY{p}{\PYZcb{}}\PY{p}{]}\PY{p}{]}
         
         \PY{c+c1}{\PYZsh{} Add the pattern to the matcher and apply the matcher to the doc}
         \PY{n}{matcher}\PY{o}{.}\PY{n}{add}\PY{p}{(}\PY{l+s+s2}{\PYZdq{}}\PY{l+s+s2}{DOWNLOAD\PYZus{}THINGS\PYZus{}PATTERN}\PY{l+s+s2}{\PYZdq{}}\PY{p}{,} \PY{n}{pattern}\PY{p}{)}
         \PY{n}{matches} \PY{o}{=} \PY{n}{matcher}\PY{p}{(}\PY{n}{doc}\PY{p}{)}
         \PY{n+nb}{print}\PY{p}{(}\PY{l+s+s2}{\PYZdq{}}\PY{l+s+s2}{Total matches found:}\PY{l+s+s2}{\PYZdq{}}\PY{p}{,} \PY{n+nb}{len}\PY{p}{(}\PY{n}{matches}\PY{p}{)}\PY{p}{)}
         
         \PY{c+c1}{\PYZsh{} Iterate over the matches and print the span text}
         \PY{k}{for} \PY{n}{match\PYZus{}id}\PY{p}{,} \PY{n}{start}\PY{p}{,} \PY{n}{end} \PY{o+ow}{in} \PY{n}{matches}\PY{p}{:}
             \PY{n+nb}{print}\PY{p}{(}\PY{l+s+s2}{\PYZdq{}}\PY{l+s+s2}{Match found:}\PY{l+s+s2}{\PYZdq{}}\PY{p}{,} \PY{n}{doc}\PY{p}{[}\PY{n}{start}\PY{p}{:}\PY{n}{end}\PY{p}{]}\PY{o}{.}\PY{n}{text}\PY{p}{)}
\end{Verbatim}


    \begin{Verbatim}[commandchars=\\\{\}]
Total matches found: 3
Match found: downloaded Fortnite
Match found: downloading Minecraft
Match found: download Winzip

    \end{Verbatim}

    \hypertarget{exercise-1.5-write-matcher}{%
\paragraph{Exercise 1.5: Write
matcher}\label{exercise-1.5-write-matcher}}

Write one pattern that matches adjectives (``ADJ'') followed by one or
two ``NOUN''s (one noun and one optional noun).

    \begin{Verbatim}[commandchars=\\\{\}]
{\color{incolor}In [{\color{incolor}39}]:} \PY{k+kn}{import} \PY{n+nn}{spacy}
         \PY{k+kn}{from} \PY{n+nn}{spacy}\PY{n+nn}{.}\PY{n+nn}{matcher} \PY{k}{import} \PY{n}{Matcher}
         
         \PY{n}{nlp} \PY{o}{=} \PY{n}{spacy}\PY{o}{.}\PY{n}{load}\PY{p}{(}\PY{l+s+s2}{\PYZdq{}}\PY{l+s+s2}{en\PYZus{}core\PYZus{}web\PYZus{}sm}\PY{l+s+s2}{\PYZdq{}}\PY{p}{)}
         \PY{n}{matcher} \PY{o}{=} \PY{n}{Matcher}\PY{p}{(}\PY{n}{nlp}\PY{o}{.}\PY{n}{vocab}\PY{p}{)}
         
         \PY{n}{doc} \PY{o}{=} \PY{n}{nlp}\PY{p}{(}
             \PY{l+s+s2}{\PYZdq{}}\PY{l+s+s2}{Features of the app include a beautiful design, smart search, automatic }\PY{l+s+s2}{\PYZdq{}}
             \PY{l+s+s2}{\PYZdq{}}\PY{l+s+s2}{labels and optional voice responses.}\PY{l+s+s2}{\PYZdq{}}
         \PY{p}{)}
         
         
         \PY{c+c1}{\PYZsh{} Exercise 1.5: Write pattern and use matcher with pattern to find matches}
         \PY{c+c1}{\PYZsh{} \PYZhy{}\PYZhy{}\PYZhy{}\PYZhy{}\PYZhy{}\PYZhy{}\PYZhy{}\PYZhy{}\PYZhy{}\PYZhy{}\PYZhy{}\PYZhy{}\PYZhy{}\PYZhy{}\PYZhy{}\PYZhy{}}
         
         \PY{c+c1}{\PYZsh{} Write a pattern for adjective plus one or two nouns}
         \PY{n}{pattern} \PY{o}{=} \PY{p}{[}\PY{p}{[}\PY{p}{\PYZob{}}\PY{l+s+s2}{\PYZdq{}}\PY{l+s+s2}{POS}\PY{l+s+s2}{\PYZdq{}}\PY{p}{:} \PY{l+s+s2}{\PYZdq{}}\PY{l+s+s2}{ADJ}\PY{l+s+s2}{\PYZdq{}}\PY{p}{\PYZcb{}}\PY{p}{,} \PY{p}{\PYZob{}}\PY{l+s+s2}{\PYZdq{}}\PY{l+s+s2}{POS}\PY{l+s+s2}{\PYZdq{}}\PY{p}{:} \PY{l+s+s2}{\PYZdq{}}\PY{l+s+s2}{NOUN}\PY{l+s+s2}{\PYZdq{}}\PY{p}{\PYZcb{}}\PY{p}{,} \PY{p}{\PYZob{}}\PY{l+s+s2}{\PYZdq{}}\PY{l+s+s2}{POS}\PY{l+s+s2}{\PYZdq{}}\PY{p}{:} \PY{l+s+s2}{\PYZdq{}}\PY{l+s+s2}{NOUN}\PY{l+s+s2}{\PYZdq{}}\PY{p}{,} \PY{l+s+s2}{\PYZdq{}}\PY{l+s+s2}{OP}\PY{l+s+s2}{\PYZdq{}}\PY{p}{:} \PY{l+s+s2}{\PYZdq{}}\PY{l+s+s2}{?}\PY{l+s+s2}{\PYZdq{}}\PY{p}{\PYZcb{}}\PY{p}{]}\PY{p}{]}
         
         \PY{c+c1}{\PYZsh{} Add the pattern to the matcher and apply the matcher to the doc}
         \PY{n}{matcher}\PY{o}{.}\PY{n}{add}\PY{p}{(}\PY{l+s+s2}{\PYZdq{}}\PY{l+s+s2}{ADJECTIVE\PYZus{}AND\PYZus{}NOUN(S)}\PY{l+s+s2}{\PYZdq{}}\PY{p}{,}\PY{n}{pattern}\PY{p}{)}
         \PY{n}{matches} \PY{o}{=} \PY{n}{matcher}\PY{p}{(}\PY{n}{doc}\PY{p}{)}
         
         
         \PY{n+nb}{print}\PY{p}{(}\PY{l+s+s2}{\PYZdq{}}\PY{l+s+s2}{Total matches found:}\PY{l+s+s2}{\PYZdq{}}\PY{p}{,} \PY{n+nb}{len}\PY{p}{(}\PY{n}{matches}\PY{p}{)}\PY{p}{)}
         
         \PY{c+c1}{\PYZsh{} Iterate over the matches and print the span text}
         \PY{k}{for} \PY{n}{match\PYZus{}id}\PY{p}{,} \PY{n}{start}\PY{p}{,} \PY{n}{end} \PY{o+ow}{in} \PY{n}{matches}\PY{p}{:}
             \PY{n+nb}{print}\PY{p}{(}\PY{l+s+s2}{\PYZdq{}}\PY{l+s+s2}{Match found:}\PY{l+s+s2}{\PYZdq{}}\PY{p}{,} \PY{n}{doc}\PY{p}{[}\PY{n}{start}\PY{p}{:}\PY{n}{end}\PY{p}{]}\PY{o}{.}\PY{n}{text}\PY{p}{)}
\end{Verbatim}


    \begin{Verbatim}[commandchars=\\\{\}]
Total matches found: 5
Match found: beautiful design
Match found: smart search
Match found: automatic labels
Match found: optional voice
Match found: optional voice responses

    \end{Verbatim}

    \hypertarget{reflection}{%
\section{Reflection}\label{reflection}}

\begin{enumerate}
\def\labelenumi{\arabic{enumi}.}
\tightlist
\item
  What is spaCy?

  \begin{itemize}
  \tightlist
  \item
    Niinkuin jokaisen muunkin aihepiiriin työstämisen helpottamiseksi
    löytyy kirjastot, löytyy myös luonnollisen kielen käsittelyynkin.
    Kirjaston nimi on spaCy.
  \item
    SpaCy on siis luonnollisen kielen käsittelyyn suunniteltu kirjasto.
    Se sisältää paljon funktoita esimerkiksi datan käsittelyyn. SpaCy
    helpottaa ja nopeuttaa huomattavasti luonnollisen kielen käsittelyä
    python-ympäristössä.
  \end{itemize}
\item
  Why you are not able to repeat parts 1.4 - 1.8 with Finnish?

  \begin{itemize}
  \tightlist
  \item
    Suomen kielelle ei ole ainakaan vielä olemassa pipeline-pakettia
    luotuna. Kukaan ei ole ottanut sitä työkseen.
  \end{itemize}
\item
  What's not included in a model package that you can load into spaCy?

  \begin{itemize}
  \tightlist
  \item
    A meta file including the language, pipeline and license.
  \item
    Binary weights to make statistical predictions.
  \item
    \textbf{The labelled data that the model was trained on.}
    \textless{}= Ei kuulu
  \item
    Strings of the model's vocabulary and their hashes.
  \end{itemize}
\item
  What is \texttt{nlp}?

  \begin{itemize}
  \tightlist
  \item
    \texttt{nlp} funktio luo tekstistä (doc) objektin ja tokenoi
    tekstin.
  \end{itemize}
\end{enumerate}

    \emph{Your answers here\ldots{}}


    % Add a bibliography block to the postdoc
    
    
    
    \end{document}
