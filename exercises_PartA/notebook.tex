
% Default to the notebook output style

    


% Inherit from the specified cell style.




    
\documentclass[11pt]{article}

    
    
    \usepackage[T1]{fontenc}
    % Nicer default font (+ math font) than Computer Modern for most use cases
    \usepackage{mathpazo}

    % Basic figure setup, for now with no caption control since it's done
    % automatically by Pandoc (which extracts ![](path) syntax from Markdown).
    \usepackage{graphicx}
    % We will generate all images so they have a width \maxwidth. This means
    % that they will get their normal width if they fit onto the page, but
    % are scaled down if they would overflow the margins.
    \makeatletter
    \def\maxwidth{\ifdim\Gin@nat@width>\linewidth\linewidth
    \else\Gin@nat@width\fi}
    \makeatother
    \let\Oldincludegraphics\includegraphics
    % Set max figure width to be 80% of text width, for now hardcoded.
    \renewcommand{\includegraphics}[1]{\Oldincludegraphics[width=.8\maxwidth]{#1}}
    % Ensure that by default, figures have no caption (until we provide a
    % proper Figure object with a Caption API and a way to capture that
    % in the conversion process - todo).
    \usepackage{caption}
    \DeclareCaptionLabelFormat{nolabel}{}
    \captionsetup{labelformat=nolabel}

    \usepackage{adjustbox} % Used to constrain images to a maximum size 
    \usepackage{xcolor} % Allow colors to be defined
    \usepackage{enumerate} % Needed for markdown enumerations to work
    \usepackage{geometry} % Used to adjust the document margins
    \usepackage{amsmath} % Equations
    \usepackage{amssymb} % Equations
    \usepackage{textcomp} % defines textquotesingle
    % Hack from http://tex.stackexchange.com/a/47451/13684:
    \AtBeginDocument{%
        \def\PYZsq{\textquotesingle}% Upright quotes in Pygmentized code
    }
    \usepackage{upquote} % Upright quotes for verbatim code
    \usepackage{eurosym} % defines \euro
    \usepackage[mathletters]{ucs} % Extended unicode (utf-8) support
    \usepackage[utf8x]{inputenc} % Allow utf-8 characters in the tex document
    \usepackage{fancyvrb} % verbatim replacement that allows latex
    \usepackage{grffile} % extends the file name processing of package graphics 
                         % to support a larger range 
    % The hyperref package gives us a pdf with properly built
    % internal navigation ('pdf bookmarks' for the table of contents,
    % internal cross-reference links, web links for URLs, etc.)
    \usepackage{hyperref}
    \usepackage{longtable} % longtable support required by pandoc >1.10
    \usepackage{booktabs}  % table support for pandoc > 1.12.2
    \usepackage[inline]{enumitem} % IRkernel/repr support (it uses the enumerate* environment)
    \usepackage[normalem]{ulem} % ulem is needed to support strikethroughs (\sout)
                                % normalem makes italics be italics, not underlines
    

    
    
    % Colors for the hyperref package
    \definecolor{urlcolor}{rgb}{0,.145,.698}
    \definecolor{linkcolor}{rgb}{.71,0.21,0.01}
    \definecolor{citecolor}{rgb}{.12,.54,.11}

    % ANSI colors
    \definecolor{ansi-black}{HTML}{3E424D}
    \definecolor{ansi-black-intense}{HTML}{282C36}
    \definecolor{ansi-red}{HTML}{E75C58}
    \definecolor{ansi-red-intense}{HTML}{B22B31}
    \definecolor{ansi-green}{HTML}{00A250}
    \definecolor{ansi-green-intense}{HTML}{007427}
    \definecolor{ansi-yellow}{HTML}{DDB62B}
    \definecolor{ansi-yellow-intense}{HTML}{B27D12}
    \definecolor{ansi-blue}{HTML}{208FFB}
    \definecolor{ansi-blue-intense}{HTML}{0065CA}
    \definecolor{ansi-magenta}{HTML}{D160C4}
    \definecolor{ansi-magenta-intense}{HTML}{A03196}
    \definecolor{ansi-cyan}{HTML}{60C6C8}
    \definecolor{ansi-cyan-intense}{HTML}{258F8F}
    \definecolor{ansi-white}{HTML}{C5C1B4}
    \definecolor{ansi-white-intense}{HTML}{A1A6B2}

    % commands and environments needed by pandoc snippets
    % extracted from the output of `pandoc -s`
    \providecommand{\tightlist}{%
      \setlength{\itemsep}{0pt}\setlength{\parskip}{0pt}}
    \DefineVerbatimEnvironment{Highlighting}{Verbatim}{commandchars=\\\{\}}
    % Add ',fontsize=\small' for more characters per line
    \newenvironment{Shaded}{}{}
    \newcommand{\KeywordTok}[1]{\textcolor[rgb]{0.00,0.44,0.13}{\textbf{{#1}}}}
    \newcommand{\DataTypeTok}[1]{\textcolor[rgb]{0.56,0.13,0.00}{{#1}}}
    \newcommand{\DecValTok}[1]{\textcolor[rgb]{0.25,0.63,0.44}{{#1}}}
    \newcommand{\BaseNTok}[1]{\textcolor[rgb]{0.25,0.63,0.44}{{#1}}}
    \newcommand{\FloatTok}[1]{\textcolor[rgb]{0.25,0.63,0.44}{{#1}}}
    \newcommand{\CharTok}[1]{\textcolor[rgb]{0.25,0.44,0.63}{{#1}}}
    \newcommand{\StringTok}[1]{\textcolor[rgb]{0.25,0.44,0.63}{{#1}}}
    \newcommand{\CommentTok}[1]{\textcolor[rgb]{0.38,0.63,0.69}{\textit{{#1}}}}
    \newcommand{\OtherTok}[1]{\textcolor[rgb]{0.00,0.44,0.13}{{#1}}}
    \newcommand{\AlertTok}[1]{\textcolor[rgb]{1.00,0.00,0.00}{\textbf{{#1}}}}
    \newcommand{\FunctionTok}[1]{\textcolor[rgb]{0.02,0.16,0.49}{{#1}}}
    \newcommand{\RegionMarkerTok}[1]{{#1}}
    \newcommand{\ErrorTok}[1]{\textcolor[rgb]{1.00,0.00,0.00}{\textbf{{#1}}}}
    \newcommand{\NormalTok}[1]{{#1}}
    
    % Additional commands for more recent versions of Pandoc
    \newcommand{\ConstantTok}[1]{\textcolor[rgb]{0.53,0.00,0.00}{{#1}}}
    \newcommand{\SpecialCharTok}[1]{\textcolor[rgb]{0.25,0.44,0.63}{{#1}}}
    \newcommand{\VerbatimStringTok}[1]{\textcolor[rgb]{0.25,0.44,0.63}{{#1}}}
    \newcommand{\SpecialStringTok}[1]{\textcolor[rgb]{0.73,0.40,0.53}{{#1}}}
    \newcommand{\ImportTok}[1]{{#1}}
    \newcommand{\DocumentationTok}[1]{\textcolor[rgb]{0.73,0.13,0.13}{\textit{{#1}}}}
    \newcommand{\AnnotationTok}[1]{\textcolor[rgb]{0.38,0.63,0.69}{\textbf{\textit{{#1}}}}}
    \newcommand{\CommentVarTok}[1]{\textcolor[rgb]{0.38,0.63,0.69}{\textbf{\textit{{#1}}}}}
    \newcommand{\VariableTok}[1]{\textcolor[rgb]{0.10,0.09,0.49}{{#1}}}
    \newcommand{\ControlFlowTok}[1]{\textcolor[rgb]{0.00,0.44,0.13}{\textbf{{#1}}}}
    \newcommand{\OperatorTok}[1]{\textcolor[rgb]{0.40,0.40,0.40}{{#1}}}
    \newcommand{\BuiltInTok}[1]{{#1}}
    \newcommand{\ExtensionTok}[1]{{#1}}
    \newcommand{\PreprocessorTok}[1]{\textcolor[rgb]{0.74,0.48,0.00}{{#1}}}
    \newcommand{\AttributeTok}[1]{\textcolor[rgb]{0.49,0.56,0.16}{{#1}}}
    \newcommand{\InformationTok}[1]{\textcolor[rgb]{0.38,0.63,0.69}{\textbf{\textit{{#1}}}}}
    \newcommand{\WarningTok}[1]{\textcolor[rgb]{0.38,0.63,0.69}{\textbf{\textit{{#1}}}}}
    
    
    % Define a nice break command that doesn't care if a line doesn't already
    % exist.
    \def\br{\hspace*{\fill} \\* }
    % Math Jax compatability definitions
    \def\gt{>}
    \def\lt{<}
    % Document parameters
    \title{Chapter4 - Alice}
    
    
    

    % Pygments definitions
    
\makeatletter
\def\PY@reset{\let\PY@it=\relax \let\PY@bf=\relax%
    \let\PY@ul=\relax \let\PY@tc=\relax%
    \let\PY@bc=\relax \let\PY@ff=\relax}
\def\PY@tok#1{\csname PY@tok@#1\endcsname}
\def\PY@toks#1+{\ifx\relax#1\empty\else%
    \PY@tok{#1}\expandafter\PY@toks\fi}
\def\PY@do#1{\PY@bc{\PY@tc{\PY@ul{%
    \PY@it{\PY@bf{\PY@ff{#1}}}}}}}
\def\PY#1#2{\PY@reset\PY@toks#1+\relax+\PY@do{#2}}

\expandafter\def\csname PY@tok@w\endcsname{\def\PY@tc##1{\textcolor[rgb]{0.73,0.73,0.73}{##1}}}
\expandafter\def\csname PY@tok@c\endcsname{\let\PY@it=\textit\def\PY@tc##1{\textcolor[rgb]{0.25,0.50,0.50}{##1}}}
\expandafter\def\csname PY@tok@cp\endcsname{\def\PY@tc##1{\textcolor[rgb]{0.74,0.48,0.00}{##1}}}
\expandafter\def\csname PY@tok@k\endcsname{\let\PY@bf=\textbf\def\PY@tc##1{\textcolor[rgb]{0.00,0.50,0.00}{##1}}}
\expandafter\def\csname PY@tok@kp\endcsname{\def\PY@tc##1{\textcolor[rgb]{0.00,0.50,0.00}{##1}}}
\expandafter\def\csname PY@tok@kt\endcsname{\def\PY@tc##1{\textcolor[rgb]{0.69,0.00,0.25}{##1}}}
\expandafter\def\csname PY@tok@o\endcsname{\def\PY@tc##1{\textcolor[rgb]{0.40,0.40,0.40}{##1}}}
\expandafter\def\csname PY@tok@ow\endcsname{\let\PY@bf=\textbf\def\PY@tc##1{\textcolor[rgb]{0.67,0.13,1.00}{##1}}}
\expandafter\def\csname PY@tok@nb\endcsname{\def\PY@tc##1{\textcolor[rgb]{0.00,0.50,0.00}{##1}}}
\expandafter\def\csname PY@tok@nf\endcsname{\def\PY@tc##1{\textcolor[rgb]{0.00,0.00,1.00}{##1}}}
\expandafter\def\csname PY@tok@nc\endcsname{\let\PY@bf=\textbf\def\PY@tc##1{\textcolor[rgb]{0.00,0.00,1.00}{##1}}}
\expandafter\def\csname PY@tok@nn\endcsname{\let\PY@bf=\textbf\def\PY@tc##1{\textcolor[rgb]{0.00,0.00,1.00}{##1}}}
\expandafter\def\csname PY@tok@ne\endcsname{\let\PY@bf=\textbf\def\PY@tc##1{\textcolor[rgb]{0.82,0.25,0.23}{##1}}}
\expandafter\def\csname PY@tok@nv\endcsname{\def\PY@tc##1{\textcolor[rgb]{0.10,0.09,0.49}{##1}}}
\expandafter\def\csname PY@tok@no\endcsname{\def\PY@tc##1{\textcolor[rgb]{0.53,0.00,0.00}{##1}}}
\expandafter\def\csname PY@tok@nl\endcsname{\def\PY@tc##1{\textcolor[rgb]{0.63,0.63,0.00}{##1}}}
\expandafter\def\csname PY@tok@ni\endcsname{\let\PY@bf=\textbf\def\PY@tc##1{\textcolor[rgb]{0.60,0.60,0.60}{##1}}}
\expandafter\def\csname PY@tok@na\endcsname{\def\PY@tc##1{\textcolor[rgb]{0.49,0.56,0.16}{##1}}}
\expandafter\def\csname PY@tok@nt\endcsname{\let\PY@bf=\textbf\def\PY@tc##1{\textcolor[rgb]{0.00,0.50,0.00}{##1}}}
\expandafter\def\csname PY@tok@nd\endcsname{\def\PY@tc##1{\textcolor[rgb]{0.67,0.13,1.00}{##1}}}
\expandafter\def\csname PY@tok@s\endcsname{\def\PY@tc##1{\textcolor[rgb]{0.73,0.13,0.13}{##1}}}
\expandafter\def\csname PY@tok@sd\endcsname{\let\PY@it=\textit\def\PY@tc##1{\textcolor[rgb]{0.73,0.13,0.13}{##1}}}
\expandafter\def\csname PY@tok@si\endcsname{\let\PY@bf=\textbf\def\PY@tc##1{\textcolor[rgb]{0.73,0.40,0.53}{##1}}}
\expandafter\def\csname PY@tok@se\endcsname{\let\PY@bf=\textbf\def\PY@tc##1{\textcolor[rgb]{0.73,0.40,0.13}{##1}}}
\expandafter\def\csname PY@tok@sr\endcsname{\def\PY@tc##1{\textcolor[rgb]{0.73,0.40,0.53}{##1}}}
\expandafter\def\csname PY@tok@ss\endcsname{\def\PY@tc##1{\textcolor[rgb]{0.10,0.09,0.49}{##1}}}
\expandafter\def\csname PY@tok@sx\endcsname{\def\PY@tc##1{\textcolor[rgb]{0.00,0.50,0.00}{##1}}}
\expandafter\def\csname PY@tok@m\endcsname{\def\PY@tc##1{\textcolor[rgb]{0.40,0.40,0.40}{##1}}}
\expandafter\def\csname PY@tok@gh\endcsname{\let\PY@bf=\textbf\def\PY@tc##1{\textcolor[rgb]{0.00,0.00,0.50}{##1}}}
\expandafter\def\csname PY@tok@gu\endcsname{\let\PY@bf=\textbf\def\PY@tc##1{\textcolor[rgb]{0.50,0.00,0.50}{##1}}}
\expandafter\def\csname PY@tok@gd\endcsname{\def\PY@tc##1{\textcolor[rgb]{0.63,0.00,0.00}{##1}}}
\expandafter\def\csname PY@tok@gi\endcsname{\def\PY@tc##1{\textcolor[rgb]{0.00,0.63,0.00}{##1}}}
\expandafter\def\csname PY@tok@gr\endcsname{\def\PY@tc##1{\textcolor[rgb]{1.00,0.00,0.00}{##1}}}
\expandafter\def\csname PY@tok@ge\endcsname{\let\PY@it=\textit}
\expandafter\def\csname PY@tok@gs\endcsname{\let\PY@bf=\textbf}
\expandafter\def\csname PY@tok@gp\endcsname{\let\PY@bf=\textbf\def\PY@tc##1{\textcolor[rgb]{0.00,0.00,0.50}{##1}}}
\expandafter\def\csname PY@tok@go\endcsname{\def\PY@tc##1{\textcolor[rgb]{0.53,0.53,0.53}{##1}}}
\expandafter\def\csname PY@tok@gt\endcsname{\def\PY@tc##1{\textcolor[rgb]{0.00,0.27,0.87}{##1}}}
\expandafter\def\csname PY@tok@err\endcsname{\def\PY@bc##1{\setlength{\fboxsep}{0pt}\fcolorbox[rgb]{1.00,0.00,0.00}{1,1,1}{\strut ##1}}}
\expandafter\def\csname PY@tok@kc\endcsname{\let\PY@bf=\textbf\def\PY@tc##1{\textcolor[rgb]{0.00,0.50,0.00}{##1}}}
\expandafter\def\csname PY@tok@kd\endcsname{\let\PY@bf=\textbf\def\PY@tc##1{\textcolor[rgb]{0.00,0.50,0.00}{##1}}}
\expandafter\def\csname PY@tok@kn\endcsname{\let\PY@bf=\textbf\def\PY@tc##1{\textcolor[rgb]{0.00,0.50,0.00}{##1}}}
\expandafter\def\csname PY@tok@kr\endcsname{\let\PY@bf=\textbf\def\PY@tc##1{\textcolor[rgb]{0.00,0.50,0.00}{##1}}}
\expandafter\def\csname PY@tok@bp\endcsname{\def\PY@tc##1{\textcolor[rgb]{0.00,0.50,0.00}{##1}}}
\expandafter\def\csname PY@tok@fm\endcsname{\def\PY@tc##1{\textcolor[rgb]{0.00,0.00,1.00}{##1}}}
\expandafter\def\csname PY@tok@vc\endcsname{\def\PY@tc##1{\textcolor[rgb]{0.10,0.09,0.49}{##1}}}
\expandafter\def\csname PY@tok@vg\endcsname{\def\PY@tc##1{\textcolor[rgb]{0.10,0.09,0.49}{##1}}}
\expandafter\def\csname PY@tok@vi\endcsname{\def\PY@tc##1{\textcolor[rgb]{0.10,0.09,0.49}{##1}}}
\expandafter\def\csname PY@tok@vm\endcsname{\def\PY@tc##1{\textcolor[rgb]{0.10,0.09,0.49}{##1}}}
\expandafter\def\csname PY@tok@sa\endcsname{\def\PY@tc##1{\textcolor[rgb]{0.73,0.13,0.13}{##1}}}
\expandafter\def\csname PY@tok@sb\endcsname{\def\PY@tc##1{\textcolor[rgb]{0.73,0.13,0.13}{##1}}}
\expandafter\def\csname PY@tok@sc\endcsname{\def\PY@tc##1{\textcolor[rgb]{0.73,0.13,0.13}{##1}}}
\expandafter\def\csname PY@tok@dl\endcsname{\def\PY@tc##1{\textcolor[rgb]{0.73,0.13,0.13}{##1}}}
\expandafter\def\csname PY@tok@s2\endcsname{\def\PY@tc##1{\textcolor[rgb]{0.73,0.13,0.13}{##1}}}
\expandafter\def\csname PY@tok@sh\endcsname{\def\PY@tc##1{\textcolor[rgb]{0.73,0.13,0.13}{##1}}}
\expandafter\def\csname PY@tok@s1\endcsname{\def\PY@tc##1{\textcolor[rgb]{0.73,0.13,0.13}{##1}}}
\expandafter\def\csname PY@tok@mb\endcsname{\def\PY@tc##1{\textcolor[rgb]{0.40,0.40,0.40}{##1}}}
\expandafter\def\csname PY@tok@mf\endcsname{\def\PY@tc##1{\textcolor[rgb]{0.40,0.40,0.40}{##1}}}
\expandafter\def\csname PY@tok@mh\endcsname{\def\PY@tc##1{\textcolor[rgb]{0.40,0.40,0.40}{##1}}}
\expandafter\def\csname PY@tok@mi\endcsname{\def\PY@tc##1{\textcolor[rgb]{0.40,0.40,0.40}{##1}}}
\expandafter\def\csname PY@tok@il\endcsname{\def\PY@tc##1{\textcolor[rgb]{0.40,0.40,0.40}{##1}}}
\expandafter\def\csname PY@tok@mo\endcsname{\def\PY@tc##1{\textcolor[rgb]{0.40,0.40,0.40}{##1}}}
\expandafter\def\csname PY@tok@ch\endcsname{\let\PY@it=\textit\def\PY@tc##1{\textcolor[rgb]{0.25,0.50,0.50}{##1}}}
\expandafter\def\csname PY@tok@cm\endcsname{\let\PY@it=\textit\def\PY@tc##1{\textcolor[rgb]{0.25,0.50,0.50}{##1}}}
\expandafter\def\csname PY@tok@cpf\endcsname{\let\PY@it=\textit\def\PY@tc##1{\textcolor[rgb]{0.25,0.50,0.50}{##1}}}
\expandafter\def\csname PY@tok@c1\endcsname{\let\PY@it=\textit\def\PY@tc##1{\textcolor[rgb]{0.25,0.50,0.50}{##1}}}
\expandafter\def\csname PY@tok@cs\endcsname{\let\PY@it=\textit\def\PY@tc##1{\textcolor[rgb]{0.25,0.50,0.50}{##1}}}

\def\PYZbs{\char`\\}
\def\PYZus{\char`\_}
\def\PYZob{\char`\{}
\def\PYZcb{\char`\}}
\def\PYZca{\char`\^}
\def\PYZam{\char`\&}
\def\PYZlt{\char`\<}
\def\PYZgt{\char`\>}
\def\PYZsh{\char`\#}
\def\PYZpc{\char`\%}
\def\PYZdl{\char`\$}
\def\PYZhy{\char`\-}
\def\PYZsq{\char`\'}
\def\PYZdq{\char`\"}
\def\PYZti{\char`\~}
% for compatibility with earlier versions
\def\PYZat{@}
\def\PYZlb{[}
\def\PYZrb{]}
\makeatother


    % Exact colors from NB
    \definecolor{incolor}{rgb}{0.0, 0.0, 0.5}
    \definecolor{outcolor}{rgb}{0.545, 0.0, 0.0}



    
    % Prevent overflowing lines due to hard-to-break entities
    \sloppy 
    % Setup hyperref package
    \hypersetup{
      breaklinks=true,  % so long urls are correctly broken across lines
      colorlinks=true,
      urlcolor=urlcolor,
      linkcolor=linkcolor,
      citecolor=citecolor,
      }
    % Slightly bigger margins than the latex defaults
    
    \geometry{verbose,tmargin=1in,bmargin=1in,lmargin=1in,rmargin=1in}
    
    

    \begin{document}
    
    
    \maketitle
    
    

    
    \hypertarget{chapter4---alice-in-wonderland-analyst}{%
\section{Chapter4 - Alice in Wonderland
analyst}\label{chapter4---alice-in-wonderland-analyst}}

    \begin{Verbatim}[commandchars=\\\{\}]
{\color{incolor}In [{\color{incolor}1}]:} \PY{k+kn}{import} \PY{n+nn}{requests}
        \PY{k+kn}{from} \PY{n+nn}{bs4} \PY{k}{import} \PY{n}{BeautifulSoup} \PY{k}{as} \PY{n}{b}
        \PY{k+kn}{from} \PY{n+nn}{spacy}\PY{n+nn}{.}\PY{n+nn}{lang}\PY{n+nn}{.}\PY{n+nn}{en} \PY{k}{import} \PY{n}{English}
        \PY{k+kn}{import} \PY{n+nn}{pickle}
        \PY{k+kn}{import} \PY{n+nn}{numpy} \PY{k}{as} \PY{n+nn}{np}
        \PY{k+kn}{import} \PY{n+nn}{pandas} \PY{k}{as} \PY{n+nn}{pd}
        \PY{k+kn}{from} \PY{n+nn}{xml}\PY{n+nn}{.}\PY{n+nn}{etree}\PY{n+nn}{.}\PY{n+nn}{ElementTree} \PY{k}{import} \PY{n}{parse}
\end{Verbatim}


    Ohjeita: - {[}x{]} Poista kirjan alusta ja lopusta The Project Gutenberg
-sivuston lisäämät otsikot ja huomautukset.

Kysymykset: - {[}x{]} \textbf{Kuinka monta tokenia}
``en\_core\_web\_lg'' -pipeline löytää dokumentista? Entä kuinka monta
näistä on \textbf{``stop word''} tokenia? - {[}x{]} Kuinka monta
\textbf{henkilö tokenia} (PERSON) ``en\_core\_web\_lg'' -pipeline löytää
kirjasta? (Eli kuinka monta kertaa kirjassa mainitaan joku henkilö.) -
{[}x{]} Kuinka monta kertaa näistä mainitaan \textbf{Alice}?

    \begin{Verbatim}[commandchars=\\\{\}]
{\color{incolor}In [{\color{incolor}2}]:} \PY{c+c1}{\PYZsh{} Funktio, joka hakee parametriksi annettusta osotteesta tekstin}
        \PY{k}{def} \PY{n+nf}{alice\PYZus{}text}\PY{p}{(}\PY{n}{url}\PY{p}{)}\PY{p}{:}
            \PY{n}{r} \PY{o}{=} \PY{n}{requests}\PY{o}{.}\PY{n}{get}\PY{p}{(}\PY{n}{url}\PY{p}{)}
            \PY{n}{soup} \PY{o}{=} \PY{n}{b}\PY{p}{(}\PY{n}{r}\PY{o}{.}\PY{n}{content}\PY{p}{,} \PY{l+s+s1}{\PYZsq{}}\PY{l+s+s1}{html.parser}\PY{l+s+s1}{\PYZsq{}}\PY{p}{)}
            \PY{n}{text} \PY{o}{=} \PY{n}{soup}\PY{o}{.}\PY{n}{get\PYZus{}text}\PY{p}{(}\PY{p}{)}
            \PY{c+c1}{\PYZsh{} Korvataan \PYZbs{}n ja \PYZbs{}r merkinnät tyhjällä}
            \PY{n}{text} \PY{o}{=} \PY{n}{text}\PY{o}{.}\PY{n}{strip}\PY{p}{(}\PY{p}{)}\PY{o}{.}\PY{n}{replace}\PY{p}{(}\PY{l+s+s2}{\PYZdq{}}\PY{l+s+se}{\PYZbs{}n}\PY{l+s+s2}{\PYZdq{}}\PY{p}{,} \PY{l+s+s2}{\PYZdq{}}\PY{l+s+s2}{ }\PY{l+s+s2}{\PYZdq{}}\PY{p}{)}\PY{o}{.}\PY{n}{replace}\PY{p}{(}\PY{l+s+s2}{\PYZdq{}}\PY{l+s+se}{\PYZbs{}r}\PY{l+s+s2}{\PYZdq{}}\PY{p}{,} \PY{l+s+s2}{\PYZdq{}}\PY{l+s+s2}{ }\PY{l+s+s2}{\PYZdq{}}\PY{p}{)}
            \PY{k}{return} \PY{n}{text}
\end{Verbatim}


    \begin{Verbatim}[commandchars=\\\{\}]
{\color{incolor}In [{\color{incolor}3}]:} \PY{n}{text} \PY{o}{=} \PY{n}{alice\PYZus{}text}\PY{p}{(}\PY{l+s+s2}{\PYZdq{}}\PY{l+s+s2}{http://www.gutenberg.org/files/11/11\PYZhy{}h/11\PYZhy{}h.htm}\PY{l+s+s2}{\PYZdq{}}\PY{p}{)}
\end{Verbatim}


    \begin{Verbatim}[commandchars=\\\{\}]
{\color{incolor}In [{\color{incolor}4}]:} \PY{k+kn}{import} \PY{n+nn}{spacy}
        \PY{k+kn}{from} \PY{n+nn}{spacy}\PY{n+nn}{.}\PY{n+nn}{lang}\PY{n+nn}{.}\PY{n+nn}{en} \PY{k}{import} \PY{n}{English}
        \PY{k+kn}{from} \PY{n+nn}{spacy}\PY{n+nn}{.}\PY{n+nn}{tokens} \PY{k}{import} \PY{n}{Doc}
\end{Verbatim}


    Ennen large-paketin käyttöä pitää ajaa terminaalissa komento:
\texttt{python\ -m\ spacy\ download\ en\_core\_lg} tai python-cellissä
\texttt{!python\ -m\ spacy\ download\ en\_core\_lg}

    \hypertarget{parsitaan-sivulta-ylimuxe4uxe4ruxe4iset-tekstit-pois}{%
\subsection{Parsitaan sivulta ylimääräiset tekstit
pois}\label{parsitaan-sivulta-ylimuxe4uxe4ruxe4iset-tekstit-pois}}

Jätetään pelkkä kirjan osuus. Tämä on hieman kömpelö tapa, mutta toimiva
eikä tässä kovin kauaa minulla mennyt.

    \begin{Verbatim}[commandchars=\\\{\}]
{\color{incolor}In [{\color{incolor}5}]:} \PY{n}{nlp} \PY{o}{=} \PY{n}{spacy}\PY{o}{.}\PY{n}{load}\PY{p}{(}\PY{l+s+s2}{\PYZdq{}}\PY{l+s+s2}{en\PYZus{}core\PYZus{}web\PYZus{}lg}\PY{l+s+s2}{\PYZdq{}}\PY{p}{)}
        \PY{n}{doc} \PY{o}{=} \PY{n}{nlp}\PY{p}{(}\PY{n}{text}\PY{p}{)}
\end{Verbatim}


    \begin{Verbatim}[commandchars=\\\{\}]
{\color{incolor}In [{\color{incolor}6}]:} \PY{c+c1}{\PYZsh{} Alku (\PYZdq{}Alice was beginning...\PYZdq{})}
        \PY{n}{doc}\PY{p}{[}\PY{l+m+mi}{729}\PY{p}{]}
\end{Verbatim}


\begin{Verbatim}[commandchars=\\\{\}]
{\color{outcolor}Out[{\color{outcolor}6}]:} Alice
\end{Verbatim}
            
    \begin{Verbatim}[commandchars=\\\{\}]
{\color{incolor}In [{\color{incolor}7}]:} \PY{c+c1}{\PYZsh{} Loppu (\PYZdq{}THE END\PYZdq{})}
        \PY{n}{doc}\PY{p}{[}\PY{o}{\PYZhy{}}\PY{l+m+mi}{3631}\PY{p}{]}
\end{Verbatim}


\begin{Verbatim}[commandchars=\\\{\}]
{\color{outcolor}Out[{\color{outcolor}7}]:} END
\end{Verbatim}
            
    \begin{Verbatim}[commandchars=\\\{\}]
{\color{incolor}In [{\color{incolor}8}]:} \PY{c+c1}{\PYZsh{} Määritetää dokumentti uudella alulla ja lopulla}
        \PY{n}{doc} \PY{o}{=} \PY{n}{doc}\PY{p}{[}\PY{l+m+mi}{729}\PY{p}{:}\PY{o}{\PYZhy{}}\PY{l+m+mi}{3630}\PY{p}{]}
\end{Verbatim}


    \hypertarget{tokens}{%
\subsection{Tokens}\label{tokens}}

    \begin{Verbatim}[commandchars=\\\{\}]
{\color{incolor}In [{\color{incolor}10}]:} \PY{n+nb}{print}\PY{p}{(}\PY{l+s+s2}{\PYZdq{}}\PY{l+s+s2}{Tokens:}\PY{l+s+s2}{\PYZdq{}}\PY{p}{,}\PY{n+nb}{len}\PY{p}{(}\PY{n}{doc}\PY{p}{)}\PY{p}{)}
         \PY{n+nb}{print}\PY{p}{(}\PY{l+s+s2}{\PYZdq{}}\PY{l+s+s2}{Stop Words:}\PY{l+s+s2}{\PYZdq{}}\PY{p}{,}\PY{n+nb}{len}\PY{p}{(}\PY{p}{[}\PY{n}{token}\PY{o}{.}\PY{n}{text} \PY{k}{for} \PY{n}{token} \PY{o+ow}{in} \PY{n}{doc} \PY{k}{if} \PY{n}{token}\PY{o}{.}\PY{n}{is\PYZus{}stop}\PY{p}{]}\PY{p}{)}\PY{p}{)}
         \PY{n+nb}{print}\PY{p}{(}\PY{n}{f}\PY{l+s+s2}{\PYZdq{}}\PY{l+s+s2}{Stop wordeja }\PY{l+s+s2}{\PYZob{}}\PY{l+s+s2}{(len([token.text for token in doc if token.is\PYZus{}stop])/len(doc))*100:.2f\PYZcb{}}\PY{l+s+s2}{\PYZpc{}}\PY{l+s+s2}{ kaikista tokeneista.}\PY{l+s+s2}{\PYZdq{}}\PY{p}{)}
\end{Verbatim}


    \begin{Verbatim}[commandchars=\\\{\}]
Tokens: 37241
Stop Words: 16939
Stop wordeja 45.48\% kaikista tokeneista.

    \end{Verbatim}

    \hypertarget{henkiluxf6t-tekstissuxe4}{%
\subsection{Henkilöt tekstissä}\label{henkiluxf6t-tekstissuxe4}}

    \begin{Verbatim}[commandchars=\\\{\}]
{\color{incolor}In [{\color{incolor}11}]:} \PY{c+c1}{\PYZsh{} Kuinka monta henkilöä \PYZdq{}en\PYZus{}core\PYZus{}web\PYZus{}lg\PYZdq{} \PYZhy{}pipeline löytää}
         \PY{n}{persons} \PY{o}{=} \PY{p}{[}\PY{n}{ent}\PY{o}{.}\PY{n}{text} \PY{k}{for} \PY{n}{ent} \PY{o+ow}{in} \PY{n}{doc}\PY{o}{.}\PY{n}{ents} \PY{k}{if} \PY{n}{ent}\PY{o}{.}\PY{n}{label\PYZus{}} \PY{o}{==} \PY{l+s+s1}{\PYZsq{}}\PY{l+s+s1}{PERSON}\PY{l+s+s1}{\PYZsq{}}\PY{p}{]}
         \PY{n+nb}{print}\PY{p}{(}\PY{l+s+s2}{\PYZdq{}}\PY{l+s+s2}{Henkilöitä löytyi}\PY{l+s+s2}{\PYZdq{}}\PY{p}{,}\PY{n+nb}{len}\PY{p}{(}\PY{n}{persons}\PY{p}{)}\PY{p}{,}\PY{l+s+s2}{\PYZdq{}}\PY{l+s+s2}{kappaletta.}\PY{l+s+s2}{\PYZdq{}}\PY{p}{)}
\end{Verbatim}


    \begin{Verbatim}[commandchars=\\\{\}]
Henkilöitä löytyi 575 kappaletta.

    \end{Verbatim}

    \begin{Verbatim}[commandchars=\\\{\}]
{\color{incolor}In [{\color{incolor}12}]:} \PY{c+c1}{\PYZsh{} Kuinka monta kertaa näistä mainitaan Alice?}
         \PY{n+nb}{print}\PY{p}{(}\PY{n}{persons}\PY{p}{)}
\end{Verbatim}


    \begin{Verbatim}[commandchars=\\\{\}]
['Alice', 'Alice', 'Alice', 'Alice', 'Alice', 'Alice', 'Alice', 'Alice', 'Alice', 'Alice', 'Dinah’ll', 'Alice  ', 'Alice', 'Alice', 'Alice  ', 'Alice', 'Alice', 'Alice', 'Alice', 'Alice', 'Alice  ', 'Alice  ', 'Alice', 'Alice', 'Alice', 'Alice', 'Alice', 'Alice', 'Alice', 'Alice', 'Alice', 'Alice', 'Alice', 'Alice', 'Alice', 'Alice', 'Alice', 'Mabel', 'Alice', 'Mabel  ', 'I’m Mabel', 'Alice', 'Alice', 'Alice', 'Alice', 'Alice', 'Alice', 'Alice', 'William', 'Alice', 'Où  est ma', 'Alice', 'Alice', 'Alice', 'Alice', 'Mouse', 'Mouse', 'Alice', 'Mouse', 'Alice', 'Alice', 'Mouse', 'Mouse', 'Mouse', 'Alice', 'Lory', 'Eaglet', 'Alice', 'Alice', 'Alice', 'Mouse', 'Alice', 'Edwin', 'Mercia  ', 'Mouse', 'Lory', 'Mouse', 'Mouse', 'Edgar Atheling', 'William', 'Alice', 'Alice', 'Eaglet', 'Eaglet', 'Alice', 'Shakespeare', 'Alice', 'Alice', 'Mouse', 'Alice', 'Alice', 'Alice', 'Mouse', 'Alice', 'Alice', 'Alice', 'Mouse', 'Fury', 'Mouse', 'Alice', 'Alice', 'Alice', 'Alice', 'Mouse', 'Alice', 'Alice', 'Alice', 'Dinah', 'Alice', 'Alice', 'Mouse', 'Alice', 'Alice', 'Mary Ann', 'Alice', '’d  ', 'Mary Ann', 'Alice', 'Dinah’ll', 'Alice', 'Alice  ', 'Alice', 'Alice', 'Alice', 'Alice', 'Mary Ann', 'Mary Ann', 'Alice  ', 'Alice', 'Alice', 'Alice', 'Pat', 'Pat', 'Alice', 'Alice', 'Bill', 'Bill', 'Bill', 'Bill', 'Bill', 'Bill', 'Bill', 'Bill', 'Brandy', 'Alice', 'Jack', 'Dinah  ', 'Alice', 'Alice', 'Alice', 'Lizard', 'Bill', 'Alice', 'Alice', 'Alice', 'Alice', 'Alice', 'Alice', 'Alice', '’d  ', 'Alice', 'CHAPTER V.  Advice', 'Alice', 'Alice', 'Alice', 'Alice', 'Alice', 'Alice', 'Alice', 'Alice', 'Alice', 'Alice', 'Alice', 'Alice', 'William', 'Alice', 'William', 'William', 'Alice', 'Alice', 'Alice', 'Alice', 'Alice', 'Alice', 'Alice', 'Alice', 'Alice', 'Alice', 'Alice', 'Alice', 'Alice', 'Alice', 'Alice', 'Alice', 'Alice', 'Alice', 'Alice', 'Alice', 'Alice', 'Alice', 'Alice', 'Alice', 'Alice', 'Footman', 'Alice', 'Alice', 'Footman', 'Alice', 'Alice  ', 'Alice', 'Alice', 'Alice', 'Alice', 'Alice', 'Alice', 'Alice', 'Alice', 'Alice', 'Alice', 'Alice', 'Alice', 'Alice', 'Alice', 'Alice', 'Alice', 'Alice', 'Cheshire Puss', 'Alice', 'Alice', 'Alice', 'Alice', 'Alice', 'Alice', 'Alice', 'Alice', 'Alice', 'Alice', 'Alice', 'Alice', 'Alice', 'Alice', 'Alice', 'Hatter', 'Hatter', 'Alice', 'Alice  ', 'Alice', 'Alice  ', 'Alice', 'Hatter', 'Alice', 'Hatter', 'Alice', 'Alice', 'Alice', 'Hatter', 'Dormouse', 'Hatter', 'Alice  ', 'Hatter', 'Alice', 'Alice', 'Hatter', 'Alice', 'Hatter', 'Alice', 'Hatter', 'Alice', 'Hatter', 'Hatter', 'Hatter', 'Alice', 'Hatter', 'Alice', 'Hatter', 'Alice', 'Hatter', 'Alice', 'Hatter', 'Alice', 'Hatter', 'Alice', 'Hatter', 'Alice', 'Hatter', 'Hatter', 'Alice', 'Alice', 'Hatter', 'Alice', 'Hatter', 'Alice  ', 'Alice', 'Alice', 'Hatter', 'Dormouse', 'Elsie', 'Alice', 'Dormouse', 'Alice', 'Dormouse', 'Alice', 'Alice', 'Alice', 'Hatter', 'Alice', 'Hatter', 'Alice', 'Alice', 'Dormouse', '’d', 'Alice', 'Dormouse', 'Alice', 'Dormouse', 'Hatter', 'Dormouse', 'Alice', 'Hatter', 'Alice', 'Alice', 'Hatter', 'Alice', 'Dormouse', 'Alice', 'Alice', 'Alice', 'Dormouse', 'Hatter', 'Alice', 'Hatter', 'Alice', 'Alice', 'Alice', 'Alice', 'Alice', 'Alice', 'Alice', 'Alice', 'Alice', 'Alice', 'Alice', 'Alice', 'Alice', 'Alice', 'Alice', 'Alice', 'Alice', 'Alice', 'Alice', 'Alice', 'Alice  ', 'Alice  ', 'Alice', 'Alice', 'Alice', 'Alice', 'Alice  ', 'Alice', 'Alice', 'Alice', 'Alice', 'Alice', 'Alice', 'Alice', 'Queen', 'Alice', 'Alice', 'Alice', 'Queen', 'Alice', '’d', 'The Mock Turtle’s Story', 'Alice', 'I’m a Duchess', 'Alice', 'Alice', 'Alice', 'Alice', 'Alice', 'Alice', 'Alice', 'Alice', 'Alice', 'Alice', 'Alice', 'Alice', 'Alice', 'Alice', 'Alice', 'Alice', 'Alice', 'Alice', 'Alice', 'Queen', 'Alice', 'Alice', 'the Mock Turtle', 'Alice', 'Mock Turtle Soup', 'Alice', 'Alice', 'Gryphon', 'Alice', 'Gryphon', 'Alice', 'Alice', 'Alice', 'Alice', 'the Mock Turtle', 'Alice', 'the Mock Turtle', 'the Mock Turtle', 'Alice', 'Alice  ', 'Alice', 'Alice', 'the Mock Turtle', 'Alice', 'The Mock Turtle', 'Alice', 'the Mock Turtle', 'Alice', 'the Mock Turtle', 'Alice', 'Alice', 'the Mock Turtle', 'Alice', 'Alice', 'Alice', 'Alice', 'the Mock Turtle', 'Alice', 'Gryphon', 'Gryphon', 'Alice', 'the Mock Turtle', 'Alice', 'Alice', 'the Mock Turtle', 'Alice', 'The Mock Turtle', 'Alice', 'the Mock Turtle', 'Alice', 'Alice', 'Gryphon', 'Gryphon', 'Gryphon', 'Gryphon', 'the Mock Turtle', 'Gryphon', 'Alice', 'Alice', 'the Mock Turtle', 'Alice', 'the Mock Turtle', 'Alice', 'the Mock Turtle', 'the Mock Turtle', 'Alice', 'Dinn', 'the Mock Turtle', 'Alice', 'the Mock Turtle', 'the Mock Turtle', 'Gryphon', 'Alice', 'Alice', 'Alice', 'Alice', 'Gryphon', 'Alice', 'Alice', 'Alice', 'the Mock Turtle', 'Alice', 'Alice', 'the Mock Turtle', 'Alice', 'William', 'Gryphon', 'Alice', 'Gryphon', 'Alice', 'Gryphon', 'the Mock Turtle', 'Alice', 'the Mock Turtle', 'Gryphon', 'Alice', 'Alice', 'Panther', 'Panther', '’d', 'Gryphon', 'the Mock Turtle', 'the Mock Turtle', 'Alice  replied', 'Beau', 'Beau', 'Beau', 'Beau', 'Gryphon', 'the Mock Turtle', 'Gryphon', 'Alice', 'Alice', 'Alice', 'Alice', 'Alice', 'Alice', 'Alice', 'Alice', 'Alice', 'Alice', 'Bill', 'Hatter', 'Hatter', 'Dormouse', 'Dormouse', 'Hatter', 'Hatter', 'Hatter', 'Hatter', 'Alice', 'Dormouse', 'Alice', 'Dormouse', 'Alice', 'Dormouse', 'Hatter', 'Hatter', 'Hatter', 'Hatter', 'Hatter', 'Dormouse', 'Hatter', 'Hatter', 'Hatter', 'Alice', 'Hatter', 'Alice', 'Hatter', 'Hatter', 'Hatter', 'Alice', 'Dormouse', 'Alice', 'Alice', 'Alice', 'Alice', 'Alice', 'Alice', 'Alice', 'Alice', 'Alice', 'Alice', 'Alice', 'Alice', 'Alice', 'Alice', 'Alice', 'Alice', 'Bill', 'Alice', 'Alice', 'Alice', 'Alice', 'Alice', 'Alice  ', 'Alice', 'Alice', 'Mock Turtle', 'Queen']

    \end{Verbatim}

    \begin{Verbatim}[commandchars=\\\{\}]
{\color{incolor}In [{\color{incolor}13}]:} \PY{c+c1}{\PYZsh{} TAPA 1 \PYZhy{} tavalliset listakikat}
         \PY{n}{alices} \PY{o}{=} \PY{p}{[}\PY{p}{]}
         \PY{k}{for} \PY{n}{alice} \PY{o+ow}{in} \PY{n}{persons}\PY{p}{:}
             \PY{k}{if} \PY{n}{alice} \PY{o}{==} \PY{l+s+s1}{\PYZsq{}}\PY{l+s+s1}{Alice}\PY{l+s+s1}{\PYZsq{}}\PY{p}{:}
                 \PY{n}{alices}\PY{o}{.}\PY{n}{append}\PY{p}{(}\PY{n}{alice}\PY{p}{)}
             \PY{k}{if} \PY{n}{alice} \PY{o}{==} \PY{l+s+s1}{\PYZsq{}}\PY{l+s+s1}{Alice  }\PY{l+s+s1}{\PYZsq{}}\PY{p}{:}
                 \PY{n}{alices}\PY{o}{.}\PY{n}{append}\PY{p}{(}\PY{n}{alice}\PY{p}{)}
         \PY{n+nb}{print}\PY{p}{(}\PY{l+s+s2}{\PYZdq{}}\PY{l+s+s2}{Alice löytyi}\PY{l+s+s2}{\PYZdq{}}\PY{p}{,}\PY{n+nb}{len}\PY{p}{(}\PY{n}{alices}\PY{p}{)}\PY{p}{,}\PY{l+s+s2}{\PYZdq{}}\PY{l+s+s2}{kertaa.}\PY{l+s+s2}{\PYZdq{}}\PY{p}{)}
         \PY{n+nb}{print}\PY{p}{(}\PY{n}{f}\PY{l+s+s2}{\PYZdq{}}\PY{l+s+s2}{Aliceita }\PY{l+s+s2}{\PYZob{}}\PY{l+s+s2}{(len(alices)/len(persons))*100:.2f\PYZcb{}}\PY{l+s+s2}{\PYZpc{}}\PY{l+s+s2}{ kaikista tekstin henkilöistä.}\PY{l+s+s2}{\PYZdq{}}\PY{p}{)}
\end{Verbatim}


    \begin{Verbatim}[commandchars=\\\{\}]
Alice löytyi 376 kertaa.
Aliceita 65.39\% kaikista tekstin henkilöistä.

    \end{Verbatim}

    \begin{Verbatim}[commandchars=\\\{\}]
{\color{incolor}In [{\color{incolor}14}]:} \PY{c+c1}{\PYZsh{} TAPA 2 \PYZhy{} hyödynnetään spacya, koska kyse on kuitenkin nlp:stä}
         \PY{k+kn}{from} \PY{n+nn}{spacy}\PY{n+nn}{.}\PY{n+nn}{matcher} \PY{k}{import} \PY{n}{Matcher}
         
         \PY{n}{matcher} \PY{o}{=} \PY{n}{Matcher}\PY{p}{(}\PY{n}{nlp}\PY{o}{.}\PY{n}{vocab}\PY{p}{)}
         \PY{n}{pattern} \PY{o}{=} \PY{p}{[}\PY{p}{[}\PY{p}{\PYZob{}}\PY{l+s+s2}{\PYZdq{}}\PY{l+s+s2}{LOWER}\PY{l+s+s2}{\PYZdq{}}\PY{p}{:} \PY{l+s+s2}{\PYZdq{}}\PY{l+s+s2}{alice}\PY{l+s+s2}{\PYZdq{}}\PY{p}{\PYZcb{}}\PY{p}{]}\PY{p}{]}
         
         \PY{n}{matcher}\PY{o}{.}\PY{n}{add}\PY{p}{(}\PY{l+s+s2}{\PYZdq{}}\PY{l+s+s2}{ALICE}\PY{l+s+s2}{\PYZdq{}}\PY{p}{,} \PY{n}{pattern}\PY{p}{)}
         \PY{n}{matches} \PY{o}{=} \PY{n}{matcher}\PY{p}{(}\PY{n}{doc}\PY{p}{)}
         \PY{n+nb}{print}\PY{p}{(}\PY{l+s+s2}{\PYZdq{}}\PY{l+s+s2}{Alice löytyi:}\PY{l+s+s2}{\PYZdq{}}\PY{p}{,} \PY{n+nb}{len}\PY{p}{(}\PY{n}{matches}\PY{p}{)}\PY{p}{,}\PY{l+s+s2}{\PYZdq{}}\PY{l+s+s2}{kertaa.}\PY{l+s+s2}{\PYZdq{}}\PY{p}{)}
\end{Verbatim}


    \begin{Verbatim}[commandchars=\\\{\}]
Alice löytyi: 396 kertaa.

    \end{Verbatim}

    Jälkimmäisellä tavalla löytyi parisenkymmentä Alicea enemmän.


    % Add a bibliography block to the postdoc
    
    
    
    \end{document}
