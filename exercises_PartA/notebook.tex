
% Default to the notebook output style

    


% Inherit from the specified cell style.




    
\documentclass[11pt]{article}

    
    
    \usepackage[T1]{fontenc}
    % Nicer default font (+ math font) than Computer Modern for most use cases
    \usepackage{mathpazo}

    % Basic figure setup, for now with no caption control since it's done
    % automatically by Pandoc (which extracts ![](path) syntax from Markdown).
    \usepackage{graphicx}
    % We will generate all images so they have a width \maxwidth. This means
    % that they will get their normal width if they fit onto the page, but
    % are scaled down if they would overflow the margins.
    \makeatletter
    \def\maxwidth{\ifdim\Gin@nat@width>\linewidth\linewidth
    \else\Gin@nat@width\fi}
    \makeatother
    \let\Oldincludegraphics\includegraphics
    % Set max figure width to be 80% of text width, for now hardcoded.
    \renewcommand{\includegraphics}[1]{\Oldincludegraphics[width=.8\maxwidth]{#1}}
    % Ensure that by default, figures have no caption (until we provide a
    % proper Figure object with a Caption API and a way to capture that
    % in the conversion process - todo).
    \usepackage{caption}
    \DeclareCaptionLabelFormat{nolabel}{}
    \captionsetup{labelformat=nolabel}

    \usepackage{adjustbox} % Used to constrain images to a maximum size 
    \usepackage{xcolor} % Allow colors to be defined
    \usepackage{enumerate} % Needed for markdown enumerations to work
    \usepackage{geometry} % Used to adjust the document margins
    \usepackage{amsmath} % Equations
    \usepackage{amssymb} % Equations
    \usepackage{textcomp} % defines textquotesingle
    % Hack from http://tex.stackexchange.com/a/47451/13684:
    \AtBeginDocument{%
        \def\PYZsq{\textquotesingle}% Upright quotes in Pygmentized code
    }
    \usepackage{upquote} % Upright quotes for verbatim code
    \usepackage{eurosym} % defines \euro
    \usepackage[mathletters]{ucs} % Extended unicode (utf-8) support
    \usepackage[utf8x]{inputenc} % Allow utf-8 characters in the tex document
    \usepackage{fancyvrb} % verbatim replacement that allows latex
    \usepackage{grffile} % extends the file name processing of package graphics 
                         % to support a larger range 
    % The hyperref package gives us a pdf with properly built
    % internal navigation ('pdf bookmarks' for the table of contents,
    % internal cross-reference links, web links for URLs, etc.)
    \usepackage{hyperref}
    \usepackage{longtable} % longtable support required by pandoc >1.10
    \usepackage{booktabs}  % table support for pandoc > 1.12.2
    \usepackage[inline]{enumitem} % IRkernel/repr support (it uses the enumerate* environment)
    \usepackage[normalem]{ulem} % ulem is needed to support strikethroughs (\sout)
                                % normalem makes italics be italics, not underlines
    

    
    
    % Colors for the hyperref package
    \definecolor{urlcolor}{rgb}{0,.145,.698}
    \definecolor{linkcolor}{rgb}{.71,0.21,0.01}
    \definecolor{citecolor}{rgb}{.12,.54,.11}

    % ANSI colors
    \definecolor{ansi-black}{HTML}{3E424D}
    \definecolor{ansi-black-intense}{HTML}{282C36}
    \definecolor{ansi-red}{HTML}{E75C58}
    \definecolor{ansi-red-intense}{HTML}{B22B31}
    \definecolor{ansi-green}{HTML}{00A250}
    \definecolor{ansi-green-intense}{HTML}{007427}
    \definecolor{ansi-yellow}{HTML}{DDB62B}
    \definecolor{ansi-yellow-intense}{HTML}{B27D12}
    \definecolor{ansi-blue}{HTML}{208FFB}
    \definecolor{ansi-blue-intense}{HTML}{0065CA}
    \definecolor{ansi-magenta}{HTML}{D160C4}
    \definecolor{ansi-magenta-intense}{HTML}{A03196}
    \definecolor{ansi-cyan}{HTML}{60C6C8}
    \definecolor{ansi-cyan-intense}{HTML}{258F8F}
    \definecolor{ansi-white}{HTML}{C5C1B4}
    \definecolor{ansi-white-intense}{HTML}{A1A6B2}

    % commands and environments needed by pandoc snippets
    % extracted from the output of `pandoc -s`
    \providecommand{\tightlist}{%
      \setlength{\itemsep}{0pt}\setlength{\parskip}{0pt}}
    \DefineVerbatimEnvironment{Highlighting}{Verbatim}{commandchars=\\\{\}}
    % Add ',fontsize=\small' for more characters per line
    \newenvironment{Shaded}{}{}
    \newcommand{\KeywordTok}[1]{\textcolor[rgb]{0.00,0.44,0.13}{\textbf{{#1}}}}
    \newcommand{\DataTypeTok}[1]{\textcolor[rgb]{0.56,0.13,0.00}{{#1}}}
    \newcommand{\DecValTok}[1]{\textcolor[rgb]{0.25,0.63,0.44}{{#1}}}
    \newcommand{\BaseNTok}[1]{\textcolor[rgb]{0.25,0.63,0.44}{{#1}}}
    \newcommand{\FloatTok}[1]{\textcolor[rgb]{0.25,0.63,0.44}{{#1}}}
    \newcommand{\CharTok}[1]{\textcolor[rgb]{0.25,0.44,0.63}{{#1}}}
    \newcommand{\StringTok}[1]{\textcolor[rgb]{0.25,0.44,0.63}{{#1}}}
    \newcommand{\CommentTok}[1]{\textcolor[rgb]{0.38,0.63,0.69}{\textit{{#1}}}}
    \newcommand{\OtherTok}[1]{\textcolor[rgb]{0.00,0.44,0.13}{{#1}}}
    \newcommand{\AlertTok}[1]{\textcolor[rgb]{1.00,0.00,0.00}{\textbf{{#1}}}}
    \newcommand{\FunctionTok}[1]{\textcolor[rgb]{0.02,0.16,0.49}{{#1}}}
    \newcommand{\RegionMarkerTok}[1]{{#1}}
    \newcommand{\ErrorTok}[1]{\textcolor[rgb]{1.00,0.00,0.00}{\textbf{{#1}}}}
    \newcommand{\NormalTok}[1]{{#1}}
    
    % Additional commands for more recent versions of Pandoc
    \newcommand{\ConstantTok}[1]{\textcolor[rgb]{0.53,0.00,0.00}{{#1}}}
    \newcommand{\SpecialCharTok}[1]{\textcolor[rgb]{0.25,0.44,0.63}{{#1}}}
    \newcommand{\VerbatimStringTok}[1]{\textcolor[rgb]{0.25,0.44,0.63}{{#1}}}
    \newcommand{\SpecialStringTok}[1]{\textcolor[rgb]{0.73,0.40,0.53}{{#1}}}
    \newcommand{\ImportTok}[1]{{#1}}
    \newcommand{\DocumentationTok}[1]{\textcolor[rgb]{0.73,0.13,0.13}{\textit{{#1}}}}
    \newcommand{\AnnotationTok}[1]{\textcolor[rgb]{0.38,0.63,0.69}{\textbf{\textit{{#1}}}}}
    \newcommand{\CommentVarTok}[1]{\textcolor[rgb]{0.38,0.63,0.69}{\textbf{\textit{{#1}}}}}
    \newcommand{\VariableTok}[1]{\textcolor[rgb]{0.10,0.09,0.49}{{#1}}}
    \newcommand{\ControlFlowTok}[1]{\textcolor[rgb]{0.00,0.44,0.13}{\textbf{{#1}}}}
    \newcommand{\OperatorTok}[1]{\textcolor[rgb]{0.40,0.40,0.40}{{#1}}}
    \newcommand{\BuiltInTok}[1]{{#1}}
    \newcommand{\ExtensionTok}[1]{{#1}}
    \newcommand{\PreprocessorTok}[1]{\textcolor[rgb]{0.74,0.48,0.00}{{#1}}}
    \newcommand{\AttributeTok}[1]{\textcolor[rgb]{0.49,0.56,0.16}{{#1}}}
    \newcommand{\InformationTok}[1]{\textcolor[rgb]{0.38,0.63,0.69}{\textbf{\textit{{#1}}}}}
    \newcommand{\WarningTok}[1]{\textcolor[rgb]{0.38,0.63,0.69}{\textbf{\textit{{#1}}}}}
    
    
    % Define a nice break command that doesn't care if a line doesn't already
    % exist.
    \def\br{\hspace*{\fill} \\* }
    % Math Jax compatability definitions
    \def\gt{>}
    \def\lt{<}
    % Document parameters
    \title{Chapter2}
    
    
    

    % Pygments definitions
    
\makeatletter
\def\PY@reset{\let\PY@it=\relax \let\PY@bf=\relax%
    \let\PY@ul=\relax \let\PY@tc=\relax%
    \let\PY@bc=\relax \let\PY@ff=\relax}
\def\PY@tok#1{\csname PY@tok@#1\endcsname}
\def\PY@toks#1+{\ifx\relax#1\empty\else%
    \PY@tok{#1}\expandafter\PY@toks\fi}
\def\PY@do#1{\PY@bc{\PY@tc{\PY@ul{%
    \PY@it{\PY@bf{\PY@ff{#1}}}}}}}
\def\PY#1#2{\PY@reset\PY@toks#1+\relax+\PY@do{#2}}

\expandafter\def\csname PY@tok@w\endcsname{\def\PY@tc##1{\textcolor[rgb]{0.73,0.73,0.73}{##1}}}
\expandafter\def\csname PY@tok@c\endcsname{\let\PY@it=\textit\def\PY@tc##1{\textcolor[rgb]{0.25,0.50,0.50}{##1}}}
\expandafter\def\csname PY@tok@cp\endcsname{\def\PY@tc##1{\textcolor[rgb]{0.74,0.48,0.00}{##1}}}
\expandafter\def\csname PY@tok@k\endcsname{\let\PY@bf=\textbf\def\PY@tc##1{\textcolor[rgb]{0.00,0.50,0.00}{##1}}}
\expandafter\def\csname PY@tok@kp\endcsname{\def\PY@tc##1{\textcolor[rgb]{0.00,0.50,0.00}{##1}}}
\expandafter\def\csname PY@tok@kt\endcsname{\def\PY@tc##1{\textcolor[rgb]{0.69,0.00,0.25}{##1}}}
\expandafter\def\csname PY@tok@o\endcsname{\def\PY@tc##1{\textcolor[rgb]{0.40,0.40,0.40}{##1}}}
\expandafter\def\csname PY@tok@ow\endcsname{\let\PY@bf=\textbf\def\PY@tc##1{\textcolor[rgb]{0.67,0.13,1.00}{##1}}}
\expandafter\def\csname PY@tok@nb\endcsname{\def\PY@tc##1{\textcolor[rgb]{0.00,0.50,0.00}{##1}}}
\expandafter\def\csname PY@tok@nf\endcsname{\def\PY@tc##1{\textcolor[rgb]{0.00,0.00,1.00}{##1}}}
\expandafter\def\csname PY@tok@nc\endcsname{\let\PY@bf=\textbf\def\PY@tc##1{\textcolor[rgb]{0.00,0.00,1.00}{##1}}}
\expandafter\def\csname PY@tok@nn\endcsname{\let\PY@bf=\textbf\def\PY@tc##1{\textcolor[rgb]{0.00,0.00,1.00}{##1}}}
\expandafter\def\csname PY@tok@ne\endcsname{\let\PY@bf=\textbf\def\PY@tc##1{\textcolor[rgb]{0.82,0.25,0.23}{##1}}}
\expandafter\def\csname PY@tok@nv\endcsname{\def\PY@tc##1{\textcolor[rgb]{0.10,0.09,0.49}{##1}}}
\expandafter\def\csname PY@tok@no\endcsname{\def\PY@tc##1{\textcolor[rgb]{0.53,0.00,0.00}{##1}}}
\expandafter\def\csname PY@tok@nl\endcsname{\def\PY@tc##1{\textcolor[rgb]{0.63,0.63,0.00}{##1}}}
\expandafter\def\csname PY@tok@ni\endcsname{\let\PY@bf=\textbf\def\PY@tc##1{\textcolor[rgb]{0.60,0.60,0.60}{##1}}}
\expandafter\def\csname PY@tok@na\endcsname{\def\PY@tc##1{\textcolor[rgb]{0.49,0.56,0.16}{##1}}}
\expandafter\def\csname PY@tok@nt\endcsname{\let\PY@bf=\textbf\def\PY@tc##1{\textcolor[rgb]{0.00,0.50,0.00}{##1}}}
\expandafter\def\csname PY@tok@nd\endcsname{\def\PY@tc##1{\textcolor[rgb]{0.67,0.13,1.00}{##1}}}
\expandafter\def\csname PY@tok@s\endcsname{\def\PY@tc##1{\textcolor[rgb]{0.73,0.13,0.13}{##1}}}
\expandafter\def\csname PY@tok@sd\endcsname{\let\PY@it=\textit\def\PY@tc##1{\textcolor[rgb]{0.73,0.13,0.13}{##1}}}
\expandafter\def\csname PY@tok@si\endcsname{\let\PY@bf=\textbf\def\PY@tc##1{\textcolor[rgb]{0.73,0.40,0.53}{##1}}}
\expandafter\def\csname PY@tok@se\endcsname{\let\PY@bf=\textbf\def\PY@tc##1{\textcolor[rgb]{0.73,0.40,0.13}{##1}}}
\expandafter\def\csname PY@tok@sr\endcsname{\def\PY@tc##1{\textcolor[rgb]{0.73,0.40,0.53}{##1}}}
\expandafter\def\csname PY@tok@ss\endcsname{\def\PY@tc##1{\textcolor[rgb]{0.10,0.09,0.49}{##1}}}
\expandafter\def\csname PY@tok@sx\endcsname{\def\PY@tc##1{\textcolor[rgb]{0.00,0.50,0.00}{##1}}}
\expandafter\def\csname PY@tok@m\endcsname{\def\PY@tc##1{\textcolor[rgb]{0.40,0.40,0.40}{##1}}}
\expandafter\def\csname PY@tok@gh\endcsname{\let\PY@bf=\textbf\def\PY@tc##1{\textcolor[rgb]{0.00,0.00,0.50}{##1}}}
\expandafter\def\csname PY@tok@gu\endcsname{\let\PY@bf=\textbf\def\PY@tc##1{\textcolor[rgb]{0.50,0.00,0.50}{##1}}}
\expandafter\def\csname PY@tok@gd\endcsname{\def\PY@tc##1{\textcolor[rgb]{0.63,0.00,0.00}{##1}}}
\expandafter\def\csname PY@tok@gi\endcsname{\def\PY@tc##1{\textcolor[rgb]{0.00,0.63,0.00}{##1}}}
\expandafter\def\csname PY@tok@gr\endcsname{\def\PY@tc##1{\textcolor[rgb]{1.00,0.00,0.00}{##1}}}
\expandafter\def\csname PY@tok@ge\endcsname{\let\PY@it=\textit}
\expandafter\def\csname PY@tok@gs\endcsname{\let\PY@bf=\textbf}
\expandafter\def\csname PY@tok@gp\endcsname{\let\PY@bf=\textbf\def\PY@tc##1{\textcolor[rgb]{0.00,0.00,0.50}{##1}}}
\expandafter\def\csname PY@tok@go\endcsname{\def\PY@tc##1{\textcolor[rgb]{0.53,0.53,0.53}{##1}}}
\expandafter\def\csname PY@tok@gt\endcsname{\def\PY@tc##1{\textcolor[rgb]{0.00,0.27,0.87}{##1}}}
\expandafter\def\csname PY@tok@err\endcsname{\def\PY@bc##1{\setlength{\fboxsep}{0pt}\fcolorbox[rgb]{1.00,0.00,0.00}{1,1,1}{\strut ##1}}}
\expandafter\def\csname PY@tok@kc\endcsname{\let\PY@bf=\textbf\def\PY@tc##1{\textcolor[rgb]{0.00,0.50,0.00}{##1}}}
\expandafter\def\csname PY@tok@kd\endcsname{\let\PY@bf=\textbf\def\PY@tc##1{\textcolor[rgb]{0.00,0.50,0.00}{##1}}}
\expandafter\def\csname PY@tok@kn\endcsname{\let\PY@bf=\textbf\def\PY@tc##1{\textcolor[rgb]{0.00,0.50,0.00}{##1}}}
\expandafter\def\csname PY@tok@kr\endcsname{\let\PY@bf=\textbf\def\PY@tc##1{\textcolor[rgb]{0.00,0.50,0.00}{##1}}}
\expandafter\def\csname PY@tok@bp\endcsname{\def\PY@tc##1{\textcolor[rgb]{0.00,0.50,0.00}{##1}}}
\expandafter\def\csname PY@tok@fm\endcsname{\def\PY@tc##1{\textcolor[rgb]{0.00,0.00,1.00}{##1}}}
\expandafter\def\csname PY@tok@vc\endcsname{\def\PY@tc##1{\textcolor[rgb]{0.10,0.09,0.49}{##1}}}
\expandafter\def\csname PY@tok@vg\endcsname{\def\PY@tc##1{\textcolor[rgb]{0.10,0.09,0.49}{##1}}}
\expandafter\def\csname PY@tok@vi\endcsname{\def\PY@tc##1{\textcolor[rgb]{0.10,0.09,0.49}{##1}}}
\expandafter\def\csname PY@tok@vm\endcsname{\def\PY@tc##1{\textcolor[rgb]{0.10,0.09,0.49}{##1}}}
\expandafter\def\csname PY@tok@sa\endcsname{\def\PY@tc##1{\textcolor[rgb]{0.73,0.13,0.13}{##1}}}
\expandafter\def\csname PY@tok@sb\endcsname{\def\PY@tc##1{\textcolor[rgb]{0.73,0.13,0.13}{##1}}}
\expandafter\def\csname PY@tok@sc\endcsname{\def\PY@tc##1{\textcolor[rgb]{0.73,0.13,0.13}{##1}}}
\expandafter\def\csname PY@tok@dl\endcsname{\def\PY@tc##1{\textcolor[rgb]{0.73,0.13,0.13}{##1}}}
\expandafter\def\csname PY@tok@s2\endcsname{\def\PY@tc##1{\textcolor[rgb]{0.73,0.13,0.13}{##1}}}
\expandafter\def\csname PY@tok@sh\endcsname{\def\PY@tc##1{\textcolor[rgb]{0.73,0.13,0.13}{##1}}}
\expandafter\def\csname PY@tok@s1\endcsname{\def\PY@tc##1{\textcolor[rgb]{0.73,0.13,0.13}{##1}}}
\expandafter\def\csname PY@tok@mb\endcsname{\def\PY@tc##1{\textcolor[rgb]{0.40,0.40,0.40}{##1}}}
\expandafter\def\csname PY@tok@mf\endcsname{\def\PY@tc##1{\textcolor[rgb]{0.40,0.40,0.40}{##1}}}
\expandafter\def\csname PY@tok@mh\endcsname{\def\PY@tc##1{\textcolor[rgb]{0.40,0.40,0.40}{##1}}}
\expandafter\def\csname PY@tok@mi\endcsname{\def\PY@tc##1{\textcolor[rgb]{0.40,0.40,0.40}{##1}}}
\expandafter\def\csname PY@tok@il\endcsname{\def\PY@tc##1{\textcolor[rgb]{0.40,0.40,0.40}{##1}}}
\expandafter\def\csname PY@tok@mo\endcsname{\def\PY@tc##1{\textcolor[rgb]{0.40,0.40,0.40}{##1}}}
\expandafter\def\csname PY@tok@ch\endcsname{\let\PY@it=\textit\def\PY@tc##1{\textcolor[rgb]{0.25,0.50,0.50}{##1}}}
\expandafter\def\csname PY@tok@cm\endcsname{\let\PY@it=\textit\def\PY@tc##1{\textcolor[rgb]{0.25,0.50,0.50}{##1}}}
\expandafter\def\csname PY@tok@cpf\endcsname{\let\PY@it=\textit\def\PY@tc##1{\textcolor[rgb]{0.25,0.50,0.50}{##1}}}
\expandafter\def\csname PY@tok@c1\endcsname{\let\PY@it=\textit\def\PY@tc##1{\textcolor[rgb]{0.25,0.50,0.50}{##1}}}
\expandafter\def\csname PY@tok@cs\endcsname{\let\PY@it=\textit\def\PY@tc##1{\textcolor[rgb]{0.25,0.50,0.50}{##1}}}

\def\PYZbs{\char`\\}
\def\PYZus{\char`\_}
\def\PYZob{\char`\{}
\def\PYZcb{\char`\}}
\def\PYZca{\char`\^}
\def\PYZam{\char`\&}
\def\PYZlt{\char`\<}
\def\PYZgt{\char`\>}
\def\PYZsh{\char`\#}
\def\PYZpc{\char`\%}
\def\PYZdl{\char`\$}
\def\PYZhy{\char`\-}
\def\PYZsq{\char`\'}
\def\PYZdq{\char`\"}
\def\PYZti{\char`\~}
% for compatibility with earlier versions
\def\PYZat{@}
\def\PYZlb{[}
\def\PYZrb{]}
\makeatother


    % Exact colors from NB
    \definecolor{incolor}{rgb}{0.0, 0.0, 0.5}
    \definecolor{outcolor}{rgb}{0.545, 0.0, 0.0}



    
    % Prevent overflowing lines due to hard-to-break entities
    \sloppy 
    % Setup hyperref package
    \hypersetup{
      breaklinks=true,  % so long urls are correctly broken across lines
      colorlinks=true,
      urlcolor=urlcolor,
      linkcolor=linkcolor,
      citecolor=citecolor,
      }
    % Slightly bigger margins than the latex defaults
    
    \geometry{verbose,tmargin=1in,bmargin=1in,lmargin=1in,rmargin=1in}
    
    

    \begin{document}
    
    
    \maketitle
    
    

    
    \hypertarget{chapter-2-large-scale-data-analysis-with-spacy}{%
\section{Chapter 2: Large-scale data analysis with
spaCy}\label{chapter-2-large-scale-data-analysis-with-spacy}}

In this chapter, you'll use your new skills to extract specific
information from large volumes of text. You''ll learn how to make the
most of spaCy's data structures, and how to effectively combine
statistical and rule-based approaches for text analysis.

\hypertarget{strings-to-hashes}{%
\subsubsection{2.1 Strings to hashes}\label{strings-to-hashes}}

\textbf{Part 1}

\begin{itemize}
\tightlist
\item
  Look up the string ``cat'' in \texttt{nlp.vocab.strings} to get the
  hash.
\item
  Look up the hash to get back the string.
\end{itemize}

    \begin{Verbatim}[commandchars=\\\{\}]
{\color{incolor}In [{\color{incolor}1}]:} \PY{k+kn}{from} \PY{n+nn}{spacy}\PY{n+nn}{.}\PY{n+nn}{lang}\PY{n+nn}{.}\PY{n+nn}{en} \PY{k}{import} \PY{n}{English}
        
        \PY{n}{nlp} \PY{o}{=} \PY{n}{English}\PY{p}{(}\PY{p}{)}
        \PY{n}{doc} \PY{o}{=} \PY{n}{nlp}\PY{p}{(}\PY{l+s+s2}{\PYZdq{}}\PY{l+s+s2}{I have a cat}\PY{l+s+s2}{\PYZdq{}}\PY{p}{)}
        
        \PY{c+c1}{\PYZsh{} Look up the hash for the word \PYZdq{}cat\PYZdq{}}
        \PY{n}{cat\PYZus{}hash} \PY{o}{=} \PY{n}{nlp}\PY{o}{.}\PY{n}{vocab}\PY{o}{.}\PY{n}{strings}\PY{p}{[}\PY{l+s+s2}{\PYZdq{}}\PY{l+s+s2}{cat}\PY{l+s+s2}{\PYZdq{}}\PY{p}{]}
        \PY{n+nb}{print}\PY{p}{(}\PY{n}{cat\PYZus{}hash}\PY{p}{)}
        
        \PY{c+c1}{\PYZsh{} Look up the cat\PYZus{}hash to get the string}
        \PY{n}{cat\PYZus{}string} \PY{o}{=} \PY{n}{nlp}\PY{o}{.}\PY{n}{vocab}\PY{o}{.}\PY{n}{strings}\PY{p}{[}\PY{n}{cat\PYZus{}hash}\PY{p}{]}
        \PY{n+nb}{print}\PY{p}{(}\PY{n}{cat\PYZus{}string}\PY{p}{)}
\end{Verbatim}


    \begin{Verbatim}[commandchars=\\\{\}]
5439657043933447811
cat

    \end{Verbatim}

    \textbf{Part 2}

\begin{itemize}
\tightlist
\item
  Look up the string label ``PERSON'' in \texttt{nlp.vocab.strings} to
  get the hash.
\item
  Look up the hash to get back the string.
\end{itemize}

    \begin{Verbatim}[commandchars=\\\{\}]
{\color{incolor}In [{\color{incolor}2}]:} \PY{k+kn}{import} \PY{n+nn}{spacy}
        
        \PY{n}{nlp} \PY{o}{=} \PY{n}{spacy}\PY{o}{.}\PY{n}{load}\PY{p}{(}\PY{l+s+s2}{\PYZdq{}}\PY{l+s+s2}{en\PYZus{}core\PYZus{}web\PYZus{}sm}\PY{l+s+s2}{\PYZdq{}}\PY{p}{)}
        \PY{n}{doc} \PY{o}{=} \PY{n}{nlp}\PY{p}{(}\PY{l+s+s2}{\PYZdq{}}\PY{l+s+s2}{David Bowie is a PERSON}\PY{l+s+s2}{\PYZdq{}}\PY{p}{)}
        
        \PY{c+c1}{\PYZsh{} Look up the hash for the string label \PYZdq{}PERSON\PYZdq{}}
        \PY{n}{person\PYZus{}hash} \PY{o}{=} \PY{n}{nlp}\PY{o}{.}\PY{n}{vocab}\PY{o}{.}\PY{n}{strings}\PY{p}{[}\PY{l+s+s2}{\PYZdq{}}\PY{l+s+s2}{PERSON}\PY{l+s+s2}{\PYZdq{}}\PY{p}{]}
        \PY{n+nb}{print}\PY{p}{(}\PY{n}{person\PYZus{}hash}\PY{p}{)}
        
        \PY{c+c1}{\PYZsh{} Look up the person\PYZus{}hash to get the string}
        \PY{n}{person\PYZus{}string} \PY{o}{=} \PY{n}{nlp}\PY{o}{.}\PY{n}{vocab}\PY{o}{.}\PY{n}{strings}\PY{p}{[}\PY{n}{person\PYZus{}hash}\PY{p}{]}
        \PY{n+nb}{print}\PY{p}{(}\PY{n}{person\PYZus{}string}\PY{p}{)}
\end{Verbatim}


    \begin{Verbatim}[commandchars=\\\{\}]
380
PERSON

    \end{Verbatim}

    \hypertarget{exercise-2.1-why-does-this-code-throw-an-error}{%
\paragraph{Exercise 2.1: Why does this code throw an
error?}\label{exercise-2.1-why-does-this-code-throw-an-error}}

    \begin{Verbatim}[commandchars=\\\{\}]
{\color{incolor}In [{\color{incolor}3}]:} \PY{c+c1}{\PYZsh{} Exercise 2.1:}
        \PY{c+c1}{\PYZsh{} \PYZhy{}\PYZhy{}\PYZhy{}\PYZhy{}\PYZhy{}\PYZhy{}\PYZhy{}\PYZhy{}\PYZhy{}\PYZhy{}\PYZhy{}\PYZhy{}\PYZhy{}\PYZhy{}\PYZhy{}\PYZhy{}\PYZhy{}}
        \PY{c+c1}{\PYZsh{} Why does this code throw an error?}
        
        
        \PY{k+kn}{from} \PY{n+nn}{spacy}\PY{n+nn}{.}\PY{n+nn}{lang}\PY{n+nn}{.}\PY{n+nn}{en} \PY{k}{import} \PY{n}{English}
        \PY{k+kn}{from} \PY{n+nn}{spacy}\PY{n+nn}{.}\PY{n+nn}{lang}\PY{n+nn}{.}\PY{n+nn}{de} \PY{k}{import} \PY{n}{German}
        
        \PY{c+c1}{\PYZsh{} Create an English and German nlp object}
        \PY{n}{nlp} \PY{o}{=} \PY{n}{English}\PY{p}{(}\PY{p}{)}
        \PY{n}{nlp\PYZus{}de} \PY{o}{=} \PY{n}{German}\PY{p}{(}\PY{p}{)}
        
        \PY{c+c1}{\PYZsh{} Get the ID for the string \PYZsq{}Bowie\PYZsq{}}
        \PY{n}{bowie\PYZus{}id} \PY{o}{=} \PY{n}{nlp}\PY{o}{.}\PY{n}{vocab}\PY{o}{.}\PY{n}{strings}\PY{p}{[}\PY{l+s+s2}{\PYZdq{}}\PY{l+s+s2}{Bowie}\PY{l+s+s2}{\PYZdq{}}\PY{p}{]}
        \PY{n+nb}{print}\PY{p}{(}\PY{n}{bowie\PYZus{}id}\PY{p}{)}
        
        \PY{c+c1}{\PYZsh{} Look up the ID for \PYZdq{}Bowie\PYZdq{} in the vocab}
        \PY{n+nb}{print}\PY{p}{(}\PY{n}{nlp\PYZus{}de}\PY{o}{.}\PY{n}{vocab}\PY{o}{.}\PY{n}{strings}\PY{p}{[}\PY{n}{bowie\PYZus{}id}\PY{p}{]}\PY{p}{)}
\end{Verbatim}


    \begin{Verbatim}[commandchars=\\\{\}]
2644858412616767388

    \end{Verbatim}

    \begin{Verbatim}[commandchars=\\\{\}]

        ---------------------------------------------------------------------------

        KeyError                                  Traceback (most recent call last)

        <ipython-input-3-09ba464d7e29> in <module>
         16 
         17 \# Look up the ID for "Bowie" in the vocab
    ---> 18 print(nlp\_de.vocab.strings[bowie\_id])
    

        /opt/conda/lib/python3.6/site-packages/spacy/strings.pyx in spacy.strings.StringStore.\_\_getitem\_\_()


        KeyError: "[E018] Can't retrieve string for hash '2644858412616767388'. This usually refers to an issue with the `Vocab` or `StringStore`."

    \end{Verbatim}

    \emph{Your answer here\ldots{}}

    \begin{itemize}
\tightlist
\item
  Luulen, että errori tulee siitä jos \texttt{Bowie} merkkijonoa ei
  löydy saksalaisesta sanastosta, joten sen hash-numerosarjaakaan ei
  voida ratkaista.
\item
  Toisekseen huomasin, että \texttt{Bowie} sanaa haettiin aluksi
  käyttämällä \texttt{nlp} -objektia, mutta ID:n tarkisuksessa
  käytettiin saksan kieliseksi \texttt{nlp\_de} määritettyä objektia (?)
\end{itemize}

    \hypertarget{creating-a-doc}{%
\subsubsection{2.2 Creating a Doc}\label{creating-a-doc}}

Let's create some Doc objects from scratch!

\textbf{Part 1}

\begin{itemize}
\tightlist
\item
  Import the \texttt{Doc} from \texttt{spacy.tokens}.
\item
  Create a \texttt{Doc} from the \texttt{words} and \texttt{spaces}.
  Don't forget to pass in the vocab!
\end{itemize}

    \begin{Verbatim}[commandchars=\\\{\}]
{\color{incolor}In [{\color{incolor}4}]:} \PY{k+kn}{from} \PY{n+nn}{spacy}\PY{n+nn}{.}\PY{n+nn}{lang}\PY{n+nn}{.}\PY{n+nn}{en} \PY{k}{import} \PY{n}{English}
        
        \PY{n}{nlp} \PY{o}{=} \PY{n}{English}\PY{p}{(}\PY{p}{)}
        
        \PY{c+c1}{\PYZsh{} Import the Doc class}
        \PY{k+kn}{from} \PY{n+nn}{spacy}\PY{n+nn}{.}\PY{n+nn}{tokens} \PY{k}{import} \PY{n}{Doc}
        
        \PY{c+c1}{\PYZsh{} Desired text: \PYZdq{}spaCy is cool!\PYZdq{}}
        \PY{n}{words} \PY{o}{=} \PY{p}{[}\PY{l+s+s2}{\PYZdq{}}\PY{l+s+s2}{spaCy}\PY{l+s+s2}{\PYZdq{}}\PY{p}{,} \PY{l+s+s2}{\PYZdq{}}\PY{l+s+s2}{is}\PY{l+s+s2}{\PYZdq{}}\PY{p}{,} \PY{l+s+s2}{\PYZdq{}}\PY{l+s+s2}{cool}\PY{l+s+s2}{\PYZdq{}}\PY{p}{,} \PY{l+s+s2}{\PYZdq{}}\PY{l+s+s2}{!}\PY{l+s+s2}{\PYZdq{}}\PY{p}{]}
        \PY{n}{spaces} \PY{o}{=} \PY{p}{[}\PY{k+kc}{True}\PY{p}{,} \PY{k+kc}{True}\PY{p}{,} \PY{k+kc}{False}\PY{p}{,} \PY{k+kc}{False}\PY{p}{]}
        
        \PY{c+c1}{\PYZsh{} Create a Doc from the words and spaces}
        \PY{n}{doc} \PY{o}{=} \PY{n}{Doc}\PY{p}{(}\PY{n}{nlp}\PY{o}{.}\PY{n}{vocab}\PY{p}{,} \PY{n}{words}\PY{o}{=}\PY{n}{words}\PY{p}{,} \PY{n}{spaces}\PY{o}{=}\PY{n}{spaces}\PY{p}{)}
        \PY{n+nb}{print}\PY{p}{(}\PY{n}{doc}\PY{o}{.}\PY{n}{text}\PY{p}{)}
\end{Verbatim}


    \begin{Verbatim}[commandchars=\\\{\}]
spaCy is cool!

    \end{Verbatim}

    \textbf{Part 2}

\begin{itemize}
\tightlist
\item
  Import the \texttt{Doc} from \texttt{spacy.tokens}.
\item
  Create a \texttt{Doc} from the \texttt{words} and \texttt{spaces}.
  Don't forget to pass in the vocab!
\end{itemize}

    \begin{Verbatim}[commandchars=\\\{\}]
{\color{incolor}In [{\color{incolor}5}]:} \PY{k+kn}{from} \PY{n+nn}{spacy}\PY{n+nn}{.}\PY{n+nn}{lang}\PY{n+nn}{.}\PY{n+nn}{en} \PY{k}{import} \PY{n}{English}
        
        \PY{n}{nlp} \PY{o}{=} \PY{n}{English}\PY{p}{(}\PY{p}{)}
        
        \PY{c+c1}{\PYZsh{} Import the Doc class}
        \PY{k+kn}{from} \PY{n+nn}{spacy}\PY{n+nn}{.}\PY{n+nn}{tokens} \PY{k}{import} \PY{n}{Doc}
        
        \PY{c+c1}{\PYZsh{} Desired text: \PYZdq{}Go, get started!\PYZdq{}}
        \PY{n}{words} \PY{o}{=} \PY{p}{[}\PY{l+s+s2}{\PYZdq{}}\PY{l+s+s2}{Go}\PY{l+s+s2}{\PYZdq{}}\PY{p}{,} \PY{l+s+s2}{\PYZdq{}}\PY{l+s+s2}{,}\PY{l+s+s2}{\PYZdq{}}\PY{p}{,} \PY{l+s+s2}{\PYZdq{}}\PY{l+s+s2}{get}\PY{l+s+s2}{\PYZdq{}}\PY{p}{,} \PY{l+s+s2}{\PYZdq{}}\PY{l+s+s2}{started}\PY{l+s+s2}{\PYZdq{}}\PY{p}{,} \PY{l+s+s2}{\PYZdq{}}\PY{l+s+s2}{!}\PY{l+s+s2}{\PYZdq{}}\PY{p}{]}
        \PY{n}{spaces} \PY{o}{=} \PY{p}{[}\PY{k+kc}{False}\PY{p}{,} \PY{k+kc}{True}\PY{p}{,} \PY{k+kc}{True}\PY{p}{,} \PY{k+kc}{False}\PY{p}{,} \PY{k+kc}{False}\PY{p}{]}
        
        \PY{c+c1}{\PYZsh{} Create a Doc from the words and spaces}
        \PY{n}{doc} \PY{o}{=} \PY{n}{Doc}\PY{p}{(}\PY{n}{nlp}\PY{o}{.}\PY{n}{vocab}\PY{p}{,} \PY{n}{words}\PY{o}{=}\PY{n}{words}\PY{p}{,} \PY{n}{spaces}\PY{o}{=}\PY{n}{spaces}\PY{p}{)}
        \PY{n+nb}{print}\PY{p}{(}\PY{n}{doc}\PY{o}{.}\PY{n}{text}\PY{p}{)}
\end{Verbatim}


    \begin{Verbatim}[commandchars=\\\{\}]
Go, get started!

    \end{Verbatim}

    \hypertarget{exercise-2.2-complete-the-code}{%
\paragraph{Exercise 2.2: Complete the
code}\label{exercise-2.2-complete-the-code}}

\begin{itemize}
\tightlist
\item
  Import the \texttt{English}.
\item
  Complete the \texttt{words} and \texttt{spaces} to match the desired
  text and create a \texttt{doc}.
\end{itemize}

    \begin{Verbatim}[commandchars=\\\{\}]
{\color{incolor}In [{\color{incolor}6}]:} \PY{c+c1}{\PYZsh{}\PYZsh{}\PYZsh{}\PYZsh{} Exercise 2.2: Complete the code}
        
        \PY{c+c1}{\PYZsh{} Import English}
        \PY{n}{nlp} \PY{o}{=} \PY{n}{English}\PY{p}{(}\PY{p}{)} 
        
        \PY{c+c1}{\PYZsh{} Import the Doc class}
        \PY{k+kn}{from} \PY{n+nn}{spacy}\PY{n+nn}{.}\PY{n+nn}{tokens} \PY{k}{import} \PY{n}{Doc}
        
        \PY{c+c1}{\PYZsh{} Desired text: \PYZdq{}Oh, really?!\PYZdq{}}
        \PY{n}{words} \PY{o}{=} \PY{p}{[}\PY{l+s+s2}{\PYZdq{}}\PY{l+s+s2}{Oh}\PY{l+s+s2}{\PYZdq{}}\PY{p}{,} \PY{l+s+s2}{\PYZdq{}}\PY{l+s+s2}{,}\PY{l+s+s2}{\PYZdq{}}\PY{p}{,} \PY{l+s+s2}{\PYZdq{}}\PY{l+s+s2}{really}\PY{l+s+s2}{\PYZdq{}}\PY{p}{,} \PY{l+s+s2}{\PYZdq{}}\PY{l+s+s2}{?}\PY{l+s+s2}{\PYZdq{}}\PY{p}{,} \PY{l+s+s2}{\PYZdq{}}\PY{l+s+s2}{!}\PY{l+s+s2}{\PYZdq{}}\PY{p}{]}
        \PY{n}{spaces} \PY{o}{=} \PY{p}{[}\PY{k+kc}{False}\PY{p}{,} \PY{k+kc}{True}\PY{p}{,} \PY{k+kc}{False}\PY{p}{,} \PY{k+kc}{False}\PY{p}{,} \PY{k+kc}{False}\PY{p}{]}
        \PY{c+c1}{\PYZsh{} Create a Doc from the words and spaces}
        \PY{n}{doc} \PY{o}{=} \PY{n}{Doc}\PY{p}{(}\PY{n}{nlp}\PY{o}{.}\PY{n}{vocab}\PY{p}{,} \PY{n}{words}\PY{o}{=}\PY{n}{words}\PY{p}{,} \PY{n}{spaces}\PY{o}{=}\PY{n}{spaces}\PY{p}{)}
        \PY{n+nb}{print}\PY{p}{(}\PY{n}{doc}\PY{o}{.}\PY{n}{text}\PY{p}{)}
\end{Verbatim}


    \begin{Verbatim}[commandchars=\\\{\}]
Oh, really?!

    \end{Verbatim}

    \hypertarget{docs-spans-and-entities-from-scratch}{%
\subsubsection{2.3 Docs, spans and entities from
scratch}\label{docs-spans-and-entities-from-scratch}}

In this exercise, you'll create the \texttt{Doc} and \texttt{Span}
objects manually, and update the named entities -- just like spaCy does
behind the scenes. A shared \texttt{nlp} object has already been
created.

\begin{itemize}
\tightlist
\item
  Import the \texttt{Doc} and \texttt{Span} classes from
  \texttt{spacy.tokens}.
\item
  Use the \texttt{Doc} class directly to create a doc from the words and
  spaces.
\item
  Create a \texttt{Span} for ``David Bowie'' from the \texttt{doc} and
  assign it the label \texttt{"PERSON"}.
\item
  Overwrite the \texttt{doc.ents} with a list of one entity, the ``David
  Bowie'' \texttt{span}.
\end{itemize}

    \begin{Verbatim}[commandchars=\\\{\}]
{\color{incolor}In [{\color{incolor}7}]:} \PY{k+kn}{from} \PY{n+nn}{spacy}\PY{n+nn}{.}\PY{n+nn}{lang}\PY{n+nn}{.}\PY{n+nn}{en} \PY{k}{import} \PY{n}{English}
        
        \PY{n}{nlp} \PY{o}{=} \PY{n}{English}\PY{p}{(}\PY{p}{)}
        
        \PY{c+c1}{\PYZsh{} Import the Doc and Span classes}
        \PY{k+kn}{from} \PY{n+nn}{spacy}\PY{n+nn}{.}\PY{n+nn}{tokens} \PY{k}{import} \PY{n}{Doc}\PY{p}{,} \PY{n}{Span}
        
        \PY{n}{words} \PY{o}{=} \PY{p}{[}\PY{l+s+s2}{\PYZdq{}}\PY{l+s+s2}{I}\PY{l+s+s2}{\PYZdq{}}\PY{p}{,} \PY{l+s+s2}{\PYZdq{}}\PY{l+s+s2}{like}\PY{l+s+s2}{\PYZdq{}}\PY{p}{,} \PY{l+s+s2}{\PYZdq{}}\PY{l+s+s2}{David}\PY{l+s+s2}{\PYZdq{}}\PY{p}{,} \PY{l+s+s2}{\PYZdq{}}\PY{l+s+s2}{Bowie}\PY{l+s+s2}{\PYZdq{}}\PY{p}{]}
        \PY{n}{spaces} \PY{o}{=} \PY{p}{[}\PY{k+kc}{True}\PY{p}{,} \PY{k+kc}{True}\PY{p}{,} \PY{k+kc}{True}\PY{p}{,} \PY{k+kc}{False}\PY{p}{]}
        
        \PY{c+c1}{\PYZsh{} Create a doc from the words and spaces}
        \PY{n}{doc} \PY{o}{=} \PY{n}{Doc}\PY{p}{(}\PY{n}{nlp}\PY{o}{.}\PY{n}{vocab}\PY{p}{,} \PY{n}{words}\PY{o}{=}\PY{n}{words}\PY{p}{,} \PY{n}{spaces}\PY{o}{=}\PY{n}{spaces}\PY{p}{)}
        \PY{n+nb}{print}\PY{p}{(}\PY{n}{doc}\PY{o}{.}\PY{n}{text}\PY{p}{)}
        
        \PY{c+c1}{\PYZsh{} Create a span for \PYZdq{}David Bowie\PYZdq{} from the doc and assign it the label \PYZdq{}PERSON\PYZdq{}}
        \PY{c+c1}{\PYZsh{} Otetaan span eli palanen teksistä}
        \PY{n}{span} \PY{o}{=} \PY{n}{Span}\PY{p}{(}\PY{n}{doc}\PY{p}{,} \PY{l+m+mi}{2}\PY{p}{,} \PY{l+m+mi}{4}\PY{p}{,} \PY{n}{label}\PY{o}{=}\PY{l+s+s2}{\PYZdq{}}\PY{l+s+s2}{PERSON}\PY{l+s+s2}{\PYZdq{}}\PY{p}{)}
        \PY{n+nb}{print}\PY{p}{(}\PY{n}{span}\PY{o}{.}\PY{n}{text}\PY{p}{,} \PY{n}{span}\PY{o}{.}\PY{n}{label\PYZus{}}\PY{p}{)}
        
        \PY{c+c1}{\PYZsh{} Add the span to the doc\PYZsq{}s entities}
        \PY{n}{doc}\PY{o}{.}\PY{n}{ents} \PY{o}{=} \PY{p}{[}\PY{n}{span}\PY{p}{]}
        
        \PY{c+c1}{\PYZsh{} Print entities\PYZsq{} text and labels}
        \PY{n+nb}{print}\PY{p}{(}\PY{p}{[}\PY{p}{(}\PY{n}{ent}\PY{o}{.}\PY{n}{text}\PY{p}{,} \PY{n}{ent}\PY{o}{.}\PY{n}{label\PYZus{}}\PY{p}{)} \PY{k}{for} \PY{n}{ent} \PY{o+ow}{in} \PY{n}{doc}\PY{o}{.}\PY{n}{ents}\PY{p}{]}\PY{p}{)}
\end{Verbatim}


    \begin{Verbatim}[commandchars=\\\{\}]
I like David Bowie
David Bowie PERSON
[('David Bowie', 'PERSON')]

    \end{Verbatim}

    \hypertarget{data-structures-and-best-practices}{%
\subsubsection{2.4 Data structures and best
practices}\label{data-structures-and-best-practices}}

The code in this example is trying to analyze a text and collect all
proper nouns that are followed by a verb.

    \begin{Verbatim}[commandchars=\\\{\}]
{\color{incolor}In [{\color{incolor}8}]:} \PY{k+kn}{import} \PY{n+nn}{spacy}
        \PY{n}{nlp} \PY{o}{=} \PY{n}{spacy}\PY{o}{.}\PY{n}{load}\PY{p}{(}\PY{l+s+s2}{\PYZdq{}}\PY{l+s+s2}{en\PYZus{}core\PYZus{}web\PYZus{}sm}\PY{l+s+s2}{\PYZdq{}}\PY{p}{)}
        \PY{n}{doc} \PY{o}{=} \PY{n}{nlp}\PY{p}{(}\PY{l+s+s2}{\PYZdq{}}\PY{l+s+s2}{Berlin looks like a nice city}\PY{l+s+s2}{\PYZdq{}}\PY{p}{)}
        
        \PY{c+c1}{\PYZsh{} Get all tokens and part\PYZhy{}of\PYZhy{}speech tags}
        
        \PY{n}{token\PYZus{}texts} \PY{o}{=} \PY{p}{[}\PY{n}{token}\PY{o}{.}\PY{n}{text} \PY{k}{for} \PY{n}{token} \PY{o+ow}{in} \PY{n}{doc}\PY{p}{]}
        \PY{n}{pos\PYZus{}tags} \PY{o}{=} \PY{p}{[}\PY{n}{token}\PY{o}{.}\PY{n}{pos\PYZus{}} \PY{k}{for} \PY{n}{token} \PY{o+ow}{in} \PY{n}{doc}\PY{p}{]}
        
        \PY{k}{for} \PY{n}{index}\PY{p}{,} \PY{n}{pos} \PY{o+ow}{in} \PY{n+nb}{enumerate}\PY{p}{(}\PY{n}{pos\PYZus{}tags}\PY{p}{)}\PY{p}{:}
            \PY{c+c1}{\PYZsh{} Check if the current token is a proper noun}
            \PY{k}{if} \PY{n}{pos} \PY{o}{==} \PY{l+s+s2}{\PYZdq{}}\PY{l+s+s2}{PROPN}\PY{l+s+s2}{\PYZdq{}}\PY{p}{:}
                \PY{c+c1}{\PYZsh{} Check if the next token is a verb}
                \PY{k}{if} \PY{n}{pos\PYZus{}tags}\PY{p}{[}\PY{n}{index} \PY{o}{+} \PY{l+m+mi}{1}\PY{p}{]} \PY{o}{==} \PY{l+s+s2}{\PYZdq{}}\PY{l+s+s2}{VERB}\PY{l+s+s2}{\PYZdq{}}\PY{p}{:}
                    \PY{n}{result} \PY{o}{=} \PY{n}{token\PYZus{}texts}\PY{p}{[}\PY{n}{index}\PY{p}{]}
                    \PY{n+nb}{print}\PY{p}{(}\PY{l+s+s2}{\PYZdq{}}\PY{l+s+s2}{Found proper noun before a verb:}\PY{l+s+s2}{\PYZdq{}}\PY{p}{,} \PY{n}{result}\PY{p}{)}
\end{Verbatim}


    \begin{Verbatim}[commandchars=\\\{\}]
Found proper noun before a verb: Berlin

    \end{Verbatim}

    \hypertarget{exercise-2.3-why-is-the-code-bad}{%
\paragraph{Exercise 2.3: Why is the code
bad?}\label{exercise-2.3-why-is-the-code-bad}}

    \emph{Your answer here\ldots{}}

    \begin{itemize}
\item
  \texttt{token\_texts} ja \texttt{pos\_tags} ovat turhia välivaiheita.
\item
  Samoin on \texttt{enumerate} -looppi.
\item
  Tosin koodi toimii, niin onhan se siinä mielessä kuitenkin hyvää
  koodia :P
\item
  Koodin voisi kirjoittaa samaan tyyliin, kuin \textbf{Chapter1}:n
  kohdassa \textbf{1.3}
\end{itemize}

    \hypertarget{exercise-2.4-rewrite-the-code}{%
\paragraph{Exercise 2.4: Rewrite the
code?}\label{exercise-2.4-rewrite-the-code}}

\begin{itemize}
\tightlist
\item
  Rewrite the code to use the native token attributes instead of lists
  of \texttt{token\_texts} and \texttt{pos\_tags}.
\item
  Loop over each \texttt{token} in the \texttt{doc} and check the
  \texttt{token.pos\_} attribute.
\item
  Use \texttt{doc{[}token.i\ +\ 1{]}} to check for the next token and
  \texttt{its\ .pos\_} attribute.
\item
  If a proper noun before a verb is found, print its
  \texttt{token.text}.
\end{itemize}

    \begin{Verbatim}[commandchars=\\\{\}]
{\color{incolor}In [{\color{incolor}9}]:} \PY{k+kn}{import} \PY{n+nn}{spacy}
        
        \PY{n}{nlp} \PY{o}{=} \PY{n}{spacy}\PY{o}{.}\PY{n}{load}\PY{p}{(}\PY{l+s+s2}{\PYZdq{}}\PY{l+s+s2}{en\PYZus{}core\PYZus{}web\PYZus{}sm}\PY{l+s+s2}{\PYZdq{}}\PY{p}{)}
        \PY{n}{doc} \PY{o}{=} \PY{n}{nlp}\PY{p}{(}\PY{l+s+s2}{\PYZdq{}}\PY{l+s+s2}{Berlin looks like a nice city}\PY{l+s+s2}{\PYZdq{}}\PY{p}{)}
        
        \PY{c+c1}{\PYZsh{} Your code here}
        \PY{c+c1}{\PYZsh{} \PYZhy{}\PYZhy{}\PYZhy{}\PYZhy{}\PYZhy{}\PYZhy{}\PYZhy{}\PYZhy{}\PYZhy{}\PYZhy{}\PYZhy{}\PYZhy{}\PYZhy{}\PYZhy{}}
        
        
        \PY{c+c1}{\PYZsh{} Iteroidaan tekstin tokenit}
        \PY{k}{for} \PY{n}{token} \PY{o+ow}{in} \PY{n}{doc}\PY{p}{:}
            \PY{c+c1}{\PYZsh{} .pos\PYZus{} :lla saatiin tokenin sanaluokka}
            \PY{c+c1}{\PYZsh{} Tarkistetaan onko sanaluokka PROPN eli erisnimi}
            \PY{k}{if} \PY{n}{token}\PY{o}{.}\PY{n}{pos\PYZus{}} \PY{o}{==} \PY{l+s+s2}{\PYZdq{}}\PY{l+s+s2}{PROPN}\PY{l+s+s2}{\PYZdq{}}\PY{p}{:}
                \PY{c+c1}{\PYZsh{} Tarkastetaan onko erisnimen jälkeen verbi}
                \PY{c+c1}{\PYZsh{} token.i+1 :llä saatiin seuraava tokeni}
                \PY{k}{if} \PY{n}{doc}\PY{p}{[}\PY{n}{token}\PY{o}{.}\PY{n}{i}\PY{o}{+}\PY{l+m+mi}{1}\PY{p}{]}\PY{o}{.}\PY{n}{pos\PYZus{}} \PY{o}{==} \PY{l+s+s2}{\PYZdq{}}\PY{l+s+s2}{VERB}\PY{l+s+s2}{\PYZdq{}}\PY{p}{:}
                    \PY{c+c1}{\PYZsh{} Lopuksi printataan alkuperäinen tokeni ...}
                    \PY{c+c1}{\PYZsh{} ... jos sellainen teksistä löytyi}
                    \PY{n+nb}{print}\PY{p}{(}\PY{l+s+s2}{\PYZdq{}}\PY{l+s+s2}{Found proper noun before a verb:}\PY{l+s+s2}{\PYZdq{}}\PY{p}{,} \PY{n}{token}\PY{o}{.}\PY{n}{text}\PY{p}{)}
\end{Verbatim}


    \begin{Verbatim}[commandchars=\\\{\}]
Found proper noun before a verb: Berlin

    \end{Verbatim}

    \hypertarget{inspecting-word-vectors}{%
\subsubsection{2.5 Inspecting word
vectors}\label{inspecting-word-vectors}}

In this exercise, you'll use a larger English model, which includes
around 20.000 word vectors. The model is already pre-installed.

\begin{itemize}
\tightlist
\item
  Load the medium \texttt{"en\_core\_web\_md"} model with word vectors.
\item
  Print the vector for \texttt{"bananas"} using the
  \texttt{token.vector} attribute.
\end{itemize}

    \begin{Verbatim}[commandchars=\\\{\}]
{\color{incolor}In [{\color{incolor}10}]:} \PY{k+kn}{import} \PY{n+nn}{spacy}
         
         \PY{c+c1}{\PYZsh{} Load the en\PYZus{}core\PYZus{}web\PYZus{}md model}
         \PY{n}{nlp} \PY{o}{=} \PY{n}{spacy}\PY{o}{.}\PY{n}{load}\PY{p}{(}\PY{l+s+s2}{\PYZdq{}}\PY{l+s+s2}{en\PYZus{}core\PYZus{}web\PYZus{}md}\PY{l+s+s2}{\PYZdq{}}\PY{p}{)}
         
         \PY{c+c1}{\PYZsh{} Process a text}
         \PY{n}{doc} \PY{o}{=} \PY{n}{nlp}\PY{p}{(}\PY{l+s+s2}{\PYZdq{}}\PY{l+s+s2}{Two bananas in pyjamas}\PY{l+s+s2}{\PYZdq{}}\PY{p}{)}
         
         \PY{c+c1}{\PYZsh{} Get the vector for the token \PYZdq{}bananas\PYZdq{}}
         \PY{n}{bananas\PYZus{}vector} \PY{o}{=} \PY{n}{doc}\PY{p}{[}\PY{l+m+mi}{1}\PY{p}{]}\PY{o}{.}\PY{n}{vector}
         \PY{n+nb}{print}\PY{p}{(}\PY{n}{bananas\PYZus{}vector}\PY{p}{)}
\end{Verbatim}


    \begin{Verbatim}[commandchars=\\\{\}]
[-2.2009e-01 -3.0322e-02 -7.9859e-02 -4.6279e-01 -3.8600e-01  3.6962e-01
 -7.7178e-01 -1.1529e-01  3.3601e-02  5.6573e-01 -2.4001e-01  4.1833e-01
  1.5049e-01  3.5621e-01 -2.1508e-01 -4.2743e-01  8.1400e-02  3.3916e-01
  2.1637e-01  1.4792e-01  4.5811e-01  2.0966e-01 -3.5706e-01  2.3800e-01
  2.7971e-02 -8.4538e-01  4.1917e-01 -3.9181e-01  4.0434e-04 -1.0662e+00
  1.4591e-01  1.4643e-03  5.1277e-01  2.6072e-01  8.3785e-02  3.0340e-01
  1.8579e-01  5.9999e-02 -4.0270e-01  5.0888e-01 -1.1358e-01 -2.8854e-01
 -2.7068e-01  1.1017e-02 -2.2217e-01  6.9076e-01  3.6459e-02  3.0394e-01
  5.6989e-02  2.2733e-01 -9.9473e-02  1.5165e-01  1.3540e-01 -2.4965e-01
  9.8078e-01 -8.0492e-01  1.9326e-01  3.1128e-01  5.5390e-02 -4.2423e-01
 -1.4082e-02  1.2708e-01  1.8868e-01  5.9777e-02 -2.2215e-01 -8.3950e-01
  9.1987e-02  1.0180e-01 -3.1299e-01  5.5083e-01 -3.0717e-01  4.4201e-01
  1.2666e-01  3.7643e-01  3.2333e-01  9.5673e-02  2.5083e-01 -6.4049e-02
  4.2143e-01 -1.9375e-01  3.8026e-01  7.0883e-03 -2.0371e-01  1.5402e-01
 -3.7877e-03 -2.9396e-01  9.6518e-01  2.0068e-01 -5.6572e-01 -2.2581e-01
  3.2251e-01 -3.4634e-01  2.7064e-01 -2.0687e-01 -4.7229e-01  3.1704e-01
 -3.4665e-01 -2.5188e-01 -1.1201e-01 -3.3937e-01  3.1518e-01 -3.2221e-01
 -2.4530e-01 -7.1571e-02 -4.3971e-01 -1.2070e+00  3.3365e-01 -5.8208e-02
  8.0899e-01  4.2335e-01  3.8678e-01 -6.0797e-01 -7.3760e-01 -2.0547e-01
 -1.7499e-01 -3.7842e-03  2.1930e-01 -5.2486e-02  3.4869e-01  4.3852e-01
 -3.4471e-01  2.8910e-01  7.2554e-02 -4.8625e-01 -3.8390e-01 -4.4760e-01
  4.3278e-01 -2.7128e-03 -9.0067e-01 -3.0819e-02 -3.8630e-01 -8.0798e-02
 -1.6243e-01  2.8830e-01 -2.6349e-01  1.7628e-01  3.5958e-01  5.7672e-01
 -5.4624e-01  3.8555e-02 -2.0182e+00  3.2916e-01  3.4672e-01  1.5398e-01
 -4.3446e-01 -4.1428e-02 -6.9588e-02  5.1513e-01 -1.3489e-01 -5.7239e-02
  4.9241e-01  1.8643e-01  3.8596e-01 -3.7329e-02 -5.4216e-01 -1.8152e-01
  4.3110e-01 -4.6967e-01  6.6801e-02  5.0323e-01 -2.4059e-01  3.6742e-01
  2.9300e-01 -8.7883e-02 -4.7940e-01 -4.3431e-02 -2.6137e-01 -6.2658e-01
  1.1446e-01  2.7682e-01  3.4800e-01  5.0018e-01  1.4269e-01 -3.3545e-01
 -3.9712e-01 -3.3121e-01 -3.4434e-01 -4.1627e-01 -3.5707e-03 -6.2350e-01
  3.7794e-01 -1.6765e-01 -4.1954e-01 -3.3134e-01  3.1232e-01 -3.9494e-01
 -4.6921e-03 -4.8884e-01 -2.2059e-02 -2.6174e-01  1.7937e-01  3.6628e-01
  5.8971e-02 -3.5991e-01 -4.4393e-01 -1.1890e-01  3.3487e-01  3.6505e-02
 -3.2788e-01  3.3425e-01 -5.6361e-01 -1.1190e-01  5.3770e-01  2.0311e-01
  1.5110e-01  1.0623e-02  3.3401e-01  4.6084e-01  5.6293e-01 -7.5432e-02
  5.4813e-01  1.9395e-01 -2.6265e-01 -3.1699e-01 -8.1778e-01  5.8169e-02
 -5.7866e-02 -1.1781e-01 -5.8742e-02 -1.4092e-01 -9.9394e-01 -9.4532e-02
  2.3503e-01 -4.9027e-01  8.5832e-01  1.1540e-01 -1.5049e-01  1.9065e-01
 -2.6705e-01  2.5326e-01 -6.7579e-01 -1.0633e-02 -5.5158e-02 -3.1004e-01
 -5.8036e-02 -1.7200e-01  1.3298e-01 -3.2899e-01 -7.5481e-02  2.9425e-02
 -3.2949e-01 -1.8691e-01 -9.5323e-01 -3.5468e-01 -3.3162e-01  5.6441e-02
  2.1790e-02  1.7182e-01 -4.4267e-01  6.9765e-01 -2.6876e-01  1.1659e-01
 -1.6584e-01  3.8296e-01  2.9109e-01  3.6318e-01  3.6961e-01  1.6305e-01
  1.8152e-01  2.2453e-01  3.9866e-02 -3.7607e-02 -3.6089e-01  7.0818e-02
 -2.1509e-01  3.6551e-01 -5.1603e-01 -5.8102e-03 -4.8320e-01 -2.5068e-01
 -5.2062e-02 -2.0828e-01  2.9060e-01  2.2084e-02 -6.8123e-01  4.2063e-01
  9.5973e-02  8.1720e-01 -1.5241e-01  6.2994e-01  2.6449e-01 -1.3516e-01
  3.2450e-01  3.0503e-01  1.2357e-01  1.5107e-01  2.8327e-01 -3.3838e-01
  4.6106e-02 -1.2361e-01  1.4516e-01 -2.7947e-02  2.6231e-02 -5.9591e-01
 -4.4183e-01  7.8440e-01 -3.4375e-02 -1.3928e+00  3.5248e-01  6.5220e-01]

    \end{Verbatim}

    \hypertarget{comparing-similarities}{%
\subsubsection{2.6 Comparing
similarities}\label{comparing-similarities}}

In this exercise, you'll be using spaCy's similarity methods to compare
Doc, Token and Span objects and get similarity scores.

\textbf{Part 1}

\begin{itemize}
\tightlist
\item
  Use the \texttt{doc.similarity} method to compare \texttt{doc1} to
  \texttt{doc2} and print the result.
\end{itemize}

    \begin{Verbatim}[commandchars=\\\{\}]
{\color{incolor}In [{\color{incolor}11}]:} \PY{k+kn}{import} \PY{n+nn}{spacy}
         
         \PY{n}{nlp} \PY{o}{=} \PY{n}{spacy}\PY{o}{.}\PY{n}{load}\PY{p}{(}\PY{l+s+s2}{\PYZdq{}}\PY{l+s+s2}{en\PYZus{}core\PYZus{}web\PYZus{}md}\PY{l+s+s2}{\PYZdq{}}\PY{p}{)}
         
         \PY{n}{doc1} \PY{o}{=} \PY{n}{nlp}\PY{p}{(}\PY{l+s+s2}{\PYZdq{}}\PY{l+s+s2}{It}\PY{l+s+s2}{\PYZsq{}}\PY{l+s+s2}{s a warm summer day}\PY{l+s+s2}{\PYZdq{}}\PY{p}{)}
         \PY{n}{doc2} \PY{o}{=} \PY{n}{nlp}\PY{p}{(}\PY{l+s+s2}{\PYZdq{}}\PY{l+s+s2}{It}\PY{l+s+s2}{\PYZsq{}}\PY{l+s+s2}{s sunny outside}\PY{l+s+s2}{\PYZdq{}}\PY{p}{)}
         
         \PY{c+c1}{\PYZsh{} Get the similarity of doc1 and doc2}
         
         \PY{c+c1}{\PYZsh{} Tarkistetaan, kuinka lähellä nämä kaksi lausetta ovat toisiaan}
         \PY{n}{similarity} \PY{o}{=} \PY{n}{doc1}\PY{o}{.}\PY{n}{similarity}\PY{p}{(}\PY{n}{doc2}\PY{p}{)}
         \PY{n+nb}{print}\PY{p}{(}\PY{n}{similarity}\PY{p}{)}
\end{Verbatim}


    \begin{Verbatim}[commandchars=\\\{\}]
0.8789265574516525

    \end{Verbatim}

    \textbf{Part 2}

\begin{itemize}
\tightlist
\item
  Use the \texttt{token.similarity} method to compare \texttt{TV} to
  \texttt{books} and print the result.
\end{itemize}

    \begin{Verbatim}[commandchars=\\\{\}]
{\color{incolor}In [{\color{incolor}12}]:} \PY{k+kn}{import} \PY{n+nn}{spacy}
         
         \PY{n}{nlp} \PY{o}{=} \PY{n}{spacy}\PY{o}{.}\PY{n}{load}\PY{p}{(}\PY{l+s+s2}{\PYZdq{}}\PY{l+s+s2}{en\PYZus{}core\PYZus{}web\PYZus{}md}\PY{l+s+s2}{\PYZdq{}}\PY{p}{)}
         
         \PY{n}{doc} \PY{o}{=} \PY{n}{nlp}\PY{p}{(}\PY{l+s+s2}{\PYZdq{}}\PY{l+s+s2}{TV and books}\PY{l+s+s2}{\PYZdq{}}\PY{p}{)}
         \PY{n}{token1}\PY{p}{,} \PY{n}{token2} \PY{o}{=} \PY{n}{doc}\PY{p}{[}\PY{l+m+mi}{0}\PY{p}{]}\PY{p}{,} \PY{n}{doc}\PY{p}{[}\PY{l+m+mi}{2}\PY{p}{]}
         
         \PY{c+c1}{\PYZsh{} Get the similarity of the tokens \PYZdq{}TV\PYZdq{} and \PYZdq{}books\PYZdq{}}
         \PY{n}{similarity} \PY{o}{=} \PY{n}{token1}\PY{o}{.}\PY{n}{similarity}\PY{p}{(}\PY{n}{token2}\PY{p}{)}
         \PY{n+nb}{print}\PY{p}{(}\PY{n}{similarity}\PY{p}{)}
\end{Verbatim}


    \begin{Verbatim}[commandchars=\\\{\}]
0.22325331

    \end{Verbatim}

    \hypertarget{exercise-2.5-similarity-between-great-restaurant-and-really-nice-bar}{%
\paragraph{Exercise 2.5: Similarity between ``great restaurant'' and
``really nice
bar''?}\label{exercise-2.5-similarity-between-great-restaurant-and-really-nice-bar}}

\begin{itemize}
\tightlist
\item
  Create spans for ``great restaurant''/``really nice bar''.
\item
  Use \texttt{span.similarity} to compare them and print the result.
\end{itemize}

    \begin{Verbatim}[commandchars=\\\{\}]
{\color{incolor}In [{\color{incolor}13}]:} \PY{k+kn}{import} \PY{n+nn}{spacy}
         
         \PY{n}{nlp} \PY{o}{=} \PY{n}{spacy}\PY{o}{.}\PY{n}{load}\PY{p}{(}\PY{l+s+s2}{\PYZdq{}}\PY{l+s+s2}{en\PYZus{}core\PYZus{}web\PYZus{}md}\PY{l+s+s2}{\PYZdq{}}\PY{p}{)}
         
         \PY{n}{doc} \PY{o}{=} \PY{n}{nlp}\PY{p}{(}\PY{l+s+s2}{\PYZdq{}}\PY{l+s+s2}{This was a great restaurant. Afterwards, we went to a really nice bar.}\PY{l+s+s2}{\PYZdq{}}\PY{p}{)}
         
         \PY{c+c1}{\PYZsh{} Exercise 2.5: Complete the code}
         \PY{c+c1}{\PYZsh{} \PYZhy{}\PYZhy{}\PYZhy{}\PYZhy{}\PYZhy{}\PYZhy{}\PYZhy{}\PYZhy{}\PYZhy{}\PYZhy{}\PYZhy{}\PYZhy{}\PYZhy{}\PYZhy{}\PYZhy{}\PYZhy{}}
         \PY{c+c1}{\PYZsh{} Create spans for \PYZdq{}great restaurant\PYZdq{} and \PYZdq{}really nice bar\PYZdq{}}
         \PY{n}{span1} \PY{o}{=} \PY{n}{doc}\PY{p}{[}\PY{l+m+mi}{3}\PY{p}{:}\PY{l+m+mi}{5}\PY{p}{]}
         \PY{n}{span2} \PY{o}{=} \PY{n}{doc}\PY{p}{[}\PY{l+m+mi}{12}\PY{p}{:}\PY{o}{\PYZhy{}}\PY{l+m+mi}{1}\PY{p}{]}
         
         \PY{n+nb}{print}\PY{p}{(}\PY{l+s+s2}{\PYZdq{}}\PY{l+s+s2}{Span1:}\PY{l+s+s2}{\PYZdq{}}\PY{p}{,}\PY{n}{span1}\PY{p}{)}
         \PY{n+nb}{print}\PY{p}{(}\PY{l+s+s2}{\PYZdq{}}\PY{l+s+s2}{Span2:}\PY{l+s+s2}{\PYZdq{}}\PY{p}{,}\PY{n}{span2}\PY{p}{)}
         
         \PY{c+c1}{\PYZsh{} Get the similarity of the spans}
         \PY{n}{similarity} \PY{o}{=} \PY{n}{span1}\PY{o}{.}\PY{n}{similarity}\PY{p}{(}\PY{n}{span2}\PY{p}{)}
         \PY{n+nb}{print}\PY{p}{(}\PY{l+s+s2}{\PYZdq{}}\PY{l+s+s2}{Similarity:}\PY{l+s+s2}{\PYZdq{}}\PY{p}{,}\PY{n}{similarity}\PY{p}{)}
\end{Verbatim}


    \begin{Verbatim}[commandchars=\\\{\}]
Span1: great restaurant
Span2: really nice bar
Similarity: 0.75173926

    \end{Verbatim}

    \hypertarget{debugging-patterns-1}{%
\subsubsection{2.6 Debugging patterns (1)}\label{debugging-patterns-1}}

\hypertarget{exercise-2.6-why-does-this-pattern-not-match-the-tokens-silicon-valley-in-the-doc}{%
\paragraph{Exercise 2.6: Why does this pattern not match the tokens
``Silicon Valley'' in the
doc?}\label{exercise-2.6-why-does-this-pattern-not-match-the-tokens-silicon-valley-in-the-doc}}

\texttt{pattern\ =\ {[}\{"LOWER":\ "silicon"\},\ \{"TEXT":\ "\ "\},\ \{"LOWER":\ "valley"\}{]}}

\texttt{doc\ =\ nlp("Can\ Silicon\ Valley\ workers\ rein\ in\ big\ tech\ from\ within?")}

    \emph{Your answer here\ldots{}}

    \begin{itemize}
\tightlist
\item
  Tokenizer ei luo tokenia välilyönnille
\end{itemize}

    \hypertarget{debugging-patterns-2}{%
\subsubsection{2.7 Debugging patterns (2)}\label{debugging-patterns-2}}

\hypertarget{exercise-2.7-both-patterns-in-this-exercise-contain-mistakes-and-wont-match-as-expected.-can-you-fix-them}{%
\paragraph{Exercise 2.7: Both patterns in this exercise contain mistakes
and won't match as expected. Can you fix
them?}\label{exercise-2.7-both-patterns-in-this-exercise-contain-mistakes-and-wont-match-as-expected.-can-you-fix-them}}

If you get stuck, try printing the tokens in the \texttt{doc} to see how
the text will be split and adjust the pattern so that each dictionary
represents one token.

\begin{itemize}
\tightlist
\item
  Edit pattern1 so that it correctly matches all case-insensitive
  mentions of ``Amazon'' plus a title-cased proper noun.
\item
  Edit pattern2 so that it correctly matches all case-insensitive
  mentions of ``ad-free'', plus the following noun.
\end{itemize}

    \begin{Verbatim}[commandchars=\\\{\}]
{\color{incolor}In [{\color{incolor}14}]:} \PY{k+kn}{import} \PY{n+nn}{spacy}
         \PY{k+kn}{from} \PY{n+nn}{spacy}\PY{n+nn}{.}\PY{n+nn}{matcher} \PY{k}{import} \PY{n}{Matcher}
         
         \PY{n}{nlp} \PY{o}{=} \PY{n}{spacy}\PY{o}{.}\PY{n}{load}\PY{p}{(}\PY{l+s+s2}{\PYZdq{}}\PY{l+s+s2}{en\PYZus{}core\PYZus{}web\PYZus{}sm}\PY{l+s+s2}{\PYZdq{}}\PY{p}{)}
         \PY{n}{doc} \PY{o}{=} \PY{n}{nlp}\PY{p}{(}
             \PY{l+s+s2}{\PYZdq{}}\PY{l+s+s2}{Twitch Prime, the perks program for Amazon Prime members offering free }\PY{l+s+s2}{\PYZdq{}}
             \PY{l+s+s2}{\PYZdq{}}\PY{l+s+s2}{loot, games and other benefits, is ditching one of its best features: }\PY{l+s+s2}{\PYZdq{}}
             \PY{l+s+s2}{\PYZdq{}}\PY{l+s+s2}{ad\PYZhy{}free viewing. According to an email sent out to Amazon Prime members }\PY{l+s+s2}{\PYZdq{}}
             \PY{l+s+s2}{\PYZdq{}}\PY{l+s+s2}{today, ad\PYZhy{}free viewing will no longer be included as a part of Twitch }\PY{l+s+s2}{\PYZdq{}}
             \PY{l+s+s2}{\PYZdq{}}\PY{l+s+s2}{Prime for new members, beginning on September 14. However, members with }\PY{l+s+s2}{\PYZdq{}}
             \PY{l+s+s2}{\PYZdq{}}\PY{l+s+s2}{existing annual subscriptions will be able to continue to enjoy ad\PYZhy{}free }\PY{l+s+s2}{\PYZdq{}}
             \PY{l+s+s2}{\PYZdq{}}\PY{l+s+s2}{viewing until their subscription comes up for renewal. Those with }\PY{l+s+s2}{\PYZdq{}}
             \PY{l+s+s2}{\PYZdq{}}\PY{l+s+s2}{monthly subscriptions will have access to ad\PYZhy{}free viewing until October 15.}\PY{l+s+s2}{\PYZdq{}}
         \PY{p}{)}
         
         \PY{c+c1}{\PYZsh{} Create the match patterns}
         
         \PY{c+c1}{\PYZsh{} Pattern1:ssä pitää muutta \PYZdq{}LOWER\PYZdq{} \PYZhy{}tokeni pienellä kirjoitettuun muotoon}
         \PY{c+c1}{\PYZsh{}pattern1 = [[\PYZob{}\PYZdq{}LOWER\PYZdq{}: \PYZdq{}Amazon\PYZdq{}\PYZcb{}, \PYZob{}\PYZdq{}IS\PYZus{}TITLE\PYZdq{}: True, \PYZdq{}POS\PYZdq{}: \PYZdq{}PROPN\PYZdq{}\PYZcb{}]]}
         \PY{n}{pattern1} \PY{o}{=} \PY{p}{[}\PY{p}{[}\PY{p}{\PYZob{}}\PY{l+s+s2}{\PYZdq{}}\PY{l+s+s2}{LOWER}\PY{l+s+s2}{\PYZdq{}}\PY{p}{:} \PY{l+s+s2}{\PYZdq{}}\PY{l+s+s2}{amazon}\PY{l+s+s2}{\PYZdq{}}\PY{p}{\PYZcb{}}\PY{p}{,} \PY{p}{\PYZob{}}\PY{l+s+s2}{\PYZdq{}}\PY{l+s+s2}{IS\PYZus{}TITLE}\PY{l+s+s2}{\PYZdq{}}\PY{p}{:} \PY{k+kc}{True}\PY{p}{,} \PY{l+s+s2}{\PYZdq{}}\PY{l+s+s2}{POS}\PY{l+s+s2}{\PYZdq{}}\PY{p}{:} \PY{l+s+s2}{\PYZdq{}}\PY{l+s+s2}{PROPN}\PY{l+s+s2}{\PYZdq{}}\PY{p}{\PYZcb{}}\PY{p}{]}\PY{p}{]}
         \PY{c+c1}{\PYZsh{} Pattern2:ssa pitää purkaa \PYZdq{}ad\PYZhy{}free\PYZdq{} useampaan osaan ...}
         \PY{c+c1}{\PYZsh{} ... sillä se on kaksiosainen ja sisältää välimerkin}
         \PY{c+c1}{\PYZsh{}pattern2 = [[\PYZob{}\PYZdq{}LOWER\PYZdq{}: \PYZdq{}ad\PYZhy{}free\PYZdq{}\PYZcb{}, \PYZob{}\PYZdq{}POS\PYZdq{}: \PYZdq{}NOUN\PYZdq{}\PYZcb{}]]}
         \PY{n}{pattern2} \PY{o}{=} \PY{p}{[}\PY{p}{[}\PY{p}{\PYZob{}}\PY{l+s+s2}{\PYZdq{}}\PY{l+s+s2}{LOWER}\PY{l+s+s2}{\PYZdq{}}\PY{p}{:} \PY{l+s+s2}{\PYZdq{}}\PY{l+s+s2}{ad}\PY{l+s+s2}{\PYZdq{}}\PY{p}{\PYZcb{}}\PY{p}{,} \PY{p}{\PYZob{}}\PY{l+s+s2}{\PYZdq{}}\PY{l+s+s2}{IS\PYZus{}PUNCT}\PY{l+s+s2}{\PYZdq{}}\PY{p}{:} \PY{k+kc}{True}\PY{p}{\PYZcb{}}\PY{p}{,} \PY{p}{\PYZob{}}\PY{l+s+s2}{\PYZdq{}}\PY{l+s+s2}{LOWER}\PY{l+s+s2}{\PYZdq{}}\PY{p}{:} \PY{l+s+s2}{\PYZdq{}}\PY{l+s+s2}{free}\PY{l+s+s2}{\PYZdq{}}\PY{p}{\PYZcb{}}\PY{p}{,} \PY{p}{\PYZob{}}\PY{l+s+s2}{\PYZdq{}}\PY{l+s+s2}{POS}\PY{l+s+s2}{\PYZdq{}}\PY{p}{:} \PY{l+s+s2}{\PYZdq{}}\PY{l+s+s2}{NOUN}\PY{l+s+s2}{\PYZdq{}}\PY{p}{\PYZcb{}}\PY{p}{]}\PY{p}{]}
         
         \PY{c+c1}{\PYZsh{} Initialize the Matcher and add the patterns}
         \PY{n}{matcher} \PY{o}{=} \PY{n}{Matcher}\PY{p}{(}\PY{n}{nlp}\PY{o}{.}\PY{n}{vocab}\PY{p}{)}
         \PY{n}{matcher}\PY{o}{.}\PY{n}{add}\PY{p}{(}\PY{l+s+s2}{\PYZdq{}}\PY{l+s+s2}{PATTERN1}\PY{l+s+s2}{\PYZdq{}}\PY{p}{,} \PY{n}{pattern1}\PY{p}{)}
         \PY{n}{matcher}\PY{o}{.}\PY{n}{add}\PY{p}{(}\PY{l+s+s2}{\PYZdq{}}\PY{l+s+s2}{PATTERN2}\PY{l+s+s2}{\PYZdq{}}\PY{p}{,} \PY{n}{pattern2}\PY{p}{)}
         
         \PY{c+c1}{\PYZsh{} Iterate over the matches}
         \PY{k}{for} \PY{n}{match\PYZus{}id}\PY{p}{,} \PY{n}{start}\PY{p}{,} \PY{n}{end} \PY{o+ow}{in} \PY{n}{matcher}\PY{p}{(}\PY{n}{doc}\PY{p}{)}\PY{p}{:}
             \PY{c+c1}{\PYZsh{} Print pattern string name and text of matched span}
             \PY{n+nb}{print}\PY{p}{(}\PY{n}{doc}\PY{o}{.}\PY{n}{vocab}\PY{o}{.}\PY{n}{strings}\PY{p}{[}\PY{n}{match\PYZus{}id}\PY{p}{]}\PY{p}{,} \PY{n}{doc}\PY{p}{[}\PY{n}{start}\PY{p}{:}\PY{n}{end}\PY{p}{]}\PY{o}{.}\PY{n}{text}\PY{p}{)}
\end{Verbatim}


    \begin{Verbatim}[commandchars=\\\{\}]
PATTERN1 Amazon Prime
PATTERN2 ad-free viewing
PATTERN1 Amazon Prime
PATTERN2 ad-free viewing
PATTERN2 ad-free viewing
PATTERN2 ad-free viewing

    \end{Verbatim}

    \hypertarget{efficient-phrase-matching}{%
\subsubsection{2.8 Efficient phrase
matching}\label{efficient-phrase-matching}}

Sometimes it's more efficient to match exact strings instead of writing
patterns describing the individual tokens. This is especially true for
finite categories of things -- like all countries of the world. We
already have a list of countries, so let's use this as the basis of our
information extraction script. A list of string names is available as
the variable COUNTRIES.

\begin{itemize}
\tightlist
\item
  Import the PhraseMatcher and initialize it with the shared vocab as
  the variable matcher.
\item
  Add the phrase patterns and call the matcher on the doc.
\end{itemize}

    \begin{Verbatim}[commandchars=\\\{\}]
{\color{incolor}In [{\color{incolor}17}]:} \PY{k+kn}{import} \PY{n+nn}{json}
         \PY{k+kn}{from} \PY{n+nn}{spacy}\PY{n+nn}{.}\PY{n+nn}{lang}\PY{n+nn}{.}\PY{n+nn}{en} \PY{k}{import} \PY{n}{English}
         
         \PY{k}{with} \PY{n+nb}{open}\PY{p}{(}\PY{l+s+s2}{\PYZdq{}}\PY{l+s+s2}{data/countries.json}\PY{l+s+s2}{\PYZdq{}}\PY{p}{,} \PY{n}{encoding}\PY{o}{=}\PY{l+s+s2}{\PYZdq{}}\PY{l+s+s2}{utf8}\PY{l+s+s2}{\PYZdq{}}\PY{p}{)} \PY{k}{as} \PY{n}{f}\PY{p}{:}
             \PY{n}{COUNTRIES} \PY{o}{=} \PY{n}{json}\PY{o}{.}\PY{n}{loads}\PY{p}{(}\PY{n}{f}\PY{o}{.}\PY{n}{read}\PY{p}{(}\PY{p}{)}\PY{p}{)}
         
         \PY{n}{nlp} \PY{o}{=} \PY{n}{English}\PY{p}{(}\PY{p}{)}
         \PY{n}{doc} \PY{o}{=} \PY{n}{nlp}\PY{p}{(}\PY{l+s+s2}{\PYZdq{}}\PY{l+s+s2}{Czech Republic may help Slovakia protect its airspace}\PY{l+s+s2}{\PYZdq{}}\PY{p}{)}
         
         \PY{c+c1}{\PYZsh{} Import the PhraseMatcher and initialize it}
         \PY{k+kn}{from} \PY{n+nn}{spacy}\PY{n+nn}{.}\PY{n+nn}{matcher} \PY{k}{import} \PY{n}{PhraseMatcher}
         
         \PY{n}{matcher} \PY{o}{=} \PY{n}{PhraseMatcher}\PY{p}{(}\PY{n}{nlp}\PY{o}{.}\PY{n}{vocab}\PY{p}{)}
         
         \PY{c+c1}{\PYZsh{} Create pattern Doc objects and add them to the matcher}
         \PY{c+c1}{\PYZsh{} This is the faster version of: [nlp(country) for country in COUNTRIES]}
         \PY{n}{patterns} \PY{o}{=} \PY{n+nb}{list}\PY{p}{(}\PY{n}{nlp}\PY{o}{.}\PY{n}{pipe}\PY{p}{(}\PY{n}{COUNTRIES}\PY{p}{)}\PY{p}{)}
         \PY{c+c1}{\PYZsh{}print(*patterns)}
         
         \PY{n}{matcher}\PY{o}{.}\PY{n}{add}\PY{p}{(}\PY{l+s+s2}{\PYZdq{}}\PY{l+s+s2}{COUNTRY}\PY{l+s+s2}{\PYZdq{}}\PY{p}{,} \PY{p}{[}\PY{o}{*}\PY{n}{patterns}\PY{p}{]}\PY{p}{)}
         
         \PY{c+c1}{\PYZsh{} Call the matcher on the test document and print the result}
         \PY{n}{matches} \PY{o}{=} \PY{n}{matcher}\PY{p}{(}\PY{n}{doc}\PY{p}{)}
         \PY{n+nb}{print}\PY{p}{(}\PY{p}{[}\PY{n}{doc}\PY{p}{[}\PY{n}{start}\PY{p}{:}\PY{n}{end}\PY{p}{]} \PY{k}{for} \PY{n}{match\PYZus{}id}\PY{p}{,} \PY{n}{start}\PY{p}{,} \PY{n}{end} \PY{o+ow}{in} \PY{n}{matches}\PY{p}{]}\PY{p}{)}
\end{Verbatim}


    \begin{Verbatim}[commandchars=\\\{\}]
[Czech Republic, Slovakia]

    \end{Verbatim}

    \hypertarget{extracting-coutries-and-relationships}{%
\subsubsection{2.9 Extracting coutries and
relationships}\label{extracting-coutries-and-relationships}}

In the previous exercise, there was a script using spaCy's PhraseMatcher
to find country names in text. Let's use that country matcher on a
longer text, analyze the syntax and update the document's entities with
the matched countries.

\begin{itemize}
\tightlist
\item
  Iterate over the matches and create a Span with the label ``GPE''
  (geopolitical entity).
\item
  Overwrite the entities in doc.ents and add the matched span.
\item
  Get the matched span's root head token.
\item
  Print the text of the head token and the span.
\end{itemize}

\hypertarget{exercise-2.8-complete-the-code.}{%
\paragraph{Exercise 2.8: Complete the
code.}\label{exercise-2.8-complete-the-code.}}

    \begin{Verbatim}[commandchars=\\\{\}]
{\color{incolor}In [{\color{incolor}18}]:} \PY{k+kn}{import} \PY{n+nn}{spacy}
         \PY{k+kn}{from} \PY{n+nn}{spacy}\PY{n+nn}{.}\PY{n+nn}{matcher} \PY{k}{import} \PY{n}{PhraseMatcher}
         \PY{k+kn}{from} \PY{n+nn}{spacy}\PY{n+nn}{.}\PY{n+nn}{tokens} \PY{k}{import} \PY{n}{Span}
         \PY{k+kn}{import} \PY{n+nn}{json}
         
         \PY{k}{with} \PY{n+nb}{open}\PY{p}{(}\PY{l+s+s2}{\PYZdq{}}\PY{l+s+s2}{data/countries.json}\PY{l+s+s2}{\PYZdq{}}\PY{p}{,} \PY{n}{encoding}\PY{o}{=}\PY{l+s+s2}{\PYZdq{}}\PY{l+s+s2}{utf8}\PY{l+s+s2}{\PYZdq{}}\PY{p}{)} \PY{k}{as} \PY{n}{f}\PY{p}{:}
             \PY{n}{COUNTRIES} \PY{o}{=} \PY{n}{json}\PY{o}{.}\PY{n}{loads}\PY{p}{(}\PY{n}{f}\PY{o}{.}\PY{n}{read}\PY{p}{(}\PY{p}{)}\PY{p}{)}
         \PY{k}{with} \PY{n+nb}{open}\PY{p}{(}\PY{l+s+s2}{\PYZdq{}}\PY{l+s+s2}{data/country\PYZus{}text.txt}\PY{l+s+s2}{\PYZdq{}}\PY{p}{,} \PY{n}{encoding}\PY{o}{=}\PY{l+s+s2}{\PYZdq{}}\PY{l+s+s2}{utf8}\PY{l+s+s2}{\PYZdq{}}\PY{p}{)} \PY{k}{as} \PY{n}{f}\PY{p}{:}
             \PY{n}{TEXT} \PY{o}{=} \PY{n}{f}\PY{o}{.}\PY{n}{read}\PY{p}{(}\PY{p}{)}
         
         \PY{n+nb}{print}\PY{p}{(}\PY{n}{TEXT}\PY{p}{)}
             
         \PY{n}{nlp} \PY{o}{=} \PY{n}{spacy}\PY{o}{.}\PY{n}{load}\PY{p}{(}\PY{l+s+s2}{\PYZdq{}}\PY{l+s+s2}{en\PYZus{}core\PYZus{}web\PYZus{}sm}\PY{l+s+s2}{\PYZdq{}}\PY{p}{)}
         \PY{n}{matcher} \PY{o}{=} \PY{n}{PhraseMatcher}\PY{p}{(}\PY{n}{nlp}\PY{o}{.}\PY{n}{vocab}\PY{p}{)}
         \PY{n}{patterns} \PY{o}{=} \PY{n+nb}{list}\PY{p}{(}\PY{n}{nlp}\PY{o}{.}\PY{n}{pipe}\PY{p}{(}\PY{n}{COUNTRIES}\PY{p}{)}\PY{p}{)}
         \PY{n}{matcher}\PY{o}{.}\PY{n}{add}\PY{p}{(}\PY{l+s+s2}{\PYZdq{}}\PY{l+s+s2}{COUNTRY}\PY{l+s+s2}{\PYZdq{}}\PY{p}{,} \PY{k+kc}{None}\PY{p}{,} \PY{o}{*}\PY{n}{patterns}\PY{p}{)}
         
         \PY{c+c1}{\PYZsh{} Create a doc and reset existing entities}
         \PY{n}{doc} \PY{o}{=} \PY{n}{nlp}\PY{p}{(}\PY{n}{TEXT}\PY{p}{)}
         \PY{n}{doc}\PY{o}{.}\PY{n}{ents} \PY{o}{=} \PY{p}{[}\PY{p}{]}
         
         \PY{c+c1}{\PYZsh{} Exercise 2.8: Complete the code.}
         \PY{c+c1}{\PYZsh{} \PYZhy{}\PYZhy{}\PYZhy{}\PYZhy{}\PYZhy{}\PYZhy{}\PYZhy{}\PYZhy{}\PYZhy{}\PYZhy{}\PYZhy{}\PYZhy{}\PYZhy{}\PYZhy{}\PYZhy{}\PYZhy{}}
         
         \PY{c+c1}{\PYZsh{} Iterate over the matches}
         \PY{k}{for} \PY{n}{match\PYZus{}id}\PY{p}{,} \PY{n}{start}\PY{p}{,} \PY{n}{end} \PY{o+ow}{in} \PY{n}{matcher}\PY{p}{(}\PY{n}{doc}\PY{p}{)}\PY{p}{:}
             \PY{c+c1}{\PYZsh{} Create a Span with the label for \PYZdq{}GPE\PYZdq{}}
             \PY{n}{span} \PY{o}{=} \PY{n}{Span}\PY{p}{(}\PY{n}{doc}\PY{p}{,} \PY{n}{start}\PY{p}{,} \PY{n}{end}\PY{p}{,} \PY{n}{label}\PY{o}{=}\PY{l+s+s2}{\PYZdq{}}\PY{l+s+s2}{GPE}\PY{l+s+s2}{\PYZdq{}}\PY{p}{)}
         
             \PY{c+c1}{\PYZsh{} Overwrite the doc.ents and add the span}
             \PY{n}{doc}\PY{o}{.}\PY{n}{ents} \PY{o}{=} \PY{n+nb}{list}\PY{p}{(}\PY{n}{doc}\PY{o}{.}\PY{n}{ents}\PY{p}{)} \PY{o}{+} \PY{p}{[}\PY{n}{span}\PY{p}{]}
         
             \PY{c+c1}{\PYZsh{} Get the span\PYZsq{}s root head token}
             \PY{n}{span\PYZus{}root\PYZus{}head} \PY{o}{=} \PY{n}{span}\PY{o}{.}\PY{n}{root}\PY{o}{.}\PY{n}{head}
             \PY{c+c1}{\PYZsh{} Print the text of the span root\PYZsq{}s head token and the span text}
             \PY{n+nb}{print}\PY{p}{(}\PY{n}{span\PYZus{}root\PYZus{}head}\PY{o}{.}\PY{n}{text}\PY{p}{,} \PY{l+s+s2}{\PYZdq{}}\PY{l+s+s2}{\PYZhy{}\PYZhy{}\PYZgt{}}\PY{l+s+s2}{\PYZdq{}}\PY{p}{,} \PY{n}{span}\PY{o}{.}\PY{n}{text}\PY{p}{)}
         
         \PY{c+c1}{\PYZsh{} Print the entities in the document}
         \PY{n+nb}{print}\PY{p}{(}\PY{p}{[}\PY{p}{(}\PY{n}{ent}\PY{o}{.}\PY{n}{text}\PY{p}{,} \PY{n}{ent}\PY{o}{.}\PY{n}{label\PYZus{}}\PY{p}{)} \PY{k}{for} \PY{n}{ent} \PY{o+ow}{in} \PY{n}{doc}\PY{o}{.}\PY{n}{ents} \PY{k}{if} \PY{n}{ent}\PY{o}{.}\PY{n}{label\PYZus{}} \PY{o}{==} \PY{l+s+s2}{\PYZdq{}}\PY{l+s+s2}{GPE}\PY{l+s+s2}{\PYZdq{}}\PY{p}{]}\PY{p}{)}
         \PY{n+nb}{print}\PY{p}{(}\PY{l+s+s2}{\PYZdq{}}\PY{l+s+s2}{GPE count:}\PY{l+s+s2}{\PYZdq{}}\PY{p}{,} \PY{n+nb}{len}\PY{p}{(}\PY{p}{[}\PY{p}{(}\PY{n}{ent}\PY{o}{.}\PY{n}{text}\PY{p}{,} \PY{n}{ent}\PY{o}{.}\PY{n}{label\PYZus{}}\PY{p}{)} \PY{k}{for} \PY{n}{ent} \PY{o+ow}{in} \PY{n}{doc}\PY{o}{.}\PY{n}{ents} \PY{k}{if} \PY{n}{ent}\PY{o}{.}\PY{n}{label\PYZus{}} \PY{o}{==} \PY{l+s+s2}{\PYZdq{}}\PY{l+s+s2}{GPE}\PY{l+s+s2}{\PYZdq{}}\PY{p}{]}\PY{p}{)}\PY{p}{)}
\end{Verbatim}


    \begin{Verbatim}[commandchars=\\\{\}]
After the Cold War, the UN saw a radical expansion in its peacekeeping duties, taking on more missions in ten years than it had in the previous four decades.Between 1988 and 2000, the number of adopted Security Council resolutions more than doubled, and the peacekeeping budget increased more than tenfold. The UN negotiated an end to the Salvadoran Civil War, launched a successful peacekeeping mission in Namibia, and oversaw democratic elections in post-apartheid South Africa and post-Khmer Rouge Cambodia. In 1991, the UN authorized a US-led coalition that repulsed the Iraqi invasion of Kuwait. Brian Urquhart, Under-Secretary-General from 1971 to 1985, later described the hopes raised by these successes as a "false renaissance" for the organization, given the more troubled missions that followed. Though the UN Charter had been written primarily to prevent aggression by one nation against another, in the early 1990s the UN faced a number of simultaneous, serious crises within nations such as Somalia, Haiti, Mozambique, and the former Yugoslavia. The UN mission in Somalia was widely viewed as a failure after the US withdrawal following casualties in the Battle of Mogadishu, and the UN mission to Bosnia faced "worldwide ridicule" for its indecisive and confused mission in the face of ethnic cleansing. In 1994, the UN Assistance Mission for Rwanda failed to intervene in the Rwandan genocide amid indecision in the Security Council. Beginning in the last decades of the Cold War, American and European critics of the UN condemned the organization for perceived mismanagement and corruption. In 1984, the US President, Ronald Reagan, withdrew his nation's funding from UNESCO (the United Nations Educational, Scientific and Cultural Organization, founded 1946) over allegations of mismanagement, followed by Britain and Singapore. Boutros Boutros-Ghali, Secretary-General from 1992 to 1996, initiated a reform of the Secretariat, reducing the size of the organization somewhat. His successor, Kofi Annan (1997–2006), initiated further management reforms in the face of threats from the United States to withhold its UN dues. In the late 1990s and 2000s, international interventions authorized by the UN took a wider variety of forms. The UN mission in the Sierra Leone Civil War of 1991–2002 was supplemented by British Royal Marines, and the invasion of Afghanistan in 2001 was overseen by NATO. In 2003, the United States invaded Iraq despite failing to pass a UN Security Council resolution for authorization, prompting a new round of questioning of the organization's effectiveness. Under the eighth Secretary-General, Ban Ki-moon, the UN has intervened with peacekeepers in crises including the War in Darfur in Sudan and the Kivu conflict in the Democratic Republic of Congo and sent observers and chemical weapons inspectors to the Syrian Civil War. In 2013, an internal review of UN actions in the final battles of the Sri Lankan Civil War in 2009 concluded that the organization had suffered "systemic failure". One hundred and one UN personnel died in the 2010 Haiti earthquake, the worst loss of life in the organization's history. The Millennium Summit was held in 2000 to discuss the UN's role in the 21st century. The three day meeting was the largest gathering of world leaders in history, and culminated in the adoption by all member states of the Millennium Development Goals (MDGs), a commitment to achieve international development in areas such as poverty reduction, gender equality, and public health. Progress towards these goals, which were to be met by 2015, was ultimately uneven. The 2005 World Summit reaffirmed the UN's focus on promoting development, peacekeeping, human rights, and global security. The Sustainable Development Goals were launched in 2015 to succeed the Millennium Development Goals. In addition to addressing global challenges, the UN has sought to improve its accountability and democratic legitimacy by engaging more with civil society and fostering a global constituency. In an effort to enhance transparency, in 2016 the organization held its first public debate between candidates for Secretary-General. On 1 January 2017, Portuguese diplomat António Guterres, who previously served as UN High Commissioner for Refugees, became the ninth Secretary-General. Guterres has highlighted several key goals for his administration, including an emphasis on diplomacy for preventing conflicts, more effective peacekeeping efforts, and streamlining the organization to be more responsive and versatile to global needs.

in --> Namibia
in --> South Africa
Africa --> Cambodia
of --> Kuwait
as --> Somalia
Somalia --> Haiti
Haiti --> Mozambique
in --> Somalia
for --> Rwanda
Britain --> Singapore
War --> Sierra Leone
of --> Afghanistan
invaded --> Iraq
in --> Sudan
of --> Congo
earthquake --> Haiti
[('Namibia', 'GPE'), ('South Africa', 'GPE'), ('Cambodia', 'GPE'), ('Kuwait', 'GPE'), ('Somalia', 'GPE'), ('Haiti', 'GPE'), ('Mozambique', 'GPE'), ('Somalia', 'GPE'), ('Rwanda', 'GPE'), ('Singapore', 'GPE'), ('Sierra Leone', 'GPE'), ('Afghanistan', 'GPE'), ('Iraq', 'GPE'), ('Sudan', 'GPE'), ('Congo', 'GPE'), ('Haiti', 'GPE')]
GPE count: 16

    \end{Verbatim}

    \textbf{Hint.} You should get print:

\begin{verbatim}
in --> Namibia
in --> South Africa
Africa --> Cambodia
of --> Kuwait
as --> Somalia
Somalia --> Haiti
Haiti --> Mozambique
in --> Somalia
for --> Rwanda
Britain --> Singapore
War --> Sierra Leone
of --> Afghanistan
invaded --> Iraq
in --> Sudan
of --> Congo
earthquake --> Haiti
[('Namibia', 'GPE'), ('South Africa', 'GPE'), ('Cambodia', 'GPE'), ('Kuwait', 'GPE'), ('Somalia', 'GPE'), ('Haiti', 'GPE'), ('Mozambique', 'GPE'), ('Somalia', 'GPE'), ('Rwanda', 'GPE'), ('Singapore', 'GPE'), ('Sierra Leone', 'GPE'), ('Afghanistan', 'GPE'), ('Iraq', 'GPE'), ('Sudan', 'GPE'), ('Congo', 'GPE'), ('Haiti', 'GPE')]
\end{verbatim}

    \hypertarget{exercise-2.9}{%
\paragraph{Exercise 2.9:}\label{exercise-2.9}}

Do the same as in exercise 2.8, but use directly
\texttt{en\_core\_web\_sm} entities. Does it find same amount of GPEs as
code in exercise 2.8?

    \begin{Verbatim}[commandchars=\\\{\}]
{\color{incolor}In [{\color{incolor}19}]:} \PY{k+kn}{import} \PY{n+nn}{spacy}
         \PY{k+kn}{from} \PY{n+nn}{spacy}\PY{n+nn}{.}\PY{n+nn}{matcher} \PY{k}{import} \PY{n}{PhraseMatcher}
         \PY{k+kn}{from} \PY{n+nn}{spacy}\PY{n+nn}{.}\PY{n+nn}{tokens} \PY{k}{import} \PY{n}{Span}
         \PY{k+kn}{import} \PY{n+nn}{json}
         
         \PY{k}{with} \PY{n+nb}{open}\PY{p}{(}\PY{l+s+s2}{\PYZdq{}}\PY{l+s+s2}{data/countries.json}\PY{l+s+s2}{\PYZdq{}}\PY{p}{,} \PY{n}{encoding}\PY{o}{=}\PY{l+s+s2}{\PYZdq{}}\PY{l+s+s2}{utf8}\PY{l+s+s2}{\PYZdq{}}\PY{p}{)} \PY{k}{as} \PY{n}{f}\PY{p}{:}
             \PY{n}{COUNTRIES} \PY{o}{=} \PY{n}{json}\PY{o}{.}\PY{n}{loads}\PY{p}{(}\PY{n}{f}\PY{o}{.}\PY{n}{read}\PY{p}{(}\PY{p}{)}\PY{p}{)}
         \PY{k}{with} \PY{n+nb}{open}\PY{p}{(}\PY{l+s+s2}{\PYZdq{}}\PY{l+s+s2}{data/country\PYZus{}text.txt}\PY{l+s+s2}{\PYZdq{}}\PY{p}{,} \PY{n}{encoding}\PY{o}{=}\PY{l+s+s2}{\PYZdq{}}\PY{l+s+s2}{utf8}\PY{l+s+s2}{\PYZdq{}}\PY{p}{)} \PY{k}{as} \PY{n}{f}\PY{p}{:}
             \PY{n}{TEXT} \PY{o}{=} \PY{n}{f}\PY{o}{.}\PY{n}{read}\PY{p}{(}\PY{p}{)}
         
         \PY{n+nb}{print}\PY{p}{(}\PY{n}{TEXT}\PY{p}{)}
         
         \PY{c+c1}{\PYZsh{} Your code here:}
         \PY{c+c1}{\PYZsh{} \PYZhy{}\PYZhy{}\PYZhy{}\PYZhy{}\PYZhy{}\PYZhy{}\PYZhy{}\PYZhy{}\PYZhy{}\PYZhy{}\PYZhy{}\PYZhy{}\PYZhy{}\PYZhy{}\PYZhy{}\PYZhy{}\PYZhy{}\PYZhy{}}
         \PY{n}{nlp} \PY{o}{=} \PY{n}{spacy}\PY{o}{.}\PY{n}{load}\PY{p}{(}\PY{l+s+s2}{\PYZdq{}}\PY{l+s+s2}{en\PYZus{}core\PYZus{}web\PYZus{}sm}\PY{l+s+s2}{\PYZdq{}}\PY{p}{)}
         \PY{n}{doc} \PY{o}{=} \PY{n}{nlp}\PY{p}{(}\PY{n}{TEXT}\PY{p}{)}
         \PY{n}{ent} \PY{o}{=} \PY{n}{doc}\PY{o}{.}\PY{n}{ents}
         
         \PY{c+c1}{\PYZsh{} Print the entities in the document}
         \PY{n+nb}{print}\PY{p}{(}\PY{p}{[}\PY{p}{(}\PY{n}{ent}\PY{o}{.}\PY{n}{text}\PY{p}{,} \PY{n}{ent}\PY{o}{.}\PY{n}{label\PYZus{}}\PY{p}{)} \PY{k}{for} \PY{n}{ent} \PY{o+ow}{in} \PY{n}{doc}\PY{o}{.}\PY{n}{ents} \PY{k}{if} \PY{n}{ent}\PY{o}{.}\PY{n}{label\PYZus{}} \PY{o}{==} \PY{l+s+s2}{\PYZdq{}}\PY{l+s+s2}{GPE}\PY{l+s+s2}{\PYZdq{}}\PY{p}{]}\PY{p}{)}
         \PY{n+nb}{print}\PY{p}{(}\PY{l+s+s2}{\PYZdq{}}\PY{l+s+s2}{GPE count:}\PY{l+s+s2}{\PYZdq{}}\PY{p}{,} \PY{n+nb}{len}\PY{p}{(}\PY{p}{[}\PY{p}{(}\PY{n}{ent}\PY{o}{.}\PY{n}{text}\PY{p}{,} \PY{n}{ent}\PY{o}{.}\PY{n}{label\PYZus{}}\PY{p}{)} \PY{k}{for} \PY{n}{ent} \PY{o+ow}{in} \PY{n}{doc}\PY{o}{.}\PY{n}{ents} \PY{k}{if} \PY{n}{ent}\PY{o}{.}\PY{n}{label\PYZus{}} \PY{o}{==} \PY{l+s+s2}{\PYZdq{}}\PY{l+s+s2}{GPE}\PY{l+s+s2}{\PYZdq{}}\PY{p}{]}\PY{p}{)}\PY{p}{)}
\end{Verbatim}


    \begin{Verbatim}[commandchars=\\\{\}]
After the Cold War, the UN saw a radical expansion in its peacekeeping duties, taking on more missions in ten years than it had in the previous four decades.Between 1988 and 2000, the number of adopted Security Council resolutions more than doubled, and the peacekeeping budget increased more than tenfold. The UN negotiated an end to the Salvadoran Civil War, launched a successful peacekeeping mission in Namibia, and oversaw democratic elections in post-apartheid South Africa and post-Khmer Rouge Cambodia. In 1991, the UN authorized a US-led coalition that repulsed the Iraqi invasion of Kuwait. Brian Urquhart, Under-Secretary-General from 1971 to 1985, later described the hopes raised by these successes as a "false renaissance" for the organization, given the more troubled missions that followed. Though the UN Charter had been written primarily to prevent aggression by one nation against another, in the early 1990s the UN faced a number of simultaneous, serious crises within nations such as Somalia, Haiti, Mozambique, and the former Yugoslavia. The UN mission in Somalia was widely viewed as a failure after the US withdrawal following casualties in the Battle of Mogadishu, and the UN mission to Bosnia faced "worldwide ridicule" for its indecisive and confused mission in the face of ethnic cleansing. In 1994, the UN Assistance Mission for Rwanda failed to intervene in the Rwandan genocide amid indecision in the Security Council. Beginning in the last decades of the Cold War, American and European critics of the UN condemned the organization for perceived mismanagement and corruption. In 1984, the US President, Ronald Reagan, withdrew his nation's funding from UNESCO (the United Nations Educational, Scientific and Cultural Organization, founded 1946) over allegations of mismanagement, followed by Britain and Singapore. Boutros Boutros-Ghali, Secretary-General from 1992 to 1996, initiated a reform of the Secretariat, reducing the size of the organization somewhat. His successor, Kofi Annan (1997–2006), initiated further management reforms in the face of threats from the United States to withhold its UN dues. In the late 1990s and 2000s, international interventions authorized by the UN took a wider variety of forms. The UN mission in the Sierra Leone Civil War of 1991–2002 was supplemented by British Royal Marines, and the invasion of Afghanistan in 2001 was overseen by NATO. In 2003, the United States invaded Iraq despite failing to pass a UN Security Council resolution for authorization, prompting a new round of questioning of the organization's effectiveness. Under the eighth Secretary-General, Ban Ki-moon, the UN has intervened with peacekeepers in crises including the War in Darfur in Sudan and the Kivu conflict in the Democratic Republic of Congo and sent observers and chemical weapons inspectors to the Syrian Civil War. In 2013, an internal review of UN actions in the final battles of the Sri Lankan Civil War in 2009 concluded that the organization had suffered "systemic failure". One hundred and one UN personnel died in the 2010 Haiti earthquake, the worst loss of life in the organization's history. The Millennium Summit was held in 2000 to discuss the UN's role in the 21st century. The three day meeting was the largest gathering of world leaders in history, and culminated in the adoption by all member states of the Millennium Development Goals (MDGs), a commitment to achieve international development in areas such as poverty reduction, gender equality, and public health. Progress towards these goals, which were to be met by 2015, was ultimately uneven. The 2005 World Summit reaffirmed the UN's focus on promoting development, peacekeeping, human rights, and global security. The Sustainable Development Goals were launched in 2015 to succeed the Millennium Development Goals. In addition to addressing global challenges, the UN has sought to improve its accountability and democratic legitimacy by engaging more with civil society and fostering a global constituency. In an effort to enhance transparency, in 2016 the organization held its first public debate between candidates for Secretary-General. On 1 January 2017, Portuguese diplomat António Guterres, who previously served as UN High Commissioner for Refugees, became the ninth Secretary-General. Guterres has highlighted several key goals for his administration, including an emphasis on diplomacy for preventing conflicts, more effective peacekeeping efforts, and streamlining the organization to be more responsive and versatile to global needs.

[('Namibia', 'GPE'), ('South Africa', 'GPE'), ('US', 'GPE'), ('Kuwait', 'GPE'), ('Somalia', 'GPE'), ('Haiti', 'GPE'), ('Yugoslavia', 'GPE'), ('Somalia', 'GPE'), ('US', 'GPE'), ('the Battle of Mogadishu', 'GPE'), ('Bosnia', 'GPE'), ('US', 'GPE'), ('Britain', 'GPE'), ('Singapore', 'GPE'), ('the United States', 'GPE'), ('Afghanistan', 'GPE'), ('the United States', 'GPE'), ('Iraq', 'GPE'), ('Sudan', 'GPE'), ('the Democratic Republic of Congo', 'GPE'), ('Haiti', 'GPE'), ('Guterres', 'GPE')]
GPE count: 22

    \end{Verbatim}

    \hypertarget{reflection}{%
\section{Reflection}\label{reflection}}

\begin{enumerate}
\def\labelenumi{\arabic{enumi}.}
\tightlist
\item
  Print Finnish stop word. Is it longer or shorter than English stop
  word list?
\item
  What is \texttt{"word\ vector"}? And how many features spaCy
  \texttt{"word\ vector"} has?
\item
  What is ``bag of words''?
\item
  What is lemma? What about lemmatization?
\item
  Why spaCy does not use stemming?
\end{enumerate}

    \emph{Your answers here\ldots{}}

    \begin{Verbatim}[commandchars=\\\{\}]
{\color{incolor}In [{\color{incolor}20}]:} \PY{k+kn}{from} \PY{n+nn}{spacy}\PY{n+nn}{.}\PY{n+nn}{lang}\PY{n+nn}{.}\PY{n+nn}{fi} \PY{k}{import} \PY{n}{Finnish}
         \PY{n+nb}{print}\PY{p}{(}\PY{l+s+s1}{\PYZsq{}}\PY{l+s+s1}{spaCy Version: }\PY{l+s+si}{\PYZpc{}s}\PY{l+s+s1}{\PYZsq{}} \PY{o}{\PYZpc{}} \PY{p}{(}\PY{n}{spacy}\PY{o}{.}\PY{n}{\PYZus{}\PYZus{}version\PYZus{}\PYZus{}}\PY{p}{)}\PY{p}{)}
         \PY{n}{english\PYZus{}stopwords} \PY{o}{=} \PY{n}{spacy}\PY{o}{.}\PY{n}{lang}\PY{o}{.}\PY{n}{en}\PY{o}{.}\PY{n}{stop\PYZus{}words}\PY{o}{.}\PY{n}{STOP\PYZus{}WORDS}
         \PY{n}{finnish\PYZus{}stopwords} \PY{o}{=} \PY{n}{spacy}\PY{o}{.}\PY{n}{lang}\PY{o}{.}\PY{n}{fi}\PY{o}{.}\PY{n}{stop\PYZus{}words}\PY{o}{.}\PY{n}{STOP\PYZus{}WORDS}
         \PY{n+nb}{print}\PY{p}{(}\PY{l+s+s1}{\PYZsq{}}\PY{l+s+s1}{Number of english stop words: }\PY{l+s+si}{\PYZpc{}d}\PY{l+s+s1}{\PYZsq{}} \PY{o}{\PYZpc{}} \PY{n+nb}{len}\PY{p}{(}\PY{n}{english\PYZus{}stopwords}\PY{p}{)}\PY{p}{)}
         \PY{n+nb}{print}\PY{p}{(}\PY{l+s+s1}{\PYZsq{}}\PY{l+s+s1}{Number of finnish stop words: }\PY{l+s+si}{\PYZpc{}d}\PY{l+s+s1}{\PYZsq{}} \PY{o}{\PYZpc{}} \PY{n+nb}{len}\PY{p}{(}\PY{n}{finnish\PYZus{}stopwords}\PY{p}{)}\PY{p}{)}
\end{Verbatim}


    \begin{Verbatim}[commandchars=\\\{\}]
spaCy Version: 3.0.5
Number of english stop words: 326
Number of finnish stop words: 822

    \end{Verbatim}

    \begin{enumerate}
\def\labelenumi{\arabic{enumi}.}
\item
  Ylempää nähdään, että suomalaisia stop wordeja on enemmän.
\item
  Sanavektori kuvaa sanojen ominaisuuksia ja se koostuu pienistä
  luvuista. Koneiden on helpompi käsitellä sanavektoreita ja tunnistaa
  siitä sanojen ominaisuudet. Sanavektori on 300 ulotteinen, mutta se
  voidaan muuttaa kaksiulotteiseksi käyttämällä PCA:ta.
\item
  Luonnollisen kielen käsittelyssä käytettävä menetelmä. Siinä muutetaan
  teksti ns. osiin ja tarkastellaan kunkin sanan esiintymistiheyttä ja
  luokkaa. Tämän avulla voidaan totteuttaa koneoppimista ja
  luokittetelua myöhemmin datan (tekstin) analysoimisessa.
\item
  Lemma on sanan perusmuoto. Lemmatilazion on taas yksi NLP:ssä datan
  (tekstin) käsittelyssä käytettävä menetelmä. Se palauttaa sanan
  perusmuodon, jotta koneoppivat algoritmin osaisivat käsitellä tekstiä
  paremmin.
\item
  SpaCy käyttää vain lemmatizingiä. Varmaan siksi, koska lemmatizing on
  parempi algoritmi ajamaan saman asian.
\end{enumerate}


    % Add a bibliography block to the postdoc
    
    
    
    \end{document}
